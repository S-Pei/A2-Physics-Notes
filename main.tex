\documentclass{article}
\usepackage{xeCJK}
\usepackage[utf8]{inputenc}
\usepackage{amsmath, amsthm}
\usepackage{amssymb}
\usepackage{amsfonts}
\usepackage{fancyhdr}
\usepackage{pgfplots}
\usepackage{tikz}
\usepackage{float}
\usepackage{subcaption}
\usepackage[margin=20mm]{geometry}
\usepackage{multirow}
\usepackage{booktabs}
\newcommand{\tabitem}{~~\llap{\textbullet}~~}
\usepackage{tabularx}
\usepackage{ragged2e}
\usepackage[skip=0.333\baselineskip]{caption}
\usepackage{array}
\usepackage{multirow}
\usepackage{booktabs}
\usepackage{longtable}
\usepackage{background}
\usepackage{comment}
\usepackage{setspace}
\usepackage{qrcode}
\usepackage{graphicx}
\usepackage{tcolorbox}
\usepackage{hyperref}
\newtheorem{theorem}{Theorem}[subsection]
\hypersetup{
    colorlinks=true
    bookmarks=true
    pdftitle={Physics A2 notes}
}
\usepackage{fancyhdr}
\pagestyle{fancy}
\lhead{\textit{A2 Physics Notes}}
\rhead{\textit{銘蓓}}
\fancyhead[C]{\leftmark}

\pagenumbering{arabic}

\backgroundsetup{contents=銘蓓, opacity=0.3, color= gray, angle=30}

\title{
\textbf{Physics}
}
\date{}
\onehalfspacing
\begin{document}



\NoBgThispage
\input{titlepage}


\newpage
\begin{flushleft}
\textbf{From the Authors}
\end{flushleft}

Hey there! If you've bought these notes, chances are, there are some things you don't understand in A-level, and would like to. Just like what we wanted during our time in A-level. These are notes, and definitely are not going to serve as your textbook. \\

These notes cater for those who want a shortcut to understand, those who want a condensed version of what's important in A2 Physics, and those who really want to know what on earth the formulae are talking about. We, the authors, needed all 3, after all, purely understanding the topics will not guarantee you a high score in A-level.\\

Combined, we've been through years of A-level past year, and continuously updated our definition list as well as important points in past years that we don't see in textbooks. All this so that we could revise for our A2. Hence, we think it would be of use to you as well.\\

These notes include quite a lot of derivations of formulae, which might get a tad bit too mathematical for some. Hence, we've colour-coded them, and if you think those derivations are too much, it's okay to ignore them, they won't appear in exams. However, if we needed those derivations and explanations, we believe there are those who do as well. Enjoy! A lot of work, including lessons by our beloved Physics lecturer, going through the A-level textbook and tons of Googling, have gone into these derivations. If you disagree with any of our derivations, don't hesitate to contact us. We definitely look forward to any discussions. After all, we're all here to learn.\\

We tried to really understand what we were learning, instead of purely learning how to answer past year questions. As students, we understand that most of the time we need somebody to teach us through 'story-telling', with different wordings from the perfectly-crafted textbooks. When there were nobody to tell us those stories, we googled through materials that were beyond A-level syllabus, put hours and days of thought to it, then told ourselves those stories in terms of mathematical equations. We've done the work for you, so these are the 'stories' we left for you.\\  

Last but not least, we hope these notes are of some help to you. Do contact us if you have any feedback regarding these notes, we'll be happy to hear them. 
\\\\\\

\begin{tcolorbox}
\begin{flushleft}
Key:\\
\textcolor{blue}{Blue $\rightarrow$ These are derivations that might get too mathematical for some. It is for a more comprehensive understanding. Those who aren't really interested in knowing more can ignore.}\\
\textcolor{red}{Red $\rightarrow$ Very very important stuff from past years, they're basically marking schemes.}
\end{flushleft}
\end{tcolorbox}

\begin{flushleft}
The official website of the notes :D\\
\end{flushleft}

\qrcode{https://peizzz.wixsite.com/a2-physics-notes}

\newpage

\maketitle


\tableofcontents

\newpage

\section{Circular Motion}


\subsection{Concepts and Formulae Derivation}


    Centripetal force is the net force acted on an object towards its centre of rotation. The object is accelerating due to this net force, but the speed of the object remains constant(assuming the acceleration is constant), the acceleration caused by the centripetal force only affects the instantaneous direction of motion. We define the velocity of covering a certain angle by angular velocity, $\omega$.
    $$\omega=\cfrac{\Delta \theta}{\Delta t}=\cfrac{2\pi}{T}$$
    where $T$ is the period (time taken for 1 complete revolution).
    
    The tangential speed, denoted as $v$, is defined as 
    $$v=\cfrac{\text{arc length traveled}}{\text{time taken}}=\cfrac{2\pi r }{T}=r\omega$$
    
    To calculate centripetal force, we can use Newton's 2nd law, $F=ma$, the acceleration towards the centre, $a$, is given by 
    $$a=\cfrac{v^2}{r}=r\omega^2$$ 
    
    \begin{theorem}
    The centripetal acceleration is defined by $a=\cfrac{v^2}{r}=r\omega^2$
    \end{theorem}
    
    \begin{proof}
    
    \begin{figure}[H]
        \centering
        \captionsetup{justification=centering,margin=2cm}
        \includegraphics[scale=.2]{centripetal.jpg}
         \end{figure}
   
    When $\Delta \theta \rightarrow 0$, $\Delta \vec{v}\rightarrow0$ which means 
    $$\vec{v}_1\approx \vec{v}_2=\vec{v} \implies |\Delta \vec{v}| \approx |\vec{v} \Delta \theta| =|\vec{v}|\Delta \theta\quad (\text{since $\Delta \theta$ is just a scalar multiple, we factor it out of the modulus})$$
    This yields the following relation (essentially comparing the $\triangle OAB$ and $\triangle SPQ$ since both triangle are similar, i.e $\triangle OAB \sim \triangle SPQ$)
    $$\cfrac{|\Delta \vec{v}|}{\stackrel\frown{AB}}=\cfrac{|\vec{v}\Delta \theta|}{r\Delta \theta}=\cfrac{|\vec{v}|}{r}=\cfrac{v}{r}$$
    where $v$ denotes the magnitude(non-vector) of the velocity, $|\vec{v}|$. Assuming it takes a small amount of time $\Delta t$ to cover the small angle $\theta$ as it approaches $0$, we can say that $\stackrel\frown{AB}=|\vec{v}|\Delta t=v\Delta t$, we can deduce the following equation
    $$\cfrac{|\Delta \vec{v}|}{v\Delta t}=\cfrac{v}{r}\implies \cfrac{|\Delta \vec{v}|}{\Delta t}=\cfrac{{\vec{v}}^{\; 2}}{r}=|\vec{a}|=a$$
    \newpage
    We can also multiply the angular velocity, $\omega$, by the magnitude velocity, $|\vec{v}|$, to get the magnitude of acceleration, $|\vec{a}|$.
    $$\omega |\vec{v}|=\cfrac{|\vec{v}|\Delta \theta}{\Delta t}=\cfrac{|\vec{v}\Delta \theta|}{\Delta t}=\cfrac{|\Delta \vec{v}|}{\Delta t}=|\vec{a}|=a$$
    since $v=r\omega$, substituting this into the result above, we get
    $$a=\omega|\vec{v}|=\omega v=r\omega^2$$
    
    Therefore with acceleration figured out, we can find the centripetal force $F_c$ using Newton's 2nd law
    $$F_c=ma=m\cfrac{{\vec{v}}^{\; 2}}{r}=mr\omega^2$$
   \end{proof} 
    \subsection{Horizontal Rotation}
   
    
\begin{figure}[H]
    \centering
    \includegraphics[scale=.17]{horizontal 1.png}
    
\end{figure}
\begin{align*}
    T \cos \theta &=mg \\ T\sin \theta &= F_c= m\cfrac{\vec{v}^{\; 2}}{r}=mr\omega^2
\end{align*}

\newpage

  \subsection{Vertical Rotation}

 \begin{figure}[H]
 \centering
 \includegraphics[scale=.2]{circle 1.png}
 \end{figure}
If the speed of the objects moving in the circle path is constant, then 
$$T_1=T_2-mg=T_3-mg \cos\theta=T_4+mg=T_5+mg \cos \theta=m\cfrac{\vec{v}^{\; 2}}{r}$$



     \subsection{Vehicle on a Banked Road}


\begin{figure}[H]
\centering
\includegraphics[scale=.2]{banked 1.jpg}
\end{figure}

$$N=mg \cos \alpha $$
$$\vec{F}_c= N\sin \alpha=m\cfrac{\vec{v}^{\; 2}}{r}$$

Since weight of the car is constant, therefore $F_c$, is constant, as a result if the velocity of the vehicle, $\vec{v}$, is increased, $r$, will so increase which implies that the car will move up the slope to increase the radius of circular rotation so that the constant is not changed.
\newpage
\section{Gravitational Field}

     \subsection{Concepts}


In A-level syllabus, direction of forces in this chapter are usually disregarded, but in the notes here, directions will be taken into account to provide better understanding of the mathematics behind it. For convention, all the displacement is measured relative to the body that is creating the gravitational field that we are considering. 

If an object has mass, $M$, it will create a gravitational field around it which expands to infinity. The strength of the gravitational field generate by mass $M$, is denoted by $\vec{g}$, which can be defined by
$$\vec{g}=-\cfrac{GM}{\vec{r}\;^2}$$
where, $G$, is universal gravitational constant $(\approx6.67\times 10^{-11}\; m^3kg^{-1}s^{-2})$, and $\vec{r}$ is the displacement between the centre of mass of object (provided that the it is a body with uniform density) that generates the gravitational field and the point of measuring its strength. To understand why negative sign is present, we first have to define where the positive direction of the displacement will be, for simplicity's sake, we define the positive direction of displacement to be anywhere away from the centre of mass of the body creating the gravitational field. Since  the direction gravitational field is towards the centre of body creating it, therefore the direction force is in opposite direction to the direction of positive displacement, hence negative sign is needed to indicate the direction.

When two objects are placed into each others' gravitational fields, both of them will experience a pulling force from each other

\begin{figure}[H]
    \centering
    \includegraphics[scale=.25]{grav.jpg}
\end{figure}
Let the displacement between $m_1$ and $m_2$ from both reference point be $\vec{R}$, then the strength of gravitational field experienced at $m_1$ due to $m_2$ will be $-\cfrac{Gm_2}{\vec{R}\;^2}$ and the strength of gravitational field experienced at $m_2$ due to $m_1$ will be $-\cfrac{Gm_1}{\vec{R}\;^2}$, although $\big|-\cfrac{Gm_2}{\vec{R}\;^2}\big|<\big|-\cfrac{Gm_1}{\vec{R}\;^2}\big|$ but since $m_1>m_2$, therefore the two forces experienced by both are in fact the same in \textbf{magnitude}.
$$|\vec{F_1}|=F_1=\cfrac{Gm_2}{\vec{R}\;^2}m_1=\cfrac{Gm_1}{\vec{R}\;^2}m_2=F_2=|\vec{F_2}|$$
This is also known as Newton's Law of Universal Gravitation.

Note that gravitational field strength is a vector, hence that means in a two body system they will be a point between the two masses where the gravitational vector field add up to \textbf{zero}. But this does not implies that the point in that particular space has no gravitational potential, this is because gravitational potential is a scalar quantity hence they do not cancel out. Another definition of the gravitational field strength will also be derived in the next section when we talk about gravitational potential.



\textcolor{blue}{In A-level syllabus we will only be concerning with the gravitational force and gravitational field strength inside a perfectly spherical and uniform density object, usually a planet. \textbf{The derivation of the following equation is beyond A-level syllabus}.}

\begin{theorem}
\textcolor{blue}{The gravitational field strength inside a planet is directly proportional to the distance from the centre of the planet.}
\end{theorem}

\begin{proof}


\textcolor{blue}{During this derivation, we will avoid considering force as vectors to avoid vector calculus, hence we will consider deriving using magnitude. We first will derive the magnitude gravitational force inside a spherical planet then using it to derive the gravitational field strength inside the spherical planet. In order to derive the equation for the gravitational force, we would first need to enter wacko mode. Let's assume an object $P$ of mass $m$ is inside a spherical planet with mass $M_p$ and radius $R$, let $r$ be the distance between the object and the centre of mass of the planet. Let's also only consider a thin slice of surface shell of mass $dM_s$, and the width of this surface shell will be $R d\theta$ for relatively small $d\theta$}

\begin{figure}[H]
    \centering
    \includegraphics[scale=.35]{wacko 1.1.PNG}
\end{figure}

\textcolor{blue}{Let's say that this slice on the surface shell exerts a small force of $dF$ on the object $P$, to figure what $dF$ is we first consider another small section within the slice with mass $d^2M_s$ (section in red), labelling the force exerted on $P$ by this small section as $d^2F$, we can calculate the force exerted using gravitational force formula 
$$d^2F=\cfrac{Gm}{s^2}d^2M_s$$
if we sum all the $d^2M_s$ along the slice we get the total mass of the slice which is $dM_s$ and the total force $dF$, in other words when we integrate over this slice we get the force exerted by in 
\begin{align*}
    \int d(dF)&=\cfrac{Gm}{s^2}\int \limits_{shell}d(dM_s) \\
    dF&=\cfrac{Gm}{s^2}dM_s
\end{align*}}

\textcolor{blue}{However, since there is partial cancellation due to the vector nature of the force in conjunction with the circular band's symmetry, the leftover component is 
$$dF_r=\cfrac{Gm}{s^2}\cos (\varphi) dM_s $$
where $dF_r$ is the resultant small force exerted by the slice}

\textcolor{blue}{To obtain the total force exerted by the whole outer spherical shell we would need to integrate again 
$$F_r= Gm \int \cfrac{\cos (\varphi)}{s^2}dM_s$$}

\textcolor{blue}{To evaluate this integral, we first need to do some variable changes. We first notice that the area of the outer spherical shell is $4\pi R^2$ and the area of the thin slices (section in blue) is $2\pi R\sin(\theta)Rd\theta=2\pi R^2 \sin (\theta) d\theta$ this implies that the mass of the thin shell $dM_s$ is given by
$$dM_s=\cfrac{2\pi R^2 \sin (\theta) }{4\pi R^2}M_sd\theta=\cfrac{1}{2}M_s\sin(\theta) d\theta$$}

\textcolor{blue}{where $M_s$ is the mass of the outer spherical shell. This gives 
$$F_r= Gm \int \cfrac{\cos (\varphi)}{s^2}dM_s = \cfrac{GmM_s}{2}\int \cfrac{\sin(\theta)\cos(\varphi)}{s^2}d\theta$$}

\textcolor{blue}{Next, we can see that by using law of cosine we get the following equations
\begin{align*}
    \cos (\varphi)&=\cfrac{r^2+s^2-R^2}{2rs} \\ 
    \cos (\theta)&=\cfrac{r^2+R^2-s^2}{2Rr} 
\end{align*}
Differentiating the second equation implicitly we get ($r,R$ are constants)
\begin{align*}
    d(cos(\theta))&=d\left(\cfrac{r^2+R^2-s^2}{2Rr}\right) \\
    -\sin (\theta)d\theta&=\cfrac{-2s}{2Rr}ds \\
    sin(\theta)d\theta&=\cfrac{s}{Rr}ds
\end{align*}}

\textcolor{blue}{Using the results obtain we can now continue with the derivation 
\begin{align*}
      F_r &= \cfrac{GmM_s}{2}\int \cfrac{\sin(\theta)\cos(\varphi)}{s^2}d\theta\\ 
        &= \cfrac{GmM_s}{2}\int \cfrac{\sin(\theta)d\theta \cos (\varphi)}{s^2} \\ 
        &= \cfrac{GmM_s}{2} \cfrac{1}{Rr}\int \cfrac{s\cos (\varphi)}{s^2}ds \\ 
        &=  \cfrac{GmM_s}{2Rr}\int \cfrac{\cos (\varphi)}{s}ds
\end{align*}
Inserting the expression of $\cos (\varphi)$ derived from law of cosine we get 
\begin{align*}
    F_r& =\cfrac{GmM_s}{2Rr}\int \cfrac{\frac{r^2+s^2-R^2}{2rs}}{s}ds \\ 
        &= \cfrac{GmM_s}{4Rr^2} \int \left(1+\cfrac{r^2-R^2}{s^2}\right) ds
\end{align*}}

\textcolor{blue}{To determine the bound of the integration, we may look at the original image, it's clear to see that to find the total force exerted by the outer spherical shell, our value $s$ ranges from its minimum value $R-r$ which is at the left hand side of the planet, to its maximum value, $R+r$ which is on the right hand side of the planet, integrating over this 2 bounds is the same as summing all the force exerted by the thin slices of shell to give the total force exerted by the outer spherical shell 
\begin{align*}
    F_r& = \cfrac{GmM_s}{4Rr^2} \int \limits_{R-r}^{R+r} \left(1+\cfrac{r^2-R^2}{s^2}\right) ds \\ 
        &= \cfrac{GmM_s}{4Rr^2} \int \limits_{R-r}^{R+r} \cfrac{d}{ds}\left[s-\cfrac{r^2-R^2}{s}\right]ds \\ 
        &= \cfrac{GmM_s}{4Rr^2} \left(s-\cfrac{r^2-R^2}{s}\right) \bigg|_{R-r}^{R+r} \\ 
        &= 0
\end{align*}}

\textcolor{blue}{Now, we have proven that the outer shell exerts no force on the object $P$, we can also proof that for the shell that has radius $R-\delta R$ by using the same method as above for all the infinitely thin slices of spherical shell enveloping the inner crust until we reach the inner sphere of radius $r$. }
\begin{figure}[H]
    \centering
    \includegraphics[scale=.35]{wacko 2.PNG}
\end{figure}

\textcolor{blue}{The magnitude of gravitational force experienced by $P$ is then just 
$$F_G=\cfrac{GM_p'm}{r^2}$$
where $M_p'$ is the mass enclosed in the sphere with radius $r$, since one of our assumptions is that the planet has uniform density, this allows us to express the mass enclosed in the sphere, $M_p'$ in terms of $r$ 
$$M_p'=\cfrac{4\pi r^3}{3}\rho$$
where $\rho$ is the density of the planet. Substituting this into $F_G$ yields
\begin{align*}
    F_G=\cfrac{GM_p'm}{r^2}&=\cfrac{4Gm\pi r^3}{r^2}=\cfrac{4Gm\pi\rho r}{3} \\ 
    \implies \vec{F}_G&=-\cfrac{4Gm\pi\rho \vec{r}}{3}
\end{align*}
This means that inside the planet, the gravitational force, $F_G$ is directly proportional to displacement between the object and the centre of mass of planet, $\vec{r}$.}

\textcolor{blue}{To obtain gravitational field strength inside planet we can divide the gravitational force experienced an object by its mass. Therefore the gravitational field strength in a planet's crust is given by 
$$\vec{g}=\cfrac{\vec{F}_G}{m}=-\cfrac{4Gm\pi\rho \vec{r}}{3m}=-\cfrac{4G\pi\rho \vec{r}}{3}$$
which is also directly proportional to the separation between the object and the centre of mass of the planet.}

\textcolor{blue}{The graph for gravitational force $F_G$ and gravitational field strength $g$ will graph of this general shape }
\begin{figure}[H]
    \centering
    \includegraphics[scale=.44]{wacko 4.png}
\end{figure}
\textcolor{blue}{where the green region representing the gravitational field strength or gravitational force in the planet or spherical object and blue region is representing position outside the planet/ spherical object.}
\end{proof}

     \subsection{Gravitational Potential and Gravitational Potential Energy}


An object of mass $M$ will generate a gravitational field with the magnitude strength equals to $\cfrac{GM}{R^2}$ at $R$ distance away from the mass $M$ as discussed from previous section. Assume another object is at infinity which is out of the influence of the gravity produced by mass $M$, then it is said to have \textbf{zero} gravitational potential energy. Now let's say the object is brought into the gravitational field and is $d$ away from mass $M$ from infinity, at that exact position in space, it is said to have gravitational potential of $\phi$, given by 
$$\phi=-\cfrac{GM}{d}$$

The negative sign is due to the relative position from infinity(which is said to have \textbf{zero} gravitational potential energy). Let the mass of the object be $m$, then the gravitational potential energy, $GPE$, of the object at this particular instance is given by
$$GPE=-\cfrac{GMm}{d}$$

As mention in the above section, we now can derive the second definition of gravitational field strength, $\vec{g}$. 

\begin{theorem}
The gravitational field strength, $g$, can be defined as the negative of the instantaneous change of gravitational potential, $\phi$, with displacement, $x$
$$g=-\cfrac{d\phi}{dx}$$
\end{theorem}

\begin{proof}


The gravitational field strength can also be thought of as the instantaneous change of gravitational potential per unit displacement, to derive this relation, we may consider using the principle of conservation of energy, let's assume that a mass $m$ (treat this object as a point particle where all the mass of the object is focused at )is put into a gravitational field, and let's zoom in enough so that the gravitational field strength is almost perfectly uniform, let's denote the strength of this uniform gravitational field as $\vec{g}$, now assume person A pushes the object with constant velocity such that it has moved a distance of $dx$ away from the gravitational field source(similar to lifting object up from the ground), the work done by A is equal to $W$, where 
$$W=\vec{F}_Adx$$
where $\vec{F}_A$ is the force exerted by A, since the object moves in constant velocity, therefore all the work done by A is transferred to the change in gravitational potential energy
\begin{align*}
    md\phi&=\vec{F}_Adx\\ 
    \vec{F}_A&=m\cfrac{d\phi}{dx}
\end{align*}
Given that the object moves in constant velocity, this implies acceleration is zero, hence the force exerted by the source $\vec{F}_S$ has the same magnitude as $\vec{F}_A$ but in opposite direction
\begin{align*}
    \vec{F}_S&=-\vec{F}_A \\ 
            &=-m\cfrac{d\phi}{dx}\\ 
    m\vec{g}&=-m\cfrac{d\phi}{dx} \\ 
    \therefore \vec{g}&=-\cfrac{d\phi}{dx}
\end{align*}        
The second definition of gravitational field strength is hence obtained.
\end{proof}

 \subsection{Satellite's Dynamics and Energetics}


The tangential speed of a satellite in its orbit with mass $m$, and orbiting $R$ meters away from the \textit{centre of the mass of earth} can be calculated using the centripetal force.
    $$\cfrac{GM_em}{R^2}=m\cfrac{{\vec{v}}^{\; 2}}{R} \implies \vec{v}=\sqrt{\cfrac{GM_e}{R}}$$
where $M_e$ is the mass of the earth.\par This implies that the kinetics energy, $KE$, of the satellite is
$$KE=\cfrac{1}{2}m{\vec{v}}^{\; 2}=\cfrac{1}{2}m\left(\sqrt{\cfrac{GM_e}{R}}\right)^2=\cfrac{GM_em}{2R}$$
The $GPE$ can be calculated using the formula in previous section.
$$GPE=-\cfrac{GM_em}{R}$$
Therefore, the total energy of the satellite can be found by adding $GPE$ and $KE$
$$GPE+KE=-\cfrac{GM_em}{R}+\cfrac{GM_em}{2R}=-\cfrac{GM_em}{2R}$$

\textit{Note that the total energy is still negative which implies the satellite is still falling into earth under the influence of earth's gravity.}
\par 
To calculate the \textit{escape velocity} of the satellite with mass $m$, let's assume the position of the satellite is on the earth's surface and stationary (other starting position are possible) and the radius of the earth being $r$, this would make the $GPE$ of the satellite at its initial state be $-\cfrac{GM_em}{r}$. The idea is to give a certain amount of initial $KE$ so that the satellite is just enough to travel out of the influence of earth's gravity. Therefore, the change in $GPE, \Delta GPE$, is the work that needs to be done onto the satellite to escape earth's gravity.(Assuming no air resistance)

$$\Delta GPE= GPE_{\infty}-GPE_0 =0-\left(-\cfrac{GM_em}{r}\right)=\cfrac{GM_em}{r}$$

To calculate the initial velocity needed for this to happen, we may assume that all the $KE$ is converted into increasing the $GPE$, therefore we get the following equation
$$\cfrac{GM_em}{r}=\cfrac{1}{2}m{\vec{v}} ^{\; 2}\implies \vec{v}=\sqrt{\cfrac{2GM_e}{r}}$$
This gives us the escape velocity as desired.

\begin{flushleft}
 %\textbf{Effects Of Centripetal Force On Objects In Rotating Planets.}
\end{flushleft}
 
\begin{figure}[H]
    \centering
    %\includegraphics[scale=0.2]{circular motion.jpg}
\end{figure} 

%When we weight ourselves at the equator and near the pole, we get different values, this is because the normal force acting on us is different and the weighing balance measures this normal reaction. At the equator, we weight less, this can be derived using the following relation
%$$F_c=F_G-N_1$$
%this relation is as shown is because we know that we are rotating with earth, therefore the centripetal force is towards earth's centre, that means gravitational force is greater than reaction force and since net force towards centre of rotation is the centripetal force, therefore we arrive at equation for the apparent weight at the equator
%$$N_1=\cfrac{GM_em}{r^2}-m\cfrac{\vec{v}^{\;2}}{r}$$
%where $M_e$ is the mass of the earth, $r$ is the radius of the earth, $m$ is the mass of the object. 

%When we are not on the equator, let $N_2$ be the normal reaction, then the reaction can be calculated as below
%$$N_2=\cfrac{GM_em}{r^2}-F_c\sin\theta=\cfrac{GM_em}{r^2}-m\cfrac{\vec{v}^{\;2}}{r \cos \theta}\sin\theta=\cfrac{GM_em}{r^2}-m\cfrac{\vec{v}^{\;2}}{r }\tan \theta$$

\newpage
\section{Oscillation}


     \subsection{Simple Harmonic Motion (SHM)}


A SHM can be modelled using a sinusoidal graph
\begin{equation}
 x=x_0\sin (\omega t) 
 \end{equation}
 \begin{equation}
     x=x_0\cos (\omega t)
 \end{equation}
 Where $x$ is the instantaneous displacement, $x_0$ is the maximum displacement and $\omega$ is the angular frequency which is defined by 
 $$\omega =\cfrac{2\pi}{T}$$
 Equation (1) is used to model motion of SHM for which when $t=0$, the starting point is at equilibrium position, whereas equation (2) is used to model oscillation for which when $t=0$, the initial starting position is at maximum displacement, $x_0$
  
 Using the above equations, we can deduce a general equation for the acceleration, $\vec{a}$ of the object in the SHM.
 
 \textbf{\textit{Equation (1)}}
 $$\cfrac{d^2x}{dt^2}=\cfrac{d}{dt}\left(\cfrac{dx}{dt}\right)=\cfrac{d}{dt}\left(\omega x_0 \cos (\omega t)\right)=-\omega^2 x_0 \sin (\omega t)$$
 $$\therefore \vec{a}=-\omega^2 x$$

 \textbf{\textit{Equation (2)}}
 $$\cfrac{d^2x}{dt^2}=\cfrac{d}{dt}\left(\cfrac{dx}{dt}\right)=\cfrac{d}{dt}\left(-\omega x_0 \sin (\omega t)\right)=-\omega^2 x_0 \cos (\omega t)$$
 $$\therefore \vec{a}=-\omega^2 x $$
 
 This gives the two conditions for which the acceleration of the SHM must satisfy, which are that the acceleration is directly proportional to displacement and is always directed towards the equilibrium position since it has different sign as the instantaneous displacement.
 
 We can also derive an equation for instantaneous velocity, $\vec{v}$, but the $\vec{v}$ depends on the modeling equation and the starting position.
 
 \textbf{\textit{Equation (1)}}
 $$\cfrac{dx}{dt}=\vec{v}=\omega x_0 \cos (\omega t)$$
 From equation (1) we may let $\omega t$ be an angle in a right angled triangle
 \begin{figure}[H]
     \centering
     \includegraphics[scale=.3]{shm.jpg}
 \end{figure}
This implies that $\cos (\omega t)=\cfrac{\sqrt{{x_0}^2-x^2}}{x_0}$, which give us the final equation of 
$$\vec{v}=\omega x_0 \cos (\omega t)=\omega x_0\cfrac{\sqrt{{x_0}^2-x^2}}{x_0}=\omega \sqrt{{x_0}^2-x^2}  $$
 
\textbf{\textit{ Equation (2)}}
 $$\cfrac{dx}{dt}=\vec{v}=-\omega x_0 \sin (\omega t)$$
 With similar idea, we can construct a similar right angled triangle from equation (2)
 \begin{figure}[H]
     \centering
     \includegraphics[scale=.3]{shm1.jpg}
 \end{figure}
 This results in $\sin (\omega t)=\cfrac{\sqrt{{x_0}^2-x^2}}{x_0}$, consequently $\vec{v}$ is given by 
 $$\vec{v}=-\omega x_0 \sin (\omega t)=-\omega x_0\cfrac{\sqrt{{x_0}^2-x^2}}{x_0}=-\omega \sqrt{{x_0}^2-x^2}$$


 \subsection{Energetics of SHM}


The total energy of the oscillating object in SHM is always constant, therefore at any point of the displacement, the sum of $KE$ and $PE$ is always constant.Therefore knowing the maximum value of $KE$ or $PE$, we can know the total energy of the system.

\begin{figure}[H]
    \centering
    \includegraphics[scale=.2]{shm2.jpg}
\end{figure}
\newpage
\begin{flushleft}
     \textbf{Damped Oscillation}
\end{flushleft}

\begin{enumerate}
    \item \textbf{Light Damping}- Amplitude \textit{slowly} reduces to zero after a large number of oscillations.
     \item \textbf{Critical Damping}- Amplitude decreases to zero in the shortest possible time after a few oscillation.
     \item \textbf{Heavy Damping}- There may be \textit{no oscillation}, but the amplitude decreases to zero after a long period of time. 
\end{enumerate}

\begin{flushleft}
     \textbf{Forced Oscillation}
\end{flushleft}

When an oscillating object is connected to an oscillator, the energy of the oscillator may be transferred into the oscillating object thus increasing its amplitude, when the frequency of the oscillator equals the natural frequency of the object, the energy transferred occurs in the most efficient way possible, as a result increasing the amplitude exponentially. This phenomenon is known as \textbf{\textit{resonance}}.

\newpage
\section{Ideal Gases}

\begin{tcolorbox}[colframe=black!20!white,title=
 \color{red}{Assumptions}]

\begin{enumerate}
\color{red}
    \item All collisions are elastic.
    \item No inter-molecular forces.
    \item Size of gas molecules/particles is negligible.
     \item All gasses obey ideal gas equation.
     $$\color{red}{PV=nRT}$$
\color{red}{where $n$ is the number of mole, $R$ is the ideal gas constant ($\approx 8.314 \; J mol^{-1}K^{-1}$).}
     \item The duration of collision is negligible compared to the time between collisions.
\end{enumerate}
\end{tcolorbox}


 \subsection{Kinetic Temperature}

\begin{theorem}
The kinetic energy, $KE$, of a molecule/particle in a gas of absolute temperature, $T$ is defined as 
$$KE=\cfrac{3}{2}kT$$
\end{theorem}


\begin{proof}

To find out the relation between kinetic energy, $KE$,  and temperature, $T$ of gas, we may first simplify the problem by only considering the $KE$ of one molecule of the gas with mass $m$ and velocity $\vec{v}$ in a closed container. Furthermore, we can also split $\vec{v}$ into three components, namely $\vec{v}_x, \vec{v}_y, \vec{v}_z$, which satisfy the following relation.
\begin{equation}
{\vec{v}}^{\; 2}={\vec{v}_x}^{\; 2}+{\vec{v}_y}^{\; 2}+{\vec{v}_z}^{\; 2}
\end{equation}
\begin{figure}[H]
    \centering
    \includegraphics[scale=.2]{ideal gas 1 (1).jpg}
\end{figure}
Let's assume that side $W$ is perpendicular to $\vec{v}_x$. When the particle collides with side $W$, since all collisions are elastic, this implies that the velocity of the particle in $x$ direction, $\vec{v}_x$, changes into $-\vec{v}_x$, this yields
$$\Delta \vec{p}=2m\vec{v}_x$$
    where $\Delta \vec{p}$ is the magnitude of the change in momentum after colliding with side $W$.\textit{[Note that only $\vec{v}_x$ is changed after collision.]}

To calculate the average force exerted by one particle on $W$, $\langle\vec{F}\rangle$, the following relation can be used
    $$\langle \vec{F} \rangle= \cfrac{\Delta \vec{p}}{\Delta t}$$
where $\Delta t$ is not the duration of collision because using the duration of collision will give us the instantaneous force exerted by the particle on the wall during the collision, instead $\Delta t$ is the time between the first collision on $W$ and the second collision $W$, so that when divided by this time interval give us the average force $\langle \vec{F} \rangle$.

\begin{figure}[H]
    \centering
    \includegraphics[scale=.2]{ideal gas 1.jpg}
\end{figure}
Notice that, the time taken to travel from $W$and $W'$ is only affected by the velocity $v_x$, this yields the following equation
$$\Delta t=\cfrac{2L}{\vec{v}_x}$$

Using both data, we get the following equation for the average force of one particle
$$\langle \vec{F} \rangle=\cfrac{\Delta \vec{p}}{\Delta t}=\cfrac{2m\vec{v}_x^{\; 2}}{2L}=\cfrac{m\vec{v}_x^{\; 2}}{L}$$

Let's assume that there are $N$ number of the same molecule in the container, this follows that the total average force exerted on $W$ be
\begin{align*}
\langle \vec{F}_{total} \rangle&=\cfrac{m{\vec{v}_{x,1}}^{\; 2}}{L}+\cfrac{m{\vec{v}_{x,2}}^{\; 2}}{L}+\cfrac{m{\vec{v}_{x,3}}^{\; 2}}{L}+\hdots+\cfrac{m{\vec{v}_{x,N}}^{\; 2}}{L}\\ &=\cfrac{m}{L}({\vec{v}_{x,1}}^{\; 2}+{\vec{v}_{x,2}}^{\; 2}+{\vec{v}_{x,3}}^{\; 2}+\hdots +{\vec{v}_{x,N}}^{\; 2})\\ &=\cfrac{Nm}{L}\left(\cfrac{\sum\limits_{i=1}^N {\vec{v}_{x,i}}^{\;2}}{N}\right)\\ &= \cfrac{Nm \langle {\vec{v}_x}^{\; 2} \rangle}{L}
\end{align*}

This follows that the pressure, $P$ on $W$ will be
\begin{equation}
P=\cfrac{\langle \vec{F}_{total} \rangle}{A}=\cfrac{Nm \langle {\vec{v}_x}^{\;2} \rangle}{L \cdot L^{\; 2}}=\cfrac{Nm \langle {\vec{v}_x}^{\;2} \rangle}{L^3}=\cfrac{Nm
\langle {\vec{v}_x}^{\;2} \rangle}{V}
\end{equation}
Going back to the ideal gas assumption, since all the particle are moving in random directions, that implies that there's no preference in the direction of velocity, hence the mean square velocity in all directions should be the same
$$\langle {\vec{v}_x}^{\;2} \rangle=\langle {\vec{v}_y}^{\;2} \rangle=\langle {\vec{v}_z}^{\;2} \rangle$$
\textit{Note that the mean velocity in all directions is \textbf{zero} because the motion are completely random.}
$$\langle \vec{v}_x \rangle =\langle \vec{v}_y \rangle=\langle \vec{v}_z \rangle=0$$
By using equation (3), if we sum up all the velocity squared of each particle in the gas and divide by the number of molecule, we obtain average velocity squared of the gas.
$$\langle \vec{v}^{\;2} \rangle=\cfrac{\sum\limits_{i=1}^N {\vec{v}_i}^{\;2}}{N}=\cfrac{\sum\limits_{i=1}^N({\vec{v}_{x,i}^{\;2}}+{\vec{v}_{y,i}^{\;2}}+{{\vec{v}_{z,i}}^{\;2}})}{N}=\langle \vec{v}_x^{\;2} \rangle+\langle \vec{v}_y^{\;2} \rangle+\langle \vec{v}_z^{\;2} \rangle=3\langle \vec{v}_x^{\;2} \rangle \implies \cfrac{\langle {\vec{v}^{\;2}}\rangle}{3}=\langle \vec{v}_x^{\;2} \rangle$$
Substituting the above results into equation (4) and rearranging it will results in 
$$PV=\cfrac{Nm\langle {\vec{v}}^{\;2} \rangle}{3}$$
Equating this to the ideal gas equation
\begin{align}
    \cfrac{Nm\langle {\vec{v}}^{\;2} \rangle}{3}&=nRT  \nonumber \\ 
    \cfrac{Nm\langle {\vec{v}}^{\;2} \rangle}{3}&= N\cfrac{R}{N_A}T \nonumber\\
    \implies \sum KE&=\cfrac{1}{2}Nm\langle {\vec{v}}^{\;2}\rangle=\cfrac{3}{2}NkT\\
    \implies KE&= \cfrac{1}{2}m\langle {\vec{v}}^{\;2} \rangle=\cfrac{3}{2}kT 
\end{align}
where $N_A$ is the Avogadro's number $(\approx 6.02\times 10^{23})$, $N$, is the number of atoms/molecules in the gas, $k$ is the Boltzmann constant $(=\cfrac{R}{N_A}\approx 1.38\times 10^{-23} m^{\;2} kg s^{-2}K^{-1})$, and $T$ is the absolute temperature. \textbf{\textit{Note that (5) is the average kinetic energy of the gas, once $N$ is eliminated we get (6), which is the average kinetic energy of one molecule in the gas.}}
\end{proof}

We can also get the root mean square velocity, $\vec{v}_{rms}$ of a molecule in the gas using (6)

$$\sqrt{\langle \vec{v}^{\; 2} \rangle }=\vec{v}_{rms}=\sqrt{\cfrac{3kT}{m}} $$

\newpage
\section{Temperature}
Temperature indicates the degree of hotness of a body. Two bodies are in thermal equilibrium if the nett rate of flow of thermal energy between them is zero and has same temperature.
\\The \textbf{Zeroth Law of Thermodynamics} states that if body A is in thermal equilibrium with body C, and body B is also in thermal equilibrium with body C, then body A and body B are in thermal equilibrium. This is the basis on which ideal thermometer and temperature scale is based on.
\\The 2 types of temperature scales are 
\begin{enumerate}
    \item Empirical scale, $^\circ C$, Symbol: $\theta$
    \item Thermodynamic scale, Kelvin (K) Symbol:T
\end{enumerate}


\subsection{Thermometric Properties}

Thermometric properties are physical quantities that has magnitudes dependant on temperature. For example:
\begin{enumerate}
    \item Volume of liquids/gases
    \item Resistance of a wire
    \item Pressure of a gas
    \item Electromotive force (emf) of a thermocouple
    \item Intensity of infrared
\end{enumerate}

\begin{flushleft}
Setting up empirical scale
\end{flushleft}
\begin{enumerate}
    \item Choose a thermometric property that changes linearly with temperature
    \item Measure the magnitude of X at 2 reference points 
    \item Measure magnitude of X at unknown temperature $\theta ^\circ C$
    \item $$\theta = \cfrac{X_\theta-X_0}{X_{100}-X_0}\times 100^\circ C$$
\end{enumerate}
\begin{flushleft}
Types of thermometers(empirical)
\end{flushleft}
\begin{enumerate}
    \item Liquid in glass (Volume of liquid/length of liquid thread)
    \\Advantages
    \begin{enumerate}
        \item Instant reading
        \item Portable
    \end{enumerate}
    Disadvantages
    \begin{enumerate}
        \item Poor accuracy
        \item Average sensitivity
        \item Limited range
    \end{enumerate}
    \item Resistance thermometer
    \\Advantages 
    \begin{enumerate}
        \item Accurate
        \item Wide range (200-1200 $^\circ C$)
    \end{enumerate}
    Disadvantages
    \begin{enumerate}
        \item Low sensitivity-Not suitable for detecting temp which changes rapidly
    \end{enumerate}
    \item Thermo-couple (emf)
    \\Advantages
    \begin{enumerate}
        \item Wide range (250-1500$^\circ C$)
        \item Fairly sensitive/short response time- suitable for rapidly changing temp
        \item small junction-access isolated location easily
    \end{enumerate}
\end{enumerate}
\\
\textbf{2 thermometers might not give the same reading because temperature scale assumes linear change of thermometric properties with temperature, but physical properties might not vary linearly with temperature, and agrees only at fixed points.}
\\\\
Thermodynamic (absolute) scale has a reference point, called the \textbf{triple point of water}.-- 273.16K/0.01 $^\circ C$. It is independent of thermometric properties and is only used in \textbf{constant-volume gas thermometer}.\textit{(measures pressure of a fixed mass of gas at a constant volume and assumes that pressure of gas changes linearly with the change in temperature(K).)}
\begin{flushleft}
Setting up thermodynamic/absolute scale
\end{flushleft}
\begin{enumerate}
    \item Measure the pressure of a fixed mass of gas at a constant volume at triple point of water  ($0.01^\circ C/273.16K),P_{tr}mmHg$
    \item Measure pressure at unknown temperature, $\theta, P_T$
    \item $$T=\cfrac{P_T}{P_{tr}}\times273.16K$$
\end{enumerate}

\begin{flushleft}
Definition of units of temperature
\begin{enumerate}
    \item $1^\circ C$(centigrade)= $\cfrac{1}{100}\times$(temp of steam-temp of ice)
    \item 1K= $\cfrac{1}{273.16}\times$temperature of the triple point of water in K)
\end{enumerate}
\end{flushleft}


\newpage
\section{Thermal Properties}
\begin{flushleft}
 \textbf{State Changes}
\end{flushleft}

\begin{figure}[H]
    \centering
    \includegraphics[scale=.2]{state change.jpg}
\end{figure}

\textbf{\textit{I}}

The solid absorbs thermal energy to increase its mean kinetic energy, this causes the temperature of the solid to increase. 

\textbf{\textit{II}}

The solid begins to experience phase changes, the heat energy absorb is no longer used to increase its mean kinetic energy of the particles, but to overcome inter-molecular forces, since kinetic energy is not increased, the temperature does not change, at this period of time, it's in solid and liquid state.

\textbf{\textit{III}}

The object has turned into liquid, the heat energy absorbed is no longer used to break inter-molecular bonds, but to increase to mean kinetic energy of the particle, hence the temperature increases after the state change.

\textbf{\textit{IV}}

Similar to \textbf{\textit{II}}, the temperature absorb is to overcome the inter-molecular forces and to expand the volume by doing work against atmospheric pressure. A liquid-vapour phase is reached, and is preparing to turn completely into vapour.

\textbf{\textit{V}}

Gas state is obtained. Thermal energy absorbed afterwards are used to increase the mean kinetic energy of the particles, therefore the temperature increase afterwards.

\begin{flushleft}
 \textbf{Specific Heat Capacity}
\end{flushleft}

When equal amount of heat energy, $Q$, is supplied to different material of the same mass, $m$, the increase in temperature, $\Delta \theta$, is different, this is because different material have different specific heat capacity, $c$. These quantities are related in the following equation.
$$Q=mc\Delta \theta$$
This is assuming that all the heat energy supplied are use to increase the mean kinetic energy of the particles, hence it can't be used to describe the energetics during state changes.

\begin{flushleft}
 \textbf{Specific Latent Heat}
\end{flushleft}

During phase changes, heat is absorbed to overcome inter-molecular force and other external factor such as atmospheric pressure depending on the phase that is about to changed into, during this period, the temperature is not change, and depending on the material, more heat is required to complete the phase change, and the heat require for specific type of material are called specific latent heat. There are two type of specific latent heat. 

For changing from solid to liquid state or vice versa, we use latent heat of fusion, $L_f$, and is related by this equation
$$Q=mL_f$$
where $Q$ is the heat absorbed/released needed to change the state of a certain material of mass $m$ with specific latent heat of fusion of $L_f$ from solid to liquid or vice versa.

For liquid to gas or vice versa, we use 
$$Q=mL_v$$
where $L_v$ is the latent heat of vaporisation.

\textit{In most cases. $L_v > L_f$ because more work is needed to overcome atmospheric pressure and to increase the separation of the atoms to a greater extend.}

\begin{flushleft}
 \textbf{Internal Energy}
\end{flushleft}

The total internal energy, $u$, of a gas can be represented as the sum of all the kinetic energy and all the potential energy of all the particles
$$u=KE+PE$$
since we assume all gases are ideal, this implies that $PE=0$, hence we obtain the following equation
$$u=KE=\cfrac{3}{2}NkT$$

When trying to find the change of energy of gas, we can simplify the change in energy into two components, which the heat energy supplied to the gas and work done on the gas or work done by the gas.
$$\Delta u=q+W$$
where $q$ is the heat energy supplied/loss and $W$ is the work done to/by the gas. \textit{Note that when heat energy is supplied, $q>0$, when heat energy is loss, $q<0$. When work is done by the gas, $W<0$, when work is done onto the gas $W>0$. \textbf{The only work that be done on the gas is decreasing its volume, the only work that be done by the gas is expanding. $W=P\Delta V$}}

Since $u=\cfrac{3}{2}NkT$, therefore if temperature is constant (i.e isothermal), then $\Delta u=0$, this implies 
\begin{align*}
    \Delta u&=q+W \\ 
    0&=q+W\\ 
    -q&=W
\end{align*}
the above result tells us that if we supply heat, i.e $q>0$, then $W<0$, which means work must be done by the gas. This makes sense because remember temperature is constant in this case, so every joule of heat supplied is used by the gas to done work, or every work done on the gas is loss as heat. Notice in either case work must be either done on the gas or by the gas, which means the volume must be changing. Knowing that temperature is constant, volume must be changing, then that must mean that the pressure must also be changing and is related by $p_1v_1=p_2v_2$, known as Boyle's law.

Now. let's consider the case of $q=0$, which means no heat is supplied to or loss by the gas, then we have 
\begin{align*}
    \Delta u&=q+W\\ 
    \Delta u&=0+W\\
    \Delta u&=W
\end{align*}
this means that change in internal energy is the work done on or by the gas. If work is done on the gas, $W>0$, then that means $\Delta u>0$, there's an increase in internal energy, in turn means that the temperature of the gas increases because $u\propto T\implies \Delta u\propto\Delta T$(\textbf{even though there is no heat supplied}). So in other words, if volume decreases (work done on the gas), the temperature must increase, and pressure can be found by $\cfrac{P_1V_1}{T_1}=\cfrac{P_2V_2}{T_2}$

For final case, we consider if the volume of the gas is constant, $V=$constant, then that means $W=0$ because if volume is constant that must mean no work is done on or by the gas. This implies that
\begin{align*}
    \Delta u&=q+W\\ 
    \Delta u&=q+0 \\ 
    \Delta u&=q
\end{align*}
which means that the heat supplied or loss directly contribute to the change in internal energy, no energy is contributed to change the volume as no work is done on or by the gas. In thermodynamic, we only consider \textbf{work done as change in volume}, even though in this case the heat supplied is directly used to increase the kinetic energy of the particles in the gas, but in thermodynamic, this is not consider as work done on the gas for reasons we will get into later. For the time being, if there is no change in volume then there is no work done on or by the gas.

\textcolor{blue}{Now we will talk about work done more formally. Work is previously defined as $W=P\Delta V$. In thermodynamics however, the proper definition for work done by a gas is
$$W=\int P\;dV$$
}
\textcolor{blue}{It is not always the case that $P$ is constant, it could be in terms of volume, temperature, time and etc.}
\begin{figure}[H]
    \centering
    \includegraphics[scale=.2]{work done 1.jpg}
\end{figure}
\textcolor{blue}{Looking at the diagram and the definitions of work done in thermodynamics, we can deduce that the work done is equal to the area under the $P-V$ graph, but how do we determine whether this work done is on the gas or by the gas? Remember that we consider work done as change in volume, so in this case, going from $A$ to $B$, there is an increase in volume, and going from $B$ to $C$ there is also an increase in volume. So the value of the area under the graph from $A\to C$ is the value of work done \textbf{by the gas} (because there is an increase in volume.)}

\textcolor{blue}{Now let's consider the case below}
\begin{figure}[H]
    \centering
    \includegraphics[scale=.2]{work done 2.jpg}
\end{figure}
\textcolor{blue}{As can be seen, the gasses goes from $C\to B \to A$, the area under graph still represents the work done, but instead of work done by gas, it is the work done on the gas becasue there is a decrease in volume.}

\newpage
\section{Communication}

\subsection{Communication channels}
 
 A few communication channels that are included in our syllabus are:
 \begin{enumerate}
     \item Wire-pairs
     \item Coaxial cables
     \item Radio waves
     \item Microwaves
     \item Optic fibres
     \item Communication satellites
 \end{enumerate}
 
 
 \subsection{Wire-pairs}

 Wire-pairs provides a very simple link between a transmitter and receiver. In modern communication, it is only used in short distance communication of low frequencies such as linking a door bell to the switch. There are a few disadvantages of wire-pairs:
 \begin{enumerate}
     \item high attenuation of signal--heating due to resistance in wires and electromagnetic loss
     \item pick up noise easily (functions as aerial transmitter/antenna)-- unwanted signals (current) generated due to electromagnetic radiation by other wires
     \item suffer from cross-talk/cross-linking--low security
     \\wire-pairs can pick up each other's signals. A way to reduce this is to twist wires together, causing its electromagnetic effects to cancel each other out, thus reducing cross-talks.
     \item limited bandwidth--bandwidth is only about 500kHz.
 \end{enumerate}
 
 
 \subsection{Coaxial cables}

 \begin{figure}[H]
     \centering
     \captionsetup{justification=centering,margin=2cm}
     \includegraphics[scale=.5]{communication.jpg}
     \caption*{(Source: Tanenbaum, Andrew S., et al.\textit{Computer Networks, 5th Edition}. Pearson, 2011.)}
 \end{figure}
 Coaxial cables consist of two wire conductors. The 1st (central) conductor transmit electrical signals and is covered by an insulator. The 2nd is earthed and acts as the 'return' for the signal. It is in the form of a thin wire braid that completely surrounds the insulator. Since it is earthed, this braiding shields the inner conductor from external interference, and prevent EM waves from inner copper wire to pass through, thus it is more secure. The braid is covered by a protective layer of insulation. Coaxial cables are used in connection from aerial to television receivers.
\newpage
Advantages of coaxial cables:
 \begin{enumerate}
     \item less attenuation
     \item picks up less noise
     \item more secure / less wire-tapping
     \item has larger bandwidth (about 50MHz)
 \end{enumerate}
 

 \subsection{Radio Waves \& Microwaves}

Alternating current in a wire acts as aerial as it produces electromagnetic waves. In communication, radio waves (30kHz-3GHz) and microwaves (3GHz-30GHz) are used. 
\\Generally, radio waves are classified into:
\begin{enumerate}
    \item Surface waves 
    \begin{enumerate}
        \item low frequency (up to 3MHz)
        \item long range
        \item can be reflected and refracted along the Earth's surface--blocked by mountains or tall buildings
    \end{enumerate}
    \item sky waves
    \begin{enumerate}
        \item frequency 3-30MHz
        \item reflected by the ionosphere\\
        However, the density of ionosphere is not constant thus reflection is inconsistent.
    \end{enumerate}
    \item space waves
    \begin{enumerate}
        \item frequency $>$30MHz
        \item can pass through the ionosphere (for satellite)
        \item used in TV, mobile phone, FM radio
    \end{enumerate}
\end{enumerate}
Microwaves (1-300GHz) can pass through the ionosphere to be reflected by satellites in space.
\\Advantages:
\begin{enumerate}
    \item large bandwidth
    \item transmitted in a straight line
    \item can be transmitted in a narrow beam using parabolic dish\\
    \textit{a parabolic dish is used to focus as much wave power as possible to the aerial.}
    \item used in wifi technology
\end{enumerate}
\begin{flushleft}
Advantages of radio and microwave links:
\end{flushleft}

\begin{enumerate}
    \item EM waves with wide range of frequencies
    \item mode of use and range depends on frequency
    \item can encode information for security
\end{enumerate}
\newpage


\subsection{Communication satellite}

\begin{enumerate}
    \item A transmitter station sends up a signal called \textbf{uplink} with frequency $f_{up}$. This uplink is attenuated thus has to be amplified by the satellite. 
    \item A strong(amplified) signal with a smaller frequency, $f_{down}$, called \textbf{downlink} is sent downwards to a receiver.\textbf{This is so that inteference do not occur between uplink and downlink to prevent the downlink from swamping (can be distinguished from) the uplink.}
\end{enumerate}
\textbf{Values of uplink and downlink are fixed, such as 6GHz and 4GHz (6/4GHz band) or 14/11GHz or 30/20GHz}

\begin{flushleft}
\item \textbf{Types of commercial artificial satellites}
\end{flushleft}
\begin{enumerate}
    \item Geo-stationary satellite 
    \begin{enumerate}
        \item Appears to be stationary above a fixed point of the Earth.
        \item Period of 24 hours
        \item Equatorial orbit
        \item Rotates from West to East with same direction of rotation of Earth
        \item does not need to be tracked
    \end{enumerate}
    \item Polar satellites
    \begin{enumerate}
        \item Orbits about the North and South pole
        \item Shorter orbit period
        \item Needs to be tracked
        \item Provides better coverage of Earth - used in weather monitoring but requires more satellites for coverage
        \item closer to Earth (radius of orbit is smaller)
    \end{enumerate}
\end{enumerate}



\subsection{Optic fibres}

An optic fibre consists of a pure glass core surrounded by an outer cladding. \textbf{Pulses of light or I.R.} travel along the fibre through \textbf{total internal reflection}. An optic fibre cable consists of hundreds of fibres.
\\Advantages:
\begin{enumerate}
    \item low attenuation, since glass absorb or scatter very little light/I.R -- so less amplifiers
    \item large bandwidth (radiation pulses have high frequencies in the order of $10^8$MHz)
    \item lower cost
    \item very light and easier to handle
    \item do not pick up EM interference, so high security and negligible cross-talk
    \item can be laid alongside existing routes eg. electric railway lines and power lines
    \item can be used in flammable situations since no sparks are produced
\end{enumerate}
Disadvantages:
\begin{enumerate}
    \item Electrical signals must be converted into pulses of light or I.R
    \item difficult to connect end to end of optic fibres (unlike metal wires)
\end{enumerate}
\newpage

\subsection{Modulation}

When a person is listening to the radio, the person is at the end of the communication system, the sound or music that is played at the radio is first passed into a microphone this changes sound wave into ac current, and this ac current is will produced EM wave when is the ac current is used to oscillate electrons, this EM wave is then imprint onto a carrier wave with certain frequency (usually very high in magnitude so that it is able to carry more information with high precision), and the information of the original EM wave is embedded into the resultant wave by variation of amplitude (amplitude modulation/AM) or variation of frequency (frequency modulation/FM).

\begin{flushleft}
\item Advantages of using modulated waves
\end{flushleft}
\begin{enumerate}
    \item Information is transmitted at the speed of light.
    \item Frequency of carrier wave can be varied. (this allows us to distinguish different modulated wave just by analysing the frequency of the carrier wave, moreover different stations can tune to receive a certain frequency of wave so that only the desired wave is received, and different frequency of modulated can avoid interference of signals.)
\end{enumerate}

\begin{flushleft}
\item \textbf{Amplitude Modulation (AM)}
\end{flushleft}
An amplitude modulated wave is a carrier wave that embeds the information of the signal wave to itself by changing its amplitude according to the amplitude of the signal wave. The change in amplitude can be done by adding the amplitude of the particular time to the amplitude of the message signal at that specific time like the example below
\begin{figure}[H]
    \centering
    \captionsetup{justification=centering,margin=2cm}
    \includegraphics[scale=.3141]{communication 1.png}
    \caption*{(Soure: Shaik, Asif. \textit{Amplitude modulation.} )}
\end{figure}

\textit{In the bottom-most graph, the frequency of the red wave is the \textbf{frequency of transmission}, frequency of the blue envelope is the \textbf{frequency of modulating waveform}.}

Notice that we can calculate the frequency of the carrier wave and the message signal just from the resultant modulated wave alone, this is because the period between the peaks $A_c+A_m$ is the period of the carrier wave and the frequency of wave in the envelope is the frequency of the carrier wave due to the fact that the modulated wave is just the carrier wave with variation in amplitude that changes according to the amplitude of message signal wave at that specific time, hence the frequency is not changed.

When this signal is sent or broadcast and received by a receiver, while analysing the spectrum of the AM signals received, we get not just one peak, instead we get 3 peaks with 3 different frequencies. (A spectrum represents the relative amounts of different frequency components in any signal)
\begin{figure}[H]
    \centering
    \includegraphics[scale=.2]{communication 2.jpg}
\end{figure}
where $f_c$ is the frequency of the carries wave and $f_m$ is the frequency of the message signal. This phenomena occurs due to what is known as the beat effect where two frequencies mix to produce the sum and difference frequencies. An example is when you tune a guitar string against another by playing the same note simultaneously. If they are not in tune, you can hear the "beat frequency" which is actually the difference between the two. AM works the opposite way: the "beat frequency" is the information which produces the side-bands. In this case, the frequency of the message signal is constant hence $f_c \pm f_m$ is also a constant,which makes the only useful signals that are transmitted are with frequencies $f_c+f_m$ and $f_c-f_m$, $f_c$ does not contain any information because it's just the carrier wave and the carrier wave signal is filtered out once it has reached the receiver. 

The range of frequencies that the signal occupies is known as the bandwidth, which can be calculated by finding the difference of the maximum and minimum frequency of the signal, and in this case it is $2f_m$, although the only useful signals are only at 2 specific frequency, but the receiver must have the ability to receive signals of this bandwidth to fully obtain the information sent.

In a more realistic case, the frequency of the signal wave that we want to transmit or embed into the carrier wave won't have a constant frequency such as music and talks.  
\begin{figure}[H]
    \centering
    \captionsetup{justification=centering,margin=2cm}
    \includegraphics[scale=.3]{communication 3.jpg}
    \caption*{ Amplitude Modulation.\\  (Source: \textit{WatElectronic.com.})}
\end{figure}
let's assume that the frequency of the carrier wave is $f_c$, and the frequency of the message signals, $f_m$, is in the range of $min\{f_m\}\leq f_m \leq max \{f_m\}$
and the AM spectrum of the signal will be as shown
\begin{figure}[H]
    \centering
    \captionsetup{justification=centering,margin=2cm}
    \includegraphics[scale=.5]{communication 4.png}
    \caption*{Amplitude Modulation Spectrum. \\  (Source: \textit{electronicsnotes.})}
\end{figure}
the frequency at the right-end of the upper sideband will be $f_c+max\{f_m\}$
and the frequency at the left-end of the lower sideband will be $f_c-max\{f_m\}$ instead of $f_c-min\{f_m\}$, this is because when the beat effect occur beating the carrier frequency and the signal frequency, $f_c \pm max \{f_m\}$ will cover all the range for the possible beat effect that will happen for all the frequency in the possible range of value of $f_m$ can be. Therefore the bandwidth in this case is $2\times max\{f_m\}$  
\begin{flushleft}
\item Advantages of AM
\end{flushleft}
\begin{enumerate}
    \item Able to propagate over a long distance. (because of its long wavelength, it can be reflected by ionosphere and reach long distances. It can also bend with the curvature of the earth because of its long wavelength.)
    \item AM transmitter and receiver are cheap.
    \item Small bandwidth for each channel which allows more channels to be share the waveband. (note that bandwidth of each channel does not overlap with one another)
\end{enumerate}
\begin{flushleft}
\item Disadvantages of AM
\end{flushleft}
\begin{enumerate}
    \item Noise get picked up easily which will affect the quality of the sound produced. (this is because the information is stored as amplitude difference, when disturbance is introduced, amplitude will varies which messes up the signal)
    \item Short bandwidth. 
\end{enumerate}
\begin{flushleft}
\item \textbf{Frequency Modulation (FM)}
\end{flushleft}
A frequency modulated wave is a resultant wave produced when a carries wave changes its frequency according to the amplitude of the message signal at any given time.
\begin{figure}[H]
    \centering
    \captionsetup{justification=centering,margin=2cm}
    \includegraphics[scale=.3141]{communication 5.png}
    \caption*{Frequency modulation. \\ (Source: \textit{physics and radio-electronics.})}
\end{figure}
The modulated wave has amplitude equals to the amplitude of the carrier wave with reason same as why AM modulated wave has frequency equals to the frequency of its carrier wave. 

For FM signals, the bandwidth is usually 5 times wider than AM and to calculate the maximum/minimum frequency of the signal in the spectrum just by only using frequencies of carrier and message signal is beyond A-level syllabus.

\begin{flushleft}
\item Advantages of FM
\end{flushleft}
\begin{enumerate}
    \item Less noise picked up in the transmission. (because only the frequency of the modulated wave contains information, hence spikes in amplitude will not have an effect in the final information received.)
    \item Long bandwidth (sounds with higher fidelity can be produced)
\end{enumerate}
\begin{flushleft}
\item Disadvantages of FM
\end{flushleft}
\begin{enumerate}
    \item FM transmitter and receiver are much more expensive.
    \item Large bandwidth causes less channels to share the same waveband.
\end{enumerate}


\subsection{Analogue and Digital Signal }


An analogue signal contains analogue quantity to contain or deliver information, an analogue quantity is one that can have any value. An example of analogue signal is the voltage generated by a sound registered by the microphone. On the other hand, a digital signal uses binary digits ($0$ and $1$s also known as bits) to transmit information. 

In transmission, all signals experience noise(distortion) and attenuation. In analogue signals, when the signal is amplified, the noise gets amplified too which is not what we wanted. But in digital signals, the noise picked up during transmission can be filtered out and the signal can be reproduced with a higher amplitude by using a regenerator amplifier.
\newpage
\begin{flushleft}
\item Advantage of digital signals over analogue signals
\end{flushleft}
\begin{enumerate}
    \item Can be transmitted over long distance because can be amplified without concerning noise amplification. 
    \item Reliable and cheap.
    \item Better security.
\end{enumerate}


\begin{flushleft}
\item Analogue-to-digital Conversion
\end{flushleft}

When given a graph of analogue voltage reading against time (continuous graph), we first need to find the shortest time interval between a crest and a trough that are directly next to each other which the analogue voltage wave contains and let this time interval will be our sampling period (to ensure that these 2 local maximum and minimum or transition between these turning point is sampled). We then start sampling at $t=0$ with sampling period calculated, the analogue reading being sampled is first rounded down (floor function is taken for the sampling reading) then converted into binary digits (base 2).\textit{ (floor function of $x$ , $\lfloor x \rfloor$, is defined as the smallest integer less than $x$, for instance $\lfloor 1.99 \rfloor =1 \; ,\; \lfloor -\pi\rfloor = -4$)} The number of digits when the sampled analogue reading is converted into binary will depend on the number of digits of the binary representation of the analogue reading that has the greatest magnitude. A \textbf{parallel-to-serial converter}, placed in the circuit after an ADC, receives all bits at one time and transmit the bits one after another. A \textbf{serial-to-parallel converter}, placed before a DAC, receives bits one after another and transmits digits all at once. An example is given below
\begin{center}
    \begin{tabular}{|c|c|}
    \hline
    Analogue Voltage Sampled \; / V      & Binary representation  \\
    \hline
    1     & 001 \\
    \hline
       6  & 110\\
    \hline
      3   & 011\\
      \hline
    \end{tabular}
\end{center}
Notice we write $3$ as $011_2$ instead of $11_2$ because the largest analogue voltage sampled is $6V$ and its binary digits contains $3$digits. (we do not need to include the subscript 2 in the binary value that is converted from analogue). \textbf{Note that when converting from analogue to digital, we can never get a perfect conversion because the readings are only sampled at certain time points, to convert with higher precision and details we would need to lower the sample period or increase the sample frequency in order to sample with great details.}

\begin{flushleft}
\item Digital-to-analogue Conversion
\end{flushleft}
 
To convert from digital to analogue we not only need to know the binary representation of the sample taken, and also that the sample period/frequency that these sample are taken. The actual converting process is quite self explanatory, let's say the given sequence are the binary representation of the analogue readings taken as sample and are in order where the first binary representation is the reading of the sample at $t=0$

$$0001 \quad0101\quad 0100 \quad0000 \quad 0100 \quad0100\quad0100$$
and the sample period is $1s$, then the graph of the analogue graph of the readings after converted from digital will be  
\begin{figure}[H]
    \centering
    \includegraphics[scale=.2]{communication 6.jpg}
\end{figure}
The graph of conversion from digital to analogue is almost always never going to be the same as the original analogue voltage graph that is being sampled, this is because the binary representation does not contain the information of every point in time, in order to have a higher precision of the analogue graph that is converted from digital, then during the first conversion from the original analogue to digital, we need 
\begin{enumerate}
    \item higher sampling frequency / shorter sampling period. (this reduces the width of the steps)
    \item more binary digits. (when the number of binary digits we use exceeds the binary digit the maximum magnitude of the analogue reading sampled required, the excess bits can be used to convert decimal places, this require additional setting and does not round down to the nearest integer.)
\end{enumerate}


\subsection{Signal Attenuation}

Signal Attenuation is the gradual decrease in signal power due to reasons such as
\begin{enumerate}
    \item Heat loss in cable. (Wire or cable)
    \item Absorption of light pulse by impurities. (Optic fibre)
    \item Scattering of signal due to imperfection of fibre. (Optic fibre)
    \item Scattering or dispersion of signal. (Radio or microwave)
    \item Absorption of signal waves. (Radio or microwave)
\end{enumerate}

To calculate attenuation, we use the given formula 
$$\log \cfrac{P_2}{P_1}$$
where $log$ is logarithm in base $10$. $P_2$ or $P_1$ can be the power of output signal, if $P_2$ is power of output signal then $P_1$ will be power of input signal and vice versa. If $P_2$ is the power of output signal and $P_1$ is power of input signal, then the attenuation will be \textbf{negative} as the output signal is attenuated, hence lower power and causes $\cfrac{P_2}{P_1}<1$ making $\log \cfrac{P_2}{P_1}<0$. If $P_2$ is the power of the input signal and $P_1$ is power of the output signal, then $\cfrac{P_2}{P_1}>1$, hence $\log \cfrac{P_2}{P_1}>0$, the attenuation ratio will be positive. Unit for attenuation is bel, $B$ (a logarithmic scale). Attenuation is usually presented in decibels, $dB$, instead of $B$ (because using $dB$ is easier to handle large numbers) but they are interchangeable
\begin{align*}
    no. \;B&=\log \cfrac{P_2}{P_1}\\
    no. \; dB&=10\log \cfrac{P_2}{P_1}
\end{align*}
A negative bel or decibel implies that the power in the numerator is the power of the output signal because negative bel is only because of $\cfrac{P_2}{P_1}<1$, this only happens because power of output signal is always less than power of input signal due to attenuation. A positive value will be the opposite. \textit{This is merely a means of expressing ratios in the logarithmic form and the sign only indicates the numerator and denominator of the ratio.}

When using wire to transmit signal, the level of noise is important. This is because at regular interval along a cable, repeaters amplifier will amplify the signal, This poses a problem when the signal transmitted is analogue, this is because when the signal power decreases until its power is about the same as noise before reaching repeater amplifier, then when it reaches the repeater amplifier, the noise and the signal power are both amplified, this makes the receiver to have a hard time distinguishing noise and signal. A digital signal will not have such problem. To determine the power of the signal that needed to be use to transmit the signal, signal-to-noise ratio is introduced, it can be calculated as follow 
\begin{align*}
signal-to-noise \;\; ratio&=\log \cfrac{P_{signal}}{P_{noise}}\;B\\
&=10\log \cfrac{P_{signal}}{P_{noise}}\; dB
\end{align*}
for different cable, environment and the distance between repeater amplifier will have different signal-to-noise ratio required for the signal to be able to be transmitted and can be distinguish from noise easily by the receiver.

\newpage
\section{Electric Fields}

 \subsection{Electric Field Strength}

The direction of electric field is from a higher electric potential to a lower electric potential. There are two ways for defining general electric field (electric fields that are not uniform) strength, $E$, at a particular point in space, one of them is as follows
\begin{equation}
    \vec{E}=\cfrac{\vec{F}}{q}
\end{equation}
where $q$ is the charge present in that particular point of space in the electric field, and $F$ is the force experience by the an object with charge $q$ due to the electric field. The second definition will be shown after equation (8).

For uniform electric field such as between two charged plates, it can be defined as
\begin{equation}
  \vec{E}=\cfrac{V}{d}
\end{equation}
where $V$ is the voltage across the two plates and $d$ is the distance between the two plates. The derivation of equation (8) will be shown in the upcoming section. Motion of charged particle in uniform electric field is \textbf{parabolic}.
\newline\\
Now we can define the second definition general electric strength at a particular point.

\begin{theorem}
The electric field strength at a point in space, $\vec{E}$, can be defined as the negative of the instantaneous change in potential difference/voltage, $V$, per unit displacement, $x$
$$\vec{E}=-\cfrac{dV}{dx}$$
\end{theorem}

\begin{proof}
The electric field strength at a point can also be thought of as the instantaneous change in electrical potential per unit distance similar to gravitational field strength. Let's assume that a test charge $+q$ is in a non uniform electric field, and if we zoom in close enough, the electric field lines become straight and that region will have a constant uniform magnetic field strength, let's say this magnetic field strength at the point when we zoomed in is $\vec{E}$. Assuming that the charge $+q$ moves a small distance $dx$ in the field in the direction of field lines, hence it has moved through a small potential $-dV$ (negative because it there is an overall decrease in electrical potential as the charge is moved from higher to lower potential). As such, the charge loses energy. By conservation of energy, work done by the field is equal to the work done on the charge and hence:
$$W=\vec{F}dx$$
$$-dVq=\vec{F}dx$$
$$-qdV=q\vec{E}dx$$
$$\vec{E}=-\cfrac{dV}{dx}$$
\end{proof}
Note that the electric field strength is a vector quantity, hence the electric field strength at a point is the sum of the electric field vector at that point similar to gravitational field strength. In fact, the electric potential a point also works the same way as gravitational potential, but electric potential generated by a charged particle can be positive or negative instead of always being negative like gravitational potential. Formula to calculate electric potential will be further discussed in the next section. \\

Now, let's assume a two charged body systems, if both of them are of the same charge, then the electric field between these two charged particle will include a point where there is zero electric field strength, but there is no point between where the electric potential is zero and also they there will be a point with the most positive/negative (local maximum/minimum) electric potential, this is because when they are same charge the electric potential always have an addition effect. \textbf{\textit{(Note that the point with local maximum electric potential is the point where the field strength is zero, this is because when we plot a graph of electric potential against a variable moving point relative to one of the charge, the local maximum/minimum of the graph will have gradient zero, and by the second definition of electric field, this implies that the electric field at that point will be zero.)}}. In the case of different charge, then there will be no point between the two charged body where the electric field strength is zero because the electric field vector can never cancels each other, moreover there will be at point with the most positive/negative (local maximum/minimum) electric field strength but there will be a point between them where the electric potential is zero because unlike charge will produce electric potential of different signs, hence they will cancels out at some point. 


 \subsection{Coulomb's Law}


Any two charged object in space will exert a force onto each other whether it's attracting force or repelling force, the magnitude of this force can be calculated using the following equation
\begin{equation}
\vec{F}=\cfrac{1}{4\pi\varepsilon_0}\cfrac{q_1 q_2}{r^2}
\end{equation}
where $\varepsilon_0$ is the permittivity of free space ($\approx8.85 \times 10^{-12}Fm^{-1})$, $q_1$ and $q_2$ are the charge of the two objects, and $r$ is the distance between the two charged object.

Substituting equation (9) into (7) we can get an equation for the electric field strength on a particular space that only depends on the charge of the charged object that is creating the electric field and the distance from the charged object. Now let's assume that a particle with charge $Q$, is creating electric field around it and another object with charge $q$ is placed into the electric field at $r$ distance away from $Q$, the electric field strength at that particular point experienced by $q$ is as follows

$$\vec{E}=\cfrac{\cfrac{1}{4\pi\varepsilon_0}\cfrac{Qq}{r^2}}{q}=\cfrac{Q}{4\pi\varepsilon_0r^2}$$

\textit{Note that $\cfrac{1}{4\pi\varepsilon_0}\approx 9\times 10^9 Nm^2C^{-2}$.}


 \subsection{Electric Potential and Electric Potential Energy}

Electric potential, also known as voltage, $V$, can be defined similarly to the gravitational potential, $\phi$, we also define that at infinity the electric potential is \textbf{zero} because it is out of the influence of the electric field, now let's assume a charged object with charge $Q$ creates a radial electric field, the electric potential at a distance of $r$ away from the charged object is given as 
$$V=\cfrac{Q}{4\pi\varepsilon_0r}$$
The electric potential can be positive or negative depending on the charge of $Q$. 

The electric potential energy at a point is defined as the energy required to move an object of charge $q$ from infinity into that particular point in space. Now let's say that the object with charge $q$ is placed into the electric field created by the charge object with charge $Q$ at a distance of $r$ away from it, then the electric potential energy is 
$$W=Vq=\cfrac{Qq}{4\pi\varepsilon_0r}$$

Notice that when $Q$ and $q$ have different signs, then $W$ is always \textbf{negative} this is because different signs will always attract hence the  appearance of the negative sign is similar to gravitational potential energy which represents they are always attracting, when they have the same signs, then $W$ will always be \textbf{positive} because they are repelling each other. When $Q$ and $q$ have the same charge and very close together, we see that $W>>0$, this implies that a lot of work is done onto the test charge to place it closely with $Q$. When they have different charge signs, placing them close together results in $W<<0$, which mean that a lot of work is done by $Q$ to pull $q$ close.


\subsection{Uniform Electric Field}
\begin{theorem}
The electric field strength, $\vec{E}$, in a uniform electric field between two plates which have a potential difference of $V$ across them and a distance $d$ apart can be defined as 
$$\vec{E}=\cfrac{V}{d}$$
\end{theorem}
\begin{proof}
To derive equation (8), we first will have to notice that the electric field strength is constant in a uniform electric field because the electric field lines are evenly spread apart 

\begin{figure}[H]
    \centering
    \includegraphics[scale=.2]{electric field 1.jpg}
\end{figure}
The given diagram is assuming $V_A>V_B$, this implies that the plate $A$ has a higher electric potential then plate $B$, now let's assume that we put a test charge $+q$ at $A$, it will travel to $B$, using equation (7), since the electric field is constant throughout the journey towards $B$ and the charge is constant, we can say that the force acting on the test charge is constant such that 
$$\vec{F}=\vec{E}q$$
Using the definition of work done, we can obtain the following equation
$$W=\vec{F}s=\vec{E}qd$$
where $d$ is the distance travel (distance between $A$ and $B$). Assuming there's no friction, we can conclude the change in electric potential energy, $
\Delta W_p$, given as 
$$\Delta W_p=q\Delta V=q(V_A-V_B)=qV_{AB}$$
of the test charge is equal to the work done on the test charge
$$\vec{E}qd=qV_{AB} \implies \vec{E}d=V_{AB}\implies \vec{E}=\cfrac{V_{AB}}{d}$$
\end{proof}
The equation for the potential of a point $C$ in a uniform electric field can be derived using the above equation

\begin{figure}[H]
    \centering
    \includegraphics[scale=.2]{electric field 2.jpg}
\end{figure}
Since the electric field is uniform in between the plates $A$ and $B$, we can obtain the following relation
$$\vec{E}=\cfrac{V_{AB}}{d}=\cfrac{V_{AC}}{x}=\cfrac{V_{CB}}{y}$$
rearranging the equation yields two equations for the electric potential at point $C$
\begin{align*}
    \vec{E}x&=V_{AC}  & \vec{E}y&=V_{CB} \\
    &= V_A-V_C &  &=V_C-V_B\\ 
    \therefore V_C&=V_A-\vec{E}x  & \therefore V_C&=V_B+\vec{E}y
\end{align*}

\begin{flushleft}
 \textbf{Potential On A Spherical Surface}
\end{flushleft}
\begin{figure}[H]
    \centering
    \captionsetup{justification=centering,margin=2cm}
    \includegraphics[scale=.45]{electric field 3.png}
    \caption*{Electric potential of a charged sphere \\(Source: \textit{Hyperphysics.})}
\end{figure}
When a spherical object has a charge of $Q$, the electric potential on the surface is
$$V=\cfrac{Q}{4\pi\varepsilon_0R}$$
where $R$ is the radius of the spherical object. \textbf{Note that there is no change in electric potential inside the spherical object and the surface of the sphere because there is no electric field inside the spherical object, this is because in a perfect spherical conductor the charges are distributed evenly such that the vector of electric field all add up to zero. Therefore since electric field strength is equal to the instantaneous change of electric potential per unit distance at a point, without electric fields means zero electric field strength, and the field strength inside the sphere is constantly zero everywhere hence there is no change in electric potential. As a result, moving charges inside the spherical object requires no energy,because there is no change in electric potential energy.}

\newpage

\section{Capacitance}

 \subsection{Concept of Capacitor and Capacitance}


Capacitor mainly consists of two plates, when a voltage is set up across them, the plate store charges by discharging charge from one plate to another, this causes one of the plates to be positively charged and the other to be negatively charged, the magnitude of charge of each of the plate is the same because for every charge discharge by one plate is gained by the other. Therefore, the total charge of the capacitor is always \textbf{zero}, when talking about the charge stored in the capacitor, we mean the magnitude of charge of one of the plates.

Capacitance, $C$, is defined as the charge stored at one plate per unit voltage across the plates.
\begin{equation}
    C=\cfrac{Q}{V}
\end{equation}


 \subsection{Energy Stored In A Capacitor}


Before the capacitor is charged, there is no potential difference between the plate, therefore the work done needed to move the first charge from one plate to another by the battery is zero, so no work is done onto the capacitor, hence zero energy is stored by the capacitor. Now let's say after some time $1C$ of charge is removed from the plates that is connected to the positive terminal of the battery to the plate that is connected to the negative terminal. This causes one of the plate to have a charge of $+1C$ and another plate with charge $-1C$, this sets up a potential difference between the two plates, the next $1C$ of charge that needs to be removed from the plate connected to the positive terminal will require more energy because the potential difference is higher now, then the next will be even higher and so on. \textbf{\textit{Note that we can also group the charge as one electron, $e$, instead of grouping them as units of $1C$.}}The graph below shows the voltage across the plates in a capacitor that is connected to battery with emf $V_0$ with respect to time. \textbf{Assuming that the battery delivers a constant power.}
\begin{figure}[H]
    \centering
    \includegraphics[scale=.2]{capacitance.jpg}
\end{figure}
At time $t=T$, the capacitance is charged up until the potential difference between the plates is the same as the potential difference between the terminal of the battery, the charging stops. We can also notice that the gradient is decreasing. This is because when the current decreases as the potential difference between the plate increases, the battery would need to supply more power to maintain the speed of flow of charge or current because it has to move the charges/electrons against a higher potential. However, since the power supplied by the battery is assumed to be constant, therefore the speed of the charge flow will decrease as most of the energy provided is used to gain the electric potential energy of the charge. As a result, the flow of charge decreases, which causes the potential difference between the plate to increase at a lower rate.

\begin{theorem}
The energy or the work done, $W$, to charge a capacitor with capacitance, $C$, to a voltage of $V$ across its plates can be defined as 
$$W=\cfrac{1}{2}QC=\cfrac{1}{2}CV^2=\cfrac{1}{2}\cfrac{Q^2}{C}$$
\end{theorem}

\begin{proof}
The main thing to notice when deriving the formula for the energy stored in the capacitor is that even though the potential difference between the plate is not increasing at constant rate, but the electric potential energy require to move the charge/electron is increasing at a constant rate or is linear for each consecutive group. We may group using $1C$, when the battery is connected, the first $1C$ of charge does not require energy to move from one plate to another because there is no potential difference, when the first $1C$ of charge is moved to the other plate, the potential difference between the plate or the potential difference that the next group of $1C$ will have to move through, $\Delta V_2$, which is defined as
$$\Delta V_2=\cfrac{\Delta Q_2}{4\pi\varepsilon_0r}=\psi\Delta Q_2$$
where $\Delta V_n$ is the potential difference between the plate at the $n$th cycle, $\Delta Q_n$, is the difference between the charge of the plates at the $n$th cycle which in this case, $n=2$ for $\Delta V_n$ and $\Delta Q_n$, $\psi$ is a constant after combining the invariant variables of each cycle in the equation. The difference in charge between the plates $\Delta Q_n$ of the $n$th cycle of removing an electron and returning to the opposite can be easily formulated by the following equation
$$\Delta Q_n=2q_0(n-1) \quad \quad n\in \mathbb{Z}^+$$
where $q_0$ is the charge of one group of charge. This implies that the potential difference across the plates at the $n$ cycle is in fact 
\begin{equation}
    \Delta V_n= 2 \psi q_0 (n-1) 
\end{equation}
It is obvious to see that $\Delta Q_n$ is increasing linearly, therefore the potential difference between the plates, $\Delta V_n$, also increases linearly. Hence we obtain a graph like this 
\begin{figure}[H]
    \centering
    \includegraphics[scale=.2]{capacitance1 .jpg}
\end{figure}
where $n_Q$ is the $n$th number of group of charge and $\Delta V_n$ is the potential difference between the plate during the $n$th group of charge is moving or the potential difference felt by the $n$th cycle of charge during its journey.

To charge up a capacitor until the final potential difference between plates, $\Delta V_f$, to have a potential difference of $V$, we can use this equation of work done to derive a formula for the energy stored in the capacitor
$$W=q\Delta V$$
let's assume that when the capacitor is charged until the potential difference between the plates is $V$, the magnitude of charge of one of the plate is $Q$ after it is charged to have the desired voltage across the plate, and also that it takes $k$ number of groups of charge to charge up the capacitor until the stated potential difference between the plates is obtained. Let's also say that the charge of each group, $q_0$, is the same for each of the $k$ cycles, this gives us the following relations.
$$q_0=\cfrac{Q}{k}$$
\begin{equation}
    \Delta V_f=\Delta V_{k+1}=V=2\psi \cfrac{Q}{k} (k+1-1)=2\psi Q
\end{equation}
The potential that the $(k+1)$th group of charge has to move through is equal to the potential between the plate when is it charged to our desired voltage instead of the $k$ cycle is because we assumed that the $k$th cycle's group of charge is the final group to increase the potential difference between the plates to the  desired voltage, which is $V$, this means that after the $k$th cycle has completed its journey, the potential difference between the plates would then be $V$, so the potential that the next cycle, $(k+1)$th cycle, would have to move through is the desired potential difference. The work done to charge the resistor will logically be the sum of energy required to move each group of charge through their respective potential difference during that cycle. This yields 
$$W=q_0\Delta V_1+q_0\Delta V_2+\dots +q_0\Delta V_k=q_0\sum \limits_{n=1}^{k}\Delta V_n=\cfrac{Q}{k}\sum \limits_{n=1}^{k}\Delta V_n$$
Note that since the current does not carry the charge one group after the other, instead the current carries a stream of charges, hence if the number of cycle is larger, the better the approximation of work done will be, therefore we get the following equation
$$W=\lim _{k\rightarrow \infty}\cfrac{Q}{k}\sum \limits_{n=1}^{k}\Delta V_n$$
expanding the equation using equation (11) and using the result of equation of (12) we will get the desired result
\begin{align*}
    W=\lim _{k\rightarrow \infty}\cfrac{Q}{k}\sum \limits_{n=1}^{k}\Delta V_n&=\lim _{k\rightarrow \infty}\cfrac{Q}{k}\sum \limits_{n=1}^{k}2 \psi q_0 (n-1) \\ &=\lim _{k\rightarrow \infty} 2\cfrac{Q^2}{k^2}\psi \sum \limits_{n=1}^{k}(n-1)\\ &= \lim _{k\rightarrow \infty} 2\cfrac{Q^2}{k^2}\psi \left(\cfrac{k(k+1)}{2}-k\right) \\ &= \lim _{k\rightarrow \infty} \cfrac{1}{2}(2\psi Q)\left(\cfrac{k^2-k}{k^2} \right)Q\\ &= \lim _{k\rightarrow \infty} \cfrac{1}{2} V\left(1-\cfrac{1}{k}\right)Q \\ &= \cfrac{1}{2}QV=\cfrac{1}{2}CV^2=\cfrac{1}{2}\cfrac{Q^2}{C} 
\end{align*}
\end{proof}

\begin{flushleft}
An alternative derivation for \textbf{Theorem 9.2.1} can be done by integration
\end{flushleft}
\begin{proof}
Let the charges on the plates be +q and -q, so moving a small element of charge $dq$ across the plates against a potential difference of $V$ will require work done of $dW$ where 
$$V=\cfrac{q}{C}$$
Hence,
$$dW=Vdq$$
$$dW=\cfrac{q}{C}dq$$
Integrating both sides will give,
$$W_{charging}=\int \limits_{0}^{Q}\cfrac{q}{C}dq$$
Hence,
$$W_{stored}=W_{charging}=\cfrac{1}{2}\cfrac{Q^2}{C}=\cfrac{1}{2}QV=\cfrac{1}{2}CV^2$$
\end{proof}



 \subsection{Capacitor in Series and Parallel}


\textbf{Capacitor in Parallel}
\begin{figure}[H]
    \centering
    \includegraphics[scale=.2]{capacitance 2.jpg}
\end{figure}

\begin{theorem}
The total capacitance, $C_T$ across a parallel setup of capacitor is defined as 
$$C_T=\sum\limits_{\forall n}C_n$$
where $C_i$ is the total capacitance in a 'level' of the parallel setup
\end{theorem}

\begin{proof}
Let's define the total capacitance of this configuration as $C_T$, also can be expressed as
$$C_T=\cfrac{Q}{V}$$
where $V$ is the voltage across the parallel setup, we can also conclude that 
$$V=V_1=V_2$$
this yields two equation 
\begin{align*}
    C_1&=\cfrac{Q_1}{V_1} & C_2&=\cfrac{Q_2}{V_2} \\
      C_1V_1&=Q_1 & C_2V_2&=Q_2 \\
       C_1V&=Q_1 & C_2V&=Q_2
      \end{align*}
since 
$$Q=Q_1+Q_2$$
we get the following result 
$$C_T=\cfrac{Q}{V}=\cfrac{C_1V+C_2V}{V}=C_1+C_2$$

By similar approach we can derive the formula for $n$ capacitors in parallel. The general result for $n$ capacitor being arranged in parallel is 
$$C_T=\sum\limits_{\forall n}C_n$$
\end{proof}

\textbf{Capacitor in Series}
\begin{figure}[H]
    \centering
    \includegraphics[scale=.2]{capacitance 3.jpg}
\end{figure}

\begin{theorem}
The total capacitance, $C_T$, across a series setup of $n$ capacitors is defined
$$\cfrac{1}{C_T}=\sum\limits_{\forall n} \cfrac{1}{C_n}$$
\end{theorem}

\begin{proof}
Let's also define the total capacitance of this configuration to be $C_T$, where
$$C_T=\cfrac{Q}{V}\implies V=\cfrac{Q}{C_T}$$
and using Kirchoff's 2nd law, we can conclude that 
$$V=V_1+V_2$$
from the above diagram we get the following equations
\begin{align*}
    C_1&=\cfrac{Q}{V_1} & C_2&=\cfrac{Q}{V_2} \\
    V_1 &=\cfrac{Q}{C_1} & V_2&=\cfrac{Q}{C_2}
\end{align*}
using these results we can conclude that 
\begin{align*} 
    \cfrac{Q}{C_T}&=\cfrac{Q}{C_1}+\cfrac{Q}{C_2}\\
    \implies \cfrac{1}{C_T}&=\cfrac{1}{C_1}+\cfrac{1}{C_2}
\end {align*}

The general result for $n$ capacitor in series is as follows
$$\cfrac{1}{C_T}=\sum\limits_{\forall n}\cfrac{1}{C_n}$$
\end{proof}

 \subsection{Charging and Discharging a Capacitor}

\begin{flushleft}
\textbf{Charging a Capacitor}
\end{flushleft}

\begin{theorem}
In a circuit with a battery/power supply with emf $\varepsilon$, a total of resistance $R$ in the circuit, and a capacitor with capacitance $C$, the magnitude of charge $q(t)$ at time $t$ at one of the plates of the capacitor during charging is defined as
$$q(t)= C\varepsilon\left(1- e^{-\cfrac{t}{RC}}\right)$$
\end{theorem}

\begin{proof}
To derive a formula for the charge on the plate, current in the circuit and the voltage across the plate, we can first imagine a simple circuit
\begin{figure}[H]
    \centering
    \includegraphics[scale=.2]{capacitance 4.jpg}
\end{figure}
where $\varepsilon$ is the emf of the battery, $I$ is the current flowing at the time, $R$ is the resistance of the resistor, $C$ is the capacitance of the capacitor and $V_C$ is the voltage across the capacitor. Using Kirchoff's 2nd law we can construct the following equation
$$\varepsilon - IR-V_C=0$$
we can use equation(10) to substitute for $V_C$, and notice that the rate of change of the charge on the plate is the current flowing at the time, let's say that at time $t$ the charge on the plate is $q(t)$, since at any time $t$, Kirchoff's 2nd law still holds, therefore we can construct the following relation and then solve for 
\begingroup
\allowdisplaybreaks
\begin{align}
    \varepsilon-\cfrac{dq(t)}{dt}R-\cfrac{q(t)}{C}&=0 \nonumber \\ 
    \implies \cfrac{dq(t)}{dt} &=\cfrac{1}{R}\left(\varepsilon-\cfrac{q(t)}{C}\right) \nonumber \\ 
    \implies \cfrac{dq(t)}{\left(\varepsilon-\cfrac{q(t)}{C}\right)}&=\cfrac{dt}{R} \nonumber \\ 
    \implies \int \limits_{q(0)}^{q(t)}  \cfrac{dq(t)}{\left(\varepsilon-\cfrac{q(t)}{C}\right)} &=\int  \limits_0^t\cfrac{dt}{R} \\
    \implies \left.-C\;ln \left(\varepsilon-\cfrac{q(t)}{C}\right)\right|_0^{q(t)} &= \left.\cfrac{t}{R} \right|_0^t\nonumber \\ \implies ln\left(\cfrac{\varepsilon-\cfrac{q(t)}{C}}{\varepsilon}\right)&=-\cfrac{t}{RC} \nonumber \\ \implies q(t)&= C\varepsilon\left(1- e^{-\cfrac{t}{RC}}\right) \nonumber
\end{align}
\endgroup
\end{proof}
In equation (13) we integrate from time $t=0$ to $t$ where it is a variable, plugging $t$ will give us the sum of all charge up to that point in time which is also equivalent to the charge stored in the plate $q(t)$, therefore the boundary for the integral on LHS is from $q(0)$ to $q(t)$ to the time that we set as variable. Note that $q(0)=0$, because at $t=0$, there is no charge stored on the plate. After some time, the charge on the capacitance will be $C\varepsilon$ because the voltage across the capacitor will be $\varepsilon$ after charging for a while
\begin{figure}[H]
    \centering
    \includegraphics[scale=.2]{capacitance 6.jpg}
\end{figure}

\begin{theorem}
In a circuit with a battery/power supply with emf $\varepsilon$, a total of resistance $R$ in the circuit, and a capacitor with capacitance $C$, the magnitude of current $I(t)$ at time $t$ in the circuit during charging is defined as
$$I(t)=I_0e^{-\cfrac{t}{RC}}$$
where $I_0$ is the initial current of the circuit
\end{theorem}

\begin{proof}
Like we have defined earlier that the rate of change of charge on the plate equals to the current at that time, we arrive at the following equation
$$I(t)=\left.\cfrac{dq(t)}{dt}\right|_{t=t}=\left.\cfrac{d}{dt}\left(C\varepsilon\left(1-e^{-\cfrac{t}{RC}}\right)\right)\right|_{t=t}=\cfrac{\varepsilon}{R}e^{-\cfrac{t}{RC}} $$
Note that $\cfrac{\varepsilon}{R}=I(0)=I_0$, this is because when $t=0$, the current in circuit only runs through the resistor because there's no potential difference in capacitor to resist the current. Therefore, we can conclude the equation for the current with in terms of time is as
$$I(t)=I_0e^{-\cfrac{t}{RC}}$$
\end{proof}

The shape of the graph is as follow
\begin{figure}[H]
    \centering
    \includegraphics[scale=.2]{capacitance 5.jpg}
\end{figure}

\begin{theorem}
In a circuit with a battery/power supply with emf $\varepsilon$, a total of resistance $R$ in the circuit, and a capacitor with capacitance $C$, the magnitude of voltage $V(t)$ at time $t$ in the circuit during charging is defined as
$$V(t)=\varepsilon\left(1- e^{-\cfrac{t}{RC}}\right)$$
\end{theorem}

\begin{proof}
To derive an equation for the voltage at time $t$ across the capacitor, we can just divide the charge at time $t$, $q(t)$ by the capacitance $C$
$$V(t)=\cfrac{q(t)}{C}=\varepsilon\left(1- e^{-\cfrac{t}{RC}}\right)$$
\end{proof}
The graph of this function can be seen in section for deriving the energy stored in a capacitor.

The time constant for these three decays can be expressed as 
$$\tau=NRC \quad\quad N\in\mathbb{Z}$$
\begin{flushleft}
\textbf{Discharging a Capacitor}
\end{flushleft}

\begin{theorem}
In a circuit with total resistance of $R$ and a capacitor with capacitance $C$ that is charged to a voltage of $\varepsilon$, the magnitude of charge, $q(t)$, at one of the plates of the capacitor at time $t$ during discharging is defined as 
$$q(t)=q_0e^{-\cfrac{t}{RC}}$$
where $q_0$ is the magnitude of the initial charge at one of the plates.
\end{theorem}

\begin{proof}
To derive equations for the charge, current and voltage across the plate at time $t$ while discharging, we can imagine another simple circuit which only consists of a charged capacitor with $\varepsilon$ volts across the plates and a resistor with resistance $R$ connected to it 
\begin{figure}[H]
    \centering
    \includegraphics[scale=.2]{capacitance 7.jpg}
\end{figure}
Using Kirchoff's 2nd law, we can build the equation below
$$\varepsilon-IR=0$$
using similar idea, we can say that the \textbf{negative} rate of change of charge on the plate at time $t$ is equal to the current at time $t$, negative because the charge on the plate is decreasing therefore the rate of change is negative, and since current is the magnitude of this changes, therefore we should take the absolute value of the rate of change of charge to be the current, but this would make the integral harder, instead we multiply by a negative sign to obtain the absolute which is also equivalent to the definition of the absolute value of a negative value
this yields

\begin{align}
    \cfrac{q(t)}{C}-\left(-\cfrac{dq(t)}{dt}\right)R&=0 \nonumber \\ \implies \cfrac{q(t)}{C}&=-\cfrac{dq(t)}{dt}R \nonumber \\\implies  -\cfrac{1}{RC}\int \limits_{0}^t dt&=\int\limits_{q(0)}^{q(t)}\cfrac{dq(t)}{t} \nonumber \\ \implies -\cfrac{t}{RC}&=ln\left(q(t)\right)\bigg|_{q_0}^{q(t)} \nonumber \\ \implies q(t)&=q_0e^{-\cfrac{t}{RC}} 
\end{align}

where $q_0$ is the initial charge at the plate, which is also equivalent to $q(0)$.
\end{proof}

\begin{theorem}
In a circuit with total resistance of $R$ and a capacitor with capacitance $C$ that is charged to a voltage of $\varepsilon$, the magnitude of current, $I(t)$, flowing in the circuit at time $t$ during discharging is defined as
$$I(t)=\cfrac{q_0}{RC}e^{-\cfrac{t}{RC}}$$
where $q_0$ is the initial charge at one of the plates of the capacitor.
\end{theorem}
\begin{proof}
Differentiating equation (14) and evaluating at time $t=t$  we get the current at time $t$
$$I(t)=\left.-\cfrac{dq(t)}{dt}\right|_{t=t}=\cfrac{q_0}{RC}e^{-\cfrac{t}{RC}}$$
\end{proof}

\begin{theorem}
In a circuit with total resistance of $R$ and a capacitor with capacitance $C$ that is charged to a voltage of $\varepsilon$, the magnitude of voltage, $V(t)$, across the plates of the capacitor at time $t$ during discharging is defined as
$$V(t)=\varepsilon e^{-\cfrac{t}{RC}}$$
\end{theorem}
\begin{proof}
Dividing equation (14) with the capacitance of the capacitor, $C$, we get the voltage across the plate at time $t$
$$V(t)=\cfrac{q(t)}{C}=\cfrac{q_0}{C}e^{-\cfrac{t}{RC}}=\varepsilon e^{-\cfrac{t}{RC}}$$
\end{proof}
\newpage 
 \section{Electronics}
 
 \subsection{Variable Resistor}
 
 \begin{flushleft}
 \textbf{Light Dependent Resistor (LDR)}
 \end{flushleft}
 A semiconductor that is sensitive to the intensity of light (energy of photons), when intensity of light is increased, or in other words the photons absorbed is sufficient to bring the electron from valence band to conduction band, then the resistance of LDR is decrease. 
 
  \begin{flushleft}
 \textbf{Thermistor}
 \end{flushleft}
 A resistor that is sensitive to heat, when temperature is increased, the resistance of thermistor will be decreased and vice versa.
 
 \begin{flushleft}
 \textbf{Metal-Wire Strain Gauge}
 \end{flushleft}
 \begin{figure}[H]
     \centering
     \captionsetup{justification=centering,margin=2cm}
     \includegraphics[scale=.5]{electronics 1.jpg}
     \caption*{Wire strain gauge \\ (Source: \textit{ForumAutomation.com})}
 \end{figure}
 Metal-wire strain gauge uses the fact that the length and cross-sectional area will affect the resistance of a wire to function. When a tensile or compression strain is exerted on the metal-wire strain gauge in the direction stated in the diagram, a change of resistance will be observed when it is probe to a ohm meter. To find the relation between the change in resistance and the change in length, we may assume that for small change in $l$ the area $A$ is unchanged, this works because the extension of the metal wire is usually very small, using the equation for the resistance of wire
$$R=\rho\cfrac{l}{A}$$
 and we can deduce the following since $\rho$ and $A$ is unchanged
 
 \begin{theorem}
 The change in resistance, $\delta R$, of the metal-wire strain gauge is directly proportional the change in length, $\delta l$, of the wire
 $$\delta R \propto \delta l$$
 \end{theorem}
 \begin{proof}
 \begingroup
 \allowdisplaybreaks
\begin{align*}
    R+\delta R &=\rho\cfrac{l+\delta l}{A} \\ 
    R+\delta R &=\rho\cfrac{l}{A}+\rho\cfrac{\delta l}{A} \\ 
    R+\delta R &=R+\rho\cfrac{\delta l}{A} \\ 
    \therefore \delta R &=\rho\cfrac{\delta l}{A} \implies  \delta R  \propto \delta l 
\end{align*}
\endgroup \end{proof}

\subsection{Output Devices}
\begin{flushleft}
\textbf{Light Emitting Diode}
\end{flushleft}
Semiconductor which emits light when it is forward bias
\begin{flushleft}
\textbf{Electric Relay}
\end{flushleft}
An electromagnetic automatic switch that uses a small current or pd to switch on or off another circuit which has a large current or pd


 
 \subsection{Operational Amplifier (Op-Amp)}
 
 \begin{figure}[H]
     \centering
     \includegraphics[scale=.15]{electronics 2 (1).jpg}
 \end{figure}
 Op-amp is a device that takes difference in potential between the non-inverting input and inverting input $(V^+-V^-)$ and amplifies it by a magnitude of around $10^5-10^6$to produce an output voltage ,$V_{out}$. But in actuality $V_{out}$ will not be close to that magnitude because in normal uses, a power supply with fixed voltage is connected across the op-amp  ($+V_s$ is connected to the anode while $-V_s$is connected to the cathode of the power supply), this bounds the output voltage to only be in between $+V_s$ and $-V_s$, when $V_{out}$ reaches this upper and lower bound, it is said to be \textbf{saturated}. 
 
 When the op-amp has no feedback loop, then the gain of the op-amp is called open-loop gain, denoted by $G_o$.Closed-loop gain will be discussed in later sections. The open-loop gain is defined as 
 $$G_o=\cfrac{V_{out}}{V^+-V^-}$$
 
 \begin{tcolorbox}[colframe=black!20!white,title=
 \color{red}{Properties of ideal op-amp}]
 \begin{enumerate}
 \color{red}
     \item Infinite open-loop gain.
     \item Infinite slew rate. (the output voltage changes instantaneously when input voltage is changed)
     \item Infinite input impedance. (to ensure that all the input voltage drops to zero when fed into the op-amp so that all the voltage are amplified)
     \item Zero output impedance. (to ensure no voltage drop when delivering $V_{out}$ so that we can assume that all the $V_{out}$ are all there is and none is loss in the op-amp)
     \item Infinite bandwidth. (function ideally for all ranges of frequency of signals)
     \item Zero noise
 \end{enumerate}
 \end{tcolorbox}
 
 \begin{flushleft}
 \textbf{Negative Feedback} 
 \end{flushleft}
 A feedback is when a fraction of the output is fed into the input of the op-amp. A negative feedback is when a fraction of the output is delivered to the inverting input. A general rule of thumb for negative feedback is that the circuit will always try to make the input of the non-inverting input and inverting input equal. The simplest negative feedback is as follows
 \begin{figure}[H]
     \centering
     \includegraphics[scale=.18]{electronics 3.jpg}
 \end{figure}
 
 \begin{theorem}
 In a zero-resistance non-inverting amplifier, the close loop gain, $G_c$ is defined as $$G_c=\cfrac{C_o}{1+G_o}$$
 \end{theorem}
 \begin{proof}
 In this case, $V_{in}$ is the voltage that we control, the voltage input to the op-amp at a certain is $V^+-V^-$ at that cycle instead of $V_{in}$, but in cases like this where the initial $V^-=0$, then the first cycle of input fed into op-amp is $V_{in}$, when we first apply voltage $V_{in}$ to $V^+$, the voltage that is fed into the op-amp is $V^+-V^-=V_{in}-0=V_{in}$, this would then produce $V_{out,1}$ equivalent to $G_oV_{in}$ where $G_o$ is the open-loop gain. When an output voltage is established, then the next cycle of input fed into the op-amp will be $V^+-V^-=V_{in}-V_{out,1}$ and the output voltage of the second cycle will be $V_{out,2}=G_o(V_{in}-V_{out,1})$ and this cycle of process will continue  so that we get the following recursive relation
 $$V_{out,n}=G_o(V_{in}-V_{out, (n-1)})$$
 this process will eventually stabilise when $V_{out,n}=V_{out,(n-1)}=V_{out,f}$ where $V_{out,f}$ is the final stabilised output voltage, this yields
 \begin{align*}
     V_{out,f} &=G_o(V_{in}-V_{out,f})\\
     (1+G_o)V_{out,f} & =G_oV_{in}\\
     \cfrac{V_{out,f}}{V_{in}}=\cfrac{V_{out,f}}{Initial \; (V^+-V^-)}& =\cfrac{G_o}{1+G_o}=G_c
 \end{align*}
 \end{proof}
 where $G_c$ is the closed-loop gain, one of its definitions being the ratio of the final stabilised voltage output to the \textbf{initial} voltage difference of the non-inverting and inverting input (the second ratio from the left in the above equation), realise that the open loop $G_o$ gain is said to be large, this makes $G_c\approx 1$ which satisfy the general rule of thumb of feedback op-amp circuit where the $Final(V^+)=Final(V^-)$ because $Final(V^-)=V_{out,f}\approx V_{in}=Final(V^+)$. \textbf{One important concept to notice here is that the time taken when the voltage $V_{in}$ is first supplied to the final stabilised $V_{out,f}$ to be obtained is zero in an ideal op-amp because of infinite slew rate, so that as soon $V^+-V^-$ changes $V_{out}$ change instantaneously, therefore, it take zero time for these infinite cycle to complete to achieve a stabilised output}. Another definition for the closed-loop gain is the ratio of the final stabalised output voltage, $V_{out,f}$ to our varying input, $V_{in}$.(the first ratio from the left in the above equation)

When our $V_{in}$ is an ac source, then at each point in time $V^+-V^-$ is the initial $V^+-V^-$, and stabilised output voltage is instantly produced at that particular time. Although the above derivations assumes that the $V_{in}$ is dc but it also works in ac because in ac, we can take any point in take and let the $V^+-V^-$at that to be the initial $V^+-V^-$ and the rest will be exactly identical (this is applicable to all the derivation below when the $V_{in}$ is ac). Note that the closed-loop gain of the op-amp is independent of the initial $V^+-V^-$ as long as the op-amp isn't changed (constant open-loop gain).
\begin{flushleft}
\textbf{Non-Inverting amplifier}
\end{flushleft}
 Non-inverting amplifier also uses negative feedback to achieve its function, it is called an non-inverting amplifier is because the gain is positive, which implies when the initial input fed into the op-amp is positive or in other word the initial $V^+-V^->0$ then $V_{out,f}$ will also be positive as the name implies, and if the initial input fed is negative, then so does the $V_{out,f}$. (Output voltage and input voltage (when initial $V^-=0$) are in-phase.)
 
 A non-inverting amplifier can be achieved by setting the op-amp in this configuration
 \begin{figure}[H]
     \centering
     \includegraphics[scale=.18]{electronics 4.jpg}
 \end{figure}
 
 \begin{theorem}
 The closed loop gain, $G_c$, of an inverting amplifier op-amp is defined as 
 $$G_c=1+\cfrac{R_1}{R_2}$$
 where $R_1$ is the resistance of the resistor between the potential point $V_{out}$ and $V^-$ whereas $R_2$ is the resistance of the resistor between the potential point $V^-$ and the earth.
 \end{theorem}
 \begin{proof}
 The initial input fed into the op-amp will be $V_{in}$ for reasons discussed in previous example, and here, working backwards will help us derive a neater formula for the closed-loop gained, by using the fact that all negative feedback op-amp will try to equalise the input of inverting and non-inverting input, let's assume that the output voltage has already been stabilised to give $V_{out,f}$, notice that only a fraction of $V_{out,f}$ is connected to the inverting input, namely
 $$V^-=V_{x,f}=\cfrac{R_2}{R_1+R_2}V_{out,f}$$
 where $V_{x,f}$ is the final stabilised voltage $V_x$. Now we can construct the following relation because the op-amp has already stabilised the voltage output
 \begin{align*}
     Final\;(V^+-V^-)&=0 \\ 
     Final \; (V^+) &= \; Final \; (V^-) \\ 
     Final \; (V^+) = \; Initial \; (V^+)= \; Initial \; (V^+-V^-)= V_{in} &=\; Final \; (V^-) = V_{x,f}= \cfrac{R_2}{R_1+R_2}V_{out,f}\\
     \implies \cfrac{V_{out,f}}{Initial \; (V^+-V^-)}= \cfrac{V_{out,f}}{V_{in}} &= 1+\cfrac{R_1}{R_2}= G_c
 \end{align*}
\end{proof}


 \begin{flushleft}
 \textbf{Inverting amplifier}
 \end{flushleft}
 
 An inverting amplifier is similar to non-inverting amplifier but the amplification is negative, this means that the magnitude of the $V_{out,f}$ is amplified but the sign of the voltage will be opposite ($180^{\circ}$ out of phase with  $V_{in}$ if ac is connected to $V_{in}$). These characteristic of inverting amplifier will be explained through its formula derivation.
 
 To achieve this conditions, configuration such as the setup below will be enough 
 \begin{figure}[H]
     \centering
     \includegraphics[scale=.2]{electronics 5 (1).jpg}
 \end{figure}
 
 \begin{theorem}
 The closed loop gain, $G_c$, of inverting amplifier op-amp is defined as 
 $$G_c=-\cfrac{R_f}{R_{in}}$$
 where $R_f$ is the resistance of the resistor between the potential points $V_{out}$ and the virtual earth whereas $R_{in}$ is the resistance between the potential point $V_{in}$ and the virtual earth.
 \end{theorem}
 \begin{proof}
 In this proof, we start off by using the fact that in a negative feedback loop op-amp, the $Final \; (V^+-V^-)=0$ and since $Final \; (V^+)=\;Initial \; (V^+) $ because it is connected to ground, we get 
 $$Final \; (V^+)=\; Initial \; (V^+)= 0 = Final \; (V^-)=V_{x,f}$$
 When the output voltage has been stabilised, let's assume that the direction of current flow as follow
 \begin{figure}[H]
     \centering
     \includegraphics[scale=.18]{electronics 6.jpg}
 \end{figure}
 Applying Kirchoff's 1st law at the point with potential $V_x$ we get 
 $$I_{in}=I_f+I_o$$
but since the input impedance of an ideal op-amp is infinite, this implies that $I_o=0$, hence we get the following results
\begin{align*}
    I_{in} &= I_f \\
    \cfrac{V_{in}-V_{x,f}}{R_{in}} &= \cfrac{V_{x,f}-V_{out}}{R_f} 
\end{align*}
 since in has already reached stability, that means $V_{x,f}=0$ so that the non-inverting equal to the inverting input which is grounded. This gives us 
 \begin{align*}
     \cfrac{V_{in}-0}{R_{in}} &= \cfrac{0-V_{out,f}}{R_f} \\
     \cfrac{V_{in}}{R_{in}} &= -\cfrac{V_{out,f}}{R_f} \\
     \implies \cfrac{V_{out,f}}{V_{in}}&=-\cfrac{R_f}{R_{in}}
 \end{align*}
 Since one of the definitions for the closed-loop gain of an op-amp circuit is the ratio of the final stabalised output voltage to the voltage we supplied, therefore we have completed the derivation of the formula for the closed-loop gain of an inverting amplifier configuration
 $$\cfrac{V_{out,f}}{V_{in}}=G_c=-\cfrac{R_f}{R_{in}}$$
 \end{proof}
 Notice that $G_c$ is always negative, this is because when looking at the ratio on the right, both numerical values of the resistance are always positive, as a result the fraction $\cfrac{V_{out,f}}{V_{in}}$ will also be negative, this explains why the voltage of input $V_{in}$ is always out of phase with the final stabalised output voltage $V_{out,f}$ due to the fact that if the numerator and denominator have the same sign, then the ratio is never negative, thus a negative ratio implies that the signs of numerator and denominator are not the same, therefore the voltages are out of phase. 
 
 
 
\begin{tcolorbox}[colframe=black!20!white,title=\textcolor{red}{Effects of negative feedback}]
 
 \begin{enumerate}
 \color{red}
     \item Reduces gain
     \item Stable voltage output
     \item Less distortion/prevents output saturation
     \item Greater bandwidth
 \end{enumerate}
 \end{tcolorbox}
 
\newpage
\section{Magnetic Fields and Electromagnetism}

 \subsection{Magnetic Field}


We usually denote magnetic field strength, also known as the magnetic flux density by the symbol, $B$, with unit $T$, tesla. Just like electric field lines, magnetic field lines also have a direction, the direction for magnetic field lines is from north pole to the south pole.

To identify the direction of the magnetic field lines, there are various method for various situations. In the case of solenoids

\begin{figure}[H]
    \centering
    \captionsetup{justification=centering,margin=2cm}
    \includegraphics[scale=.2]{magnetic field (1).jpg}
    \caption*{Magnetic field of a solenoid \\ (Source: \textit{Mini physics})}
\end{figure}
there are two ways to identify the poles of the magnetic field and the direction of the magnetic field lines. First, we can use right-hand grip rule, in this case the thumb is not representing the current but instead the direction of magnetic field lines, while the other four fingers follow the direction of current in the coil, in the diagram, the thumb will be pointing to the right, hence the right side is the north pole. The second method is by looking through the middle of the solenoid from one of the ends and identify the path of current to determine the pole of the end, there are two possible current path 

\begin{figure}[H]
    \centering
    \captionsetup{justification=centering,margin=2cm}
    \includegraphics[scale=.2]{magnetic field 2.jpg}
    \caption*{North and South pole of a solenoid \\ (Source: \textit{MammothMemory})}
\end{figure}

which can be represented as S and N, when looking through the centre of the solenoid in the diagram from the left, we can see that it resembles the letter S, which indicates that the left end is the south pole.

Another case is the direction of magnetic field around a current carrying wire.

\begin{figure}[H]
    \centering
    \captionsetup{justification=centering,margin=2cm}
    \includegraphics[scale=.35]{magnetic field 3.jpg}
    \caption*{Direction of magnetic field lines around a conducting wire \\ (Source: \textit{physics.stackexchange})}
\end{figure}

We can use the right hand grip rule for this case, but in this case, the thumb represent the current and the other fingers represent the direction of the magnetic field.


 \subsection{Motor Effect}

When a current conducting conductor is placed in a uniform magnetic field, a force is produced on the conductor, or rather, a force is produced onto the electron that is passing through the magnetic field in the conductor. The magnitude force is given by 

\begin{equation}
\vec{F}=BIl\sin\theta
\end{equation}

where $B$ is the magnetic field strength, $I$ is the current flowing through the conductor and $l$ is the length of conductor in the magnetic field. We can group $\sin \theta $ with $B$ which give us the strength of magnetic field that is perpendicular to the conductor or group it with $l$ which gives us the length of conductor that is perpendicular to the magnetic field.

\begin{theorem}
The magnetic force, $\vec{F}$ acting on a particle with charge $q$, moving with a velocity $\vec{v}$ in a magnetic field with strength $B$ and with an angle $\theta$ between the direction of motion and direction of magnetic field lines is defined as
$$\vec{F}=Bq\vec{v}\sin \theta$$
\end{theorem}
\begin{proof}
We can also use equation (15) to derive the equation for the force acting on each electron when travelling in the magnetic field, note that the force acting on the conductor can be expressed the sum of all forces acting on the electron in the conductor that passing through the magnetic field at that time, which gives us the following result

\begin{align*}
    \vec{F}&=n\vec{F}_e \\ BIl\sin \theta &=n\vec{F}_e \\ B\left(\cfrac{nq}{t}\right)l\sin\theta&=n\vec{F}_e  \\ \implies \vec{F}_e&=Bq\cfrac{l}{t}\sin\theta\\ &=Bq\vec{v}\sin\theta
\end{align*}
\end{proof}
where $\vec{F}_e$ is the force acting on one electron in the conductor when passing through the magnetic field, $q$ is the charge of the electron, $\vec{v}$ is the drift speed of the electron in the conductor.


The direction of the force acting on the electron in the conductor can be identify using Fleming Left Hand Rule

\begin{figure}[H]
    \centering
    \captionsetup{justification=centering,margin=2cm}
    \includegraphics[scale=.13]{magnetic field 4.jpg}
    \caption*{Fleming's Left Hand Rule \\ (Source: \textit{PNGWING})}
\end{figure}


 \subsection{Charged Particle in Magnetic Field and Electric Field}


When a charged particle is moving through a magnetic field, the force acting on it is always perpendicular to the velocity component of the charged particle that is perpendicular to the magnetic field. This doesn't mean that the direction of force is parallel to the direction of magnetic field. Since the force is acting perpendicular to the component of the velocity that is perpendicular to the magnetic fields, that implies that the magnitude of this velocity component is not change but its direction is, this is exactly the same case as centripetal force, therefore the charged particle will travel in a circular path in a magnetic field. \textbf{Note that the other component of the velocity after dissecting will be parallel to the magnetic field if the other one is perpendicular to the magnetic field, hence the one parallel to the magnetic field will keeps its magnitude and direction because according to equation (15), when $\theta=0$, $\vec{F}=0$.}

\begin{figure}[H]
    \centering
    \captionsetup{justification=centering,margin=2cm}
    \includegraphics[scale=.5]{magnetic field 5.jpg}
    \caption*{Motion of charged particle around conducting wire \\ (Source: \textit{The Open Door Website})}
\end{figure}

Now let's assume that a charge particle is mass $m$, charge $q$, travelling at a speed $v$, towards a magnetic field lines with angle $\theta$ and magnetic field strength $B$, we can first get the velocity component that is perpendicular to the magnetic field as follow

$$\vec{v}_{\perp}=\vec{v}\sin \theta$$
Then the force acting on the particle would be 
$$\vec{F}=Bq\vec{v}_{\perp}$$
Since the force is acting perpendicular to $\vec{v}_{\perp}$, that implies $\vec{v}_{\perp}$ is also the tangential speed, we obtain the following equation
$$Bq\vec{v}_{\perp}=\cfrac{m\vec{v}_{\perp}^{\;2}}{r}$$
rearranging gives us the formula for the radius of the \textbf{circular path}
$$r=\cfrac{m\vec{v}_{\perp}}{Bq}$$
Note that the radius only depends on the component of the speed that is perpendicular to the magnetic field. Since $\vec{v}_{\perp}$ is constant throughout the journey of the circular path, we can substitute 
$$\vec{v}_{\perp}=\cfrac{2\pi r}{T}$$

where $T$ is the period to complete one revolution, into the equation to get an equation for the period of one complete revolution.

When a charged particle is sent through a region that contain magnetic and electric fields that are set up in such a way that the magnetic force produced is in the opposite direction of the electric force exerted on the charge particle, we can obtain a relation between them. Notice that the magnetic force on the charged particle depends on the velocity (assuming the charged particle travels perpendicularly to the magnetic field)

$$\vec{F}=Bq\vec{v}$$

while the electric force does not depend on velocity

$$\vec{F}=Eq$$

When the electric and magnetic field are set up perfectly such that the charged particle pass right through the field without deflecting that implies the electric force exerted on the charge particle is equal in magnitude to the magnetic force

$$Bq\vec{v}=Eq$$

rearranging give us the ideal initial speed when entering the fields

$$\vec{v}=\cfrac{E}{B}$$

when the initial speed entering the fields is less that the fraction, then the electric force will dominate and cause the charged particle to move in a \textbf{parabolic path}, if the initial speed is more that the fraction, then magnetic force will dominate and cause the charged particle to move in a \textbf{circular path}.


 \subsection{The Hall Effect}


\begin{figure}[H]
    \centering
    \includegraphics[scale=.2]{magnetic field 6.jpg}
\end{figure}

In the above figure, we can see that when a current is pass through a conductor and a uniform magnetic field is pass through the same object in a way such that the direction of magnetic field is perpendicular to the current, the motor effect force will be perpendicular to both the current and the magnetic field, this can be determined using Fleming Left Hand Rule. This causes the electron to move to the side $X$ which causes the $X$ to be negatively charged, Kirchoff's 1st law states that charges must be conserved,  therefore an electron is released from face $Y$, and into the wire to continue to flow of current, this results in $Y$ to be positively charged due to electron deficiency, this sets up a electrical potential difference between $X$ and $Y$. This potential difference creates an electric field with its strength being 
$$E=\cfrac{V_H}{d}$$
with $V_H$ being the potential difference across the side $X$ and $Y$, also known as the Hall's voltage. This produces an electric force towards the face $Y$, as the charges accumulate, the potential difference increases, field strength increases, electric force increases and eventually the electric force balanced out with the force from motor effect. This gives us a relation to construct a formula for the Hall's voltage, $V_H$

\begin{align*}
    Eq&=Bq\vec{v} \\ \cfrac{V_H}{d}q&=Bq\vec{v} \\ \implies V_H&=B\vec{v}d \\ &=B\left(\cfrac{I}{NAq}\right)d \\ &= \cfrac{BI}{Nhq}
\end{align*}

where $N$ is the electron density of the conductor.

 \subsection{Magnetic Field Around A Wire}


The magnetic field around a wire is circular, and spreads out to a wider area the further away from the wire, the strength of the magnetic field is given by
$$B=\cfrac{\mu_0I}{2\pi r}$$
where $\mu_0$ is the permeability of free space ($\approx 1.26\times 10^{-6}\;m \; kg \; s^{-2}A^{-2}$), $I$ is the current in the wire that produce the magnetic field, and $r$ is the distance away from the wire.

Now let's imagine there are two wires, with length $l_1$ and $l_2$, carrying different current $I_1$ and $I_2$ respectively. Let's also say that they are parallel to each other and have a constant distance $d$ between them.  

\begin{figure}[H]
    \centering
    \includegraphics[scale=.15]{magnetic field 7 (1).jpg}
\end{figure}

To simplify the problem we can draw plane perpendicular to both of them and cutting both of them, and we only concern whats is on the plane, notice that magnetic field strength at point $B$ is 
$$B=\cfrac{\mu_0I_1}{2\pi d}$$
since both wire are parallel, therefore the entire wire $l_2$ is passing through a uniform magnetic field created by $l_1$, therefore the motor force, $F_2$ acting on $l_2$ can be found using equation (15)
$$\vec{F}_2=BIl\sin\theta=\cfrac{\mu_0I_1}{2\pi d}I_2l_2$$
using similar concept we can find the force, $F_1$ acting on $l_1$ due to the current flowing in $l_2$, which is given by 
$$\vec{F}_1=BIl\sin\theta=\cfrac{\mu_0I_2}{2\pi d}I_1l_1$$
if the both wires have the same length that are present in each other's magnetic field then we get the result 
    $$\vec{F}_1=\vec{F}_2$$
In this particular case, the forces are towards each other, we can determine that by using Fleming Left Hand Rule at $A$ and $B$, when the current are in opposite direction, then the forces will be acting away from each other which can also be determined using  Fleming Left Hand Rule.

%To determine the mass of an electron, $m_e$, we would first need to determine the charge-to-mass ratio, let the charge of an electron be $q_e$, to determine the charge-to-mass ratio we can use a deflection tube

%The heated filament will release electrons due to thermionic emission, and we set a potential difference between $A$ and $B$ with $B$ being the positive terminal, let's denote this potential difference as $V_{AB}$, this will accelerate the electron towards the magnetic field between $X$ and $Y$, to find the velocity as it enters $B$, we can use the work done to obtain an equation
%$$V_{AB}q_e=$$



\newpage
\section{Electromagnetic Induction}

 \subsection{Magnetic Flux}


Magnetic flux, denoted as $\Phi$, with unit Webar $Wb$ can be thought of as the number of perpendicular magnetic field lines passing through an area of $A$, it can be defined as 
$$B\cos \theta=\cfrac{\Phi}{A}$$
where $\theta$ is the angle between the normal of the plane and the field line passing through, this relation can be seen in the figure below 
\begin{figure}[H]
    \centering
    \includegraphics[scale=.2]{magnetic field 9.jpg}
\end{figure}
this implies that the magnetic flux on $\Lambda$ is $BA\cos \theta$.

When a coil with the area being capture by one turn is $A$, and has $N$ turns, then the magnetic flux capture through the coil is captured by each turn of the coil, this is coiled magnetic flux linkage. 
$$\Phi_{\Sigma}=N\times \Phi=NBA\sin\theta$$
where $\Phi_{\Sigma}$ is the magnetic flux linkage and $\Phi$ is the magnetic flux on the area $A$.


 \subsection{Electromagnetic Induction}


We can created induced emf or induced current by moving a conductor/wire in a magnetic field, the direction of the induced current can be determined using Fleming right hand rule.
\begin{figure}[H]
    \centering
    \captionsetup{justification=centering,margin=2cm}
    \includegraphics[scale=.25]{magnetic field 10 (1).jpg}
    \caption*{Fleming's Right Hand Rule \\ (Source: \textit{Byjus})}
\end{figure}
induced emf can be calculated using Faraday's law, then it can be used to determine
the induced current using Ohm's law.
\newpage

     \subsection{Faraday's Law of Electromagnetic Induction}


Faraday's law states that the magnitude of induced emf, $E$, is equal to the rate of change of magnetic flux or flux linkage
$$E=\cfrac{d\Phi_{\Sigma}}{dt}$$
Note that when the magnetic flux changes steadily, then the instantaneous induced emf is equal to the overall rate of change of magnetic flux.
$$E=\cfrac{d\Phi_{\Sigma}}{dt}=\cfrac{\Delta \Phi_{\Sigma}}{\Delta t}$$


 \subsection{Lenz's Law of Electromagnetic Induction}


Lenz's law states that the induced emf/current always moves in a way such that it oppose the action creating. For example, assuming a stationary coil is approached by a bar magnet to one of its end. Let's say that the north pole of the bar magnet is approaching it, then according the Lenz's law, the end that is being approach by the north pole of the magnet will be a north pole as well when the magnetic field lines of the bar magnet cuts the coil so that it oppose the magnet by repelling it. Since we know that end is a north pole, we then can use the methods in previous to determine the direction of current flowing (in previous sections we use two methods to find the pole of the ends using the direction of current, now in this case we know the pole, we then can use this knowledge to determine the direction of current flow).


\subsection{Self-Inductance or Back EMF}


In a circuit involving coils or motor, self-inductance will occur due to Lenz's law such as when the switch of the circuit is switched on or off. At the instant when the switch of a circuit containing an inductor(i.e. coil) is closed, although voltage increases instantaneously but current does not, instead when an oscilloscope with current probe is connected to the circuit, the graph of the current with respect to time will look something like this   

\begin{figure}[H]
    \centering
    \includegraphics[scale=.2]{self-inductance 1.png}
    \caption*{For the case DC circuit}
   
\end{figure}

This happens because when the switch is first closed, the current starts flows in the each turn of the coil, this generates a changing magnetic flux around it which cuts adjacent turns 

\begin{figure}[H]
    \centering
    \captionsetup{justification=centering,margin=2cm}
    \includegraphics[scale=.666]{self inductance.png}
    \caption*{Self inductance \\ (Source: \textit{Iowa State University})}
\end{figure}

According to Lenz's law, the induced current flows in such a way that resist the initial change that induces it, hence it is flowing in the opposite direction to the primary current which can be seen in the coil diagram. Looking back the graph of current against time, we can see that the gradient decreases to zero, this means that the rate of change of magnetic flux is decreasing in turn causing the magnitude of induced current to decrease. After a while, when the current reaches its maximum value, the magnetic flux becomes stationary hence no current induced, we get a line from the graph. (In the case of AC circuit, the graph of $I$ against $t$ is more or less unchanged, the sinusoidal shape is maintained.)

When the switch of the circuit if open, current also does not drop instantaneously because of the same reason that when switch is open, the changing magnetic flux induces an induced current, but this time since the negative change in magnetic flux(decreasing in flux density), therefore the current is flowing in the direction of where the primary current is flowing initially. As a result, the induced current drives the circuit for a short period of time even when the switch is open. The graph of $I$ against $t$ will have the below shape

\begin{figure}[H]
    \centering
    \includegraphics[scale=.2]{self-inductance graph 1.jpg}
\end{figure}

These are the basic information regarding inductance and back emf needed for A-levels.

\newpage
\section{Alternating Current}

The value of current can be modelled using a sine graph
$$I=I_0\sin (\omega t)$$
where $I_0$ is the peak current and $omega$ is the angular frequency.

An alternating current will also create an alternating potential difference and vice versa, therefore the voltage can also be expressed as
$$V=V_0\sin(\omega t)$$


 \subsection{Root-mean-square}


In a circuit that is running ac current, it's easy to see that the average current is not zero because there is work done on the load, hence a better way to define the average effective current is called root-mean-square current $I_{rms}$.[A circuit with rms current of $x$ can be thought of having a dc current with magnitude $x$] \textbf{Note that the \textit{mean} voltage/current of ac voltage/current is zero.}

\begin{figure}[H]
    \centering
    \includegraphics[scale=.17]{ac .jpg}
\end{figure}
\begin{figure}[H]
    \centering
    \includegraphics[scale=.17]{ac 1.jpg}
\end{figure}

it is defined as 
$$I_{rms}=\sqrt{\langle I^2\rangle}=\cfrac{I_0}{\sqrt{2}}$$

This also imples that the root-mean-square voltage, $V_{rms}$, is defined by 
$$V_{rms}=\sqrt{\langle V^2\rangle}=\cfrac{V_0}{\sqrt{2}}$$
can be derived the same way as using the graph of alternating voltage graph.

The power dissipated across a load with resistance $R$, is given by 
$$P=I_{rms}^2R=\cfrac{1}{2}I_0^2R$$
which is equivalent to half the power dissipated at the peak current. 


 \subsection{Transformer}


\begin{figure}[H]
    \centering
    \captionsetup{justification=centering,margin=2cm}
    \includegraphics[scale=.43]{electrical-transformer.png}
    \caption*{Transformer \\ (Source: \textit{Electrical 4 U})}
\end{figure}

\begin{flushleft}
\textbf{Concept of transformer}
\end{flushleft}

AC current must be use in the primary winding in order for this to work, this is because the if ac is not used, there will be no change in magnetic flux linkage in secondary winding, hence no emf will be induced, the emf induced on the secondary winding by the primary winding can be found using Faraday's Law
$$V_s=\cfrac{dN_s\Phi}{dt}$$
where $V_s$ is the output emf or the emf induced in the secondary winding, $N_s$ is the number of winding in secondary coil, $\Phi$ is the magnetic flux at the region of the secondary winding in space due to the changing current in primary winding.

This causes the induced emf/current in secondary coil to be $90^{\circ}$ out of phase with the emf/current in the primary winding, this is because since the magnetic flux at the secondary winding $\Phi$ due the current in primary winding is in phase with the current running in the primary winding, and the rate of change of the magnetic flux at secondary winding, $\cfrac{d\Phi}{dt}$, is $90^{\circ}$ out of phase with $\Phi$ and current in primary winding, the induced emf is also $90^{\circ}$ since the induced emf is in phase with $\cfrac{d\Phi}{dt}$
\begin{figure}[H]
    \centering
    \includegraphics[scale=.2]{ac 10.jpg}
\end{figure}

The output emf or output voltage is related to the input voltage by this equation 
$$\cfrac{V_p}{V_s}=\cfrac{N_p}{N_s}$$
where $V_p$ is the emf in primary winding and $N_p$ is the number of winding in primary coil and $V_s$ is the emf in secondary winding and $N_s$ is the number of winding in secondary coil.

In an ideal transformer, all the power is transmitted from primary to secondary coil through magnetic field, therefore $V_iI_i$ is the same at both primary and secondary winding. The efficiency of a transformer can be calculated using the following equation
$$\text{Efficiency}=\cfrac{V_sI_s}{V_pI_p}\times 100\%$$

A soft iron core is used in transformer to contained as much magnetic flux as possible, power in the transmission can be loss due to flux leakage, eddy current, resistance, heat loss due to magnetic hysteresis, etc.


\subsection{Design Features of Transformer}


A one piece iron core is used to reduce the magnetic flux leakage. A laminated iron core is used to reduce effect of eddy current which in turn reduces heat lose in the transformer. This is because using many layers reduces the area of loop of the eddy current by preventing electrons from crossing the insulating gap between the lamination which can be seen in the diagram below, since current of eddy current is proportional to the area, therefore this is able to reduce the magnitude of eddy current induced in the iron core due to changing magnetic flux.

\begin{figure}[H]
    \centering
    \captionsetup{justification=centering,margin=2cm}
    \includegraphics[scale=.5]{transformer 1.png}
    \caption*{Eddy current reduction \\ (Source: \textit{Wikipedia})}
\end{figure}

\begin{tcolorbox}[colframe=black!20!white,title=
\textcolor{red}{****Working principle of transformer (4m)}]
\begin{itemize}
\color{red}
    \item AC in primary coil 
    \item produces a changing magnetic flux in core
    \item changing flux links to secondary coil
    \item secondary coil experiences a change in magnetic flux linkage. According to Faraday's law, changing flux induces emf
\end{itemize}
\end{tcolorbox}

\subsection{Advantage of using AC for transmission}

\begin{enumerate}
    \item Compatible with transformer
    \item Voltage can be varied easily and effectively( typically a high voltage is used during transmission to reduce heat loss in the wire as higher voltage can deliver the same power without running a high current through the transmitting cable.)
\end{enumerate}

     \subsection{Rectification}


A diode can be used to rectify an ac current circuit because it only allows current to flow in one way. When the current is flowing in the direction that the diode permits, then it is called a forward bias, when it's flowing in the direction that the diode forbids, then no current will flow in the circuit, this is called reverse bias.

\begin{figure}[H]
    \centering
    \includegraphics[scale=.2]{ac 2.jpg}
\end{figure}

 A single diode rectifier circuit only allows current to flow in one direction and in one direction only, the graph of the current current will have the shape below 
 \begin{figure}[H]
     \centering
     \includegraphics[scale=.2]{ac 3.jpg}
 \end{figure}
When a capacitor is connected parallel to the load, 
\begin{figure}[H]
    \centering
    \includegraphics[scale=.2]{ac 4.jpg}
\end{figure}
then the current in the circuit will be 
\begin{figure}[H]
    \centering
    \includegraphics[scale=.2]{ac 5.jpg}
\end{figure}

When four diodes are used in a circuit, we can rectify the full cycle of the ac current, this is called full bridge rectifier
\begin{figure}[H]
    \centering
    \includegraphics[scale=.2]{ac 6 .jpg}
\end{figure}
the current for a full bridge rectifier looks like 
\begin{figure}[H]
    \centering
    \includegraphics[scale=.2]{ac 7.jpg}
\end{figure}

We can also add a capacitor to turn the full bridge rectifier into almost dc like circuit 
\begin{figure}[H]
    \centering
    \includegraphics[scale=.2]{ac 8.jpg}
\end{figure}
 the current in this circuit will look like 
 \begin{figure}[H]
     \centering
     \includegraphics[scale=.2]{ac 9.jpg}
 \end{figure}

\newpage
\section{Nuclear Physics}

 \subsection{Mass and Energy}


At microscopic scale, we usually use atomic mass unit, $u$, which is defined to be $\cfrac{1}{12}$ of the mass of a carbon-12 atom, to measure mass and $MeV$ to measure energy
\begin{align*}
    1u&=1.66\times 10^{-27}kg \\ 1MeV&=1.6\times10^{-19}\;J
\end{align*}

In almost every nuclear reaction, the mass of the reactant, $m_R$, is not equal to the mass of the product, $m_P$, the difference between the mass of product and the reactant is called the mass defect, $\Delta m$
$$\Delta m=m_R-m_P$$
this mass defect is related to the energy absorbed or released during the nuclear reaction by Einstein's equation
$$E=\Delta m c^2$$
where $c$ is the speed of light ($\approx 3\times 10^{8}\;ms^{-1}$). 

The binding energy of a nucleus is defined to be the energy released when it is formed from its nucleons, the value of the binding energy is equivalent to the minimum energy require to split the nucleus into individual nucleons. This implies that the higher the binding energy, the greater the amount of energy needed to separate the nucleus into individual nucleons, which means that it is more stable. If we divide the binding energy of a nucleus by the number of nucleons in the nucleus we get an average value for which represent the average energy require to remove one nucleon from the nucleus, which is also called binding energy per nucleon. It is found that the element with 56 nucleons, namely $^{56}$Fe has the highest binding energy per nucleon, which implies that iron is the most stable element which also explains why the core of a large star stops evolving when it reaches iron by nuclear fission.

\begin{figure}[H]
    \centering
    \captionsetup{justification=centering,margin=2cm}
    \includegraphics[scale=.45]{Binding-energy-per-nucleon-B-Z-N-A-as-a-function-of-the-mass-number.png}
    \caption*{Graph of binding energy per nucleon against mass number \\ (Source: \textit{ResearchGate})}
\end{figure}
Energy absorbed/released in a nuclear reaction can also be calculated using binding energy
\begin{align*}
    E&=(\text{binding energy})_{reactant}-(
    \text{binding energy})_{product} \\ 
    &=(\text{binding energy per nucleon $\times$ no. nucleon})-(\text{binding energy per nucleon $\times$ no. nucleon})
\end{align*}

\newpage

\subsection{Decay}

When a nucleus is large enough, its stability decreases to a point until it releases fragments of the content in the nucleus, this process is called radioactive decay. Nuclei decay in a spontaneous fashion, therefore we can only use the average value of the data to predict the decay pattern. 

The average probability of a nucleus to decay over a time period is called decay constant which is denoted by $\lambda$. The rate of decay is called activity, $A$, activity can be the number of decay per minute, per hour or other time length, but the number of decay per second has a more specific unit which is called becquerel, $Bq$
$$1Bq=1s^{-1}$$

The activity of a radioactive sample at a specific time $t$ can be calculated using the formula 
\begin{equation}
A(t)=\lambda N(t)=-\cfrac{dN(t)}{dt}
\end{equation}
where $N(t)$ is the number of nuclei at time $t$. The activity is equal to the negative of the rate of change of the number of undecayed nuclei is because the rate of change of the number of undecayed nuclei is decreasing because the number of undecayed nuclei left for decay decreases while the decay constant stays constant, so that in and of itself is already a negative value, since the activity is a number that represents the number for decay per unit time, we multiply the rate of change of the number of undecayed nuclei by a negative sign to make it positive (in reality we take the absolute value of the rate of change of the number of undecayed nuclei to be the activity, but since the rate of change is negative and by the definition of absolute value, we conclude that it is equivalent to multiplying a negative sign, this also makes the integral easier as we will see in the following section).

\begin{theorem}
The number of remaining atoms, $N(t)$, of a sample of radioactive element with decay constant, $\lambda$, that have not decayed after time $t$ is defined as 
$$N(t)=N_0e^{-\lambda t}$$
where $N_0$ is the initial number of atoms in the sample.
\end{theorem}
\begin{proof}
Using equation (16) we can construct a function for the activity and the number of undecayed nuclei at time $t$ with variable $t$
\begin{align*}
    \lambda N(t)&=-\cfrac{dN(t)}{dt} \\ \implies  -\lambda dt &= \cfrac{dN(t)}{N(t)} \\ \implies  -\lambda \int \limits_0^t dt&=\int \limits_{N(0)}^{N(t)}\cfrac{dN(t)}{N(t)} \\ \implies -\lambda t&= ln (N(t))\bigg|_{N_0}^{N(t)} \\ \implies -\lambda t&=ln\left(\cfrac{N(t)}{N_0}\right) \\ \implies N(t)&=N_0e^{-\lambda t}
\end{align*}
where $N_0$ is the number of undecayed nuclei at $t=0$, which is also equal to the initial number of nuclei $N(0)$.
\end{proof}

\begin{theorem}
The activity, $A(t)$, of a sample of radioactive elements with decay constant, $\lambda$, after some time $t$ is defined as 
$$A(t)=A_0e^{-\lambda t}$$
where $A_0$ is the initial activity of the sample.
\end{theorem}
\begin{proof}
Now using the definition of activity we can derive the equation for activity at time $t$ by differentiating the equation above at time $t$
\begin{align*}
    A(t)&=\left.-\cfrac{dN(t)}{dt}\right|_{t=t} \\ &= -\cfrac{d}{dt}N_0e^{-\lambda t} \\  &= -\left(-\lambda N_0e^{-\lambda t}\right) \\ &= A_0e^{-\lambda t}
\end{align*}
where $A_0$ is the activity at $t=0$, which is equal to $\lambda N_0$ and $A(0)$. 
 \end{proof}
 Half life of decay is the time taken for which the number of undecayed nuclei left is half of its initial number. The equation can be derived quite easily as follows, let's assumed that it take $t_{\frac{1}{2}}$ for a sample to decay until half of the initial nuclei remains, this give us the following equation
 $$\cfrac{N_0}{2}=N_0e^{-\lambda t_{\frac{1}{2}}}$$
 cancelling $N_0$ and rearranging the equation, we will then arrive at the desired result
 $$t_{\frac{1}{2}}=\cfrac{ln\;2}{\lambda}$$
 
 
 
 \subsection{Nuclear Fission}

 Splitting of a large unstable nucleus into 2 smaller nuclei of approximately same mass, releasing large amounts of energy
 

\subsection{Nuclear Fusion}
 
 2 small nuclei combine to form a larger nucleus, releasing a large amount of energy
 \\
 
\textcolor{red}{ \textbf{Large amount of kinetic energy is required for 2 positive nuclei to collide due to its repulsion. Thus, extremely high temperature in the core of the Sun provides sufficient energy for nuclear fusion to occur.}}

\newpage 
 \section{Quantum Physics}
 
 
 \subsection{Concept}

 
 In quantum physics, energy is quantised and can be divided into packets which can either be totally absorbed or not being absorbed at all. For now we are only concern about the energy packets of electromagnetic radiation, namely photons. A high intensity EM wave contains more photons and a high frequency EM wave has photons with higher energy.
 
  
 \subsection{Photoelectric effect}

 
 \begin{figure}[H]
     \centering
     \captionsetup{justification=centering,margin=2cm}
     \includegraphics[scale=.5]{quantum physics 1.jpg}
     \caption*{Photoelectric effect \\ (Source: \textit{CR4-GlobalSpecs})}
 \end{figure}
  The photoelectric effect can be demonstrated using the above apparatus. When a certain range of frequency of EM radiation is used, there will be a reading on the ammeter which implies that there is current flowing even though the circuit is not complete. This is because when the photon of the EM radiation has enough energy, the electrons on the surface of the metal is able to break free from the metal surface upon absorbing the photon. The energy of a photon, $E$, of a EM wave of frequency $f$  is given by 
  $$E=hf$$
where $h$ is the Planck's constant $(\approx 6.63 \times 10^{-34} \; m^2 kg \; s^{-1})$. And the minimum energy require for an electron on the surface of the metal to escape the metal is called a work function and is denoted by $\Phi$,
  $$\Phi=hf_0$$
where $f_0$ is the minimum frequency (threshold frequency) of EM wave so that the energy of its photon has sufficient energy to release an electron from the metal surface once it is absorbed.

When an electron on the metal surface absorb a photon with energy more than the energy required to escape from the metal surface, the remaining energy is converted into kinetic energy. To calculate this kinetic energy when the energy of the photon is not known can be done by just reversing the polarity of the cell in the above diagram. This makes the metal to be connected to anode and the detector to be connected to cathode, this results in a repulsion experienced by the escaped electron, when we increase the potential difference between the terminal, less and less electron reaches the detector, when there is no electron detected, we can calculate the maximum kinetic energy of the electron by equating the work done to move through the potential difference between the terminal to the loss of kinetic energy 
$$Vq_e=\cfrac{1}{2}m_ev^2$$
 where $q_e$ is the charge of an electron, $m_e$ is the mass of an electron and $V$ is the voltage where there is no electron detected (i.e stopping voltage). When the work function is also known, we can use the above method to calculate the energy of the photon absorbed or vice versa by using the following equation that can be constructed by using principal of conservation of energy
 $$E=\Phi + \cfrac{1}{2}m_ev^2$$
 
 Going back to the initial diagram, when we alter the intensity of the EM wave wave but keeping the frequency constant (frequency is at least equals to the threshold frequency), we will observe that the reading of the ammeter increases as the intensity of the EM wave used increases, this is because higher intensity implies greater number photons, which can be seen easily using the formula of intensity 
 
\begin{align*}
     I&=\cfrac{P}{A} \\
      &=\cfrac{E}{At} \\ 
      &=\cfrac{nhf}{At}
\end{align*}
 where $I$ is the intensity of EM wave,$P$ is power, $A$ is area of the power being dissipated, $n$ is the number of photons. Since frequency of light is constant, therefore increasing intensity implies increase in number of photons, if provided that the frequency is at least the threshold frequency, then as intensity increases, number of electron emitted per second will be larger hence greater current. On the other hand, if we changes the frequency of the EM wave but keeping the intensity of the EM wave invariant, as long as the frequency is more than the threshold frequency, then we would observe that the magnitude of photoelectric current decreases as frequency increases, this is because when $f$ increases $n$ must decreases in order for $I$ to be constant, when the frequency is less than the threshold frequency, we will read a constant zero.
 
 \begin{tcolorbox}[colframe=black!20!white,title=\color{red}{Evidence provided by the observation of photoelectric emission}]
 \begin{enumerate}
 \color{red}
    \item Instantaneous emission of electron
     \item Threshold frequency below which no photoelectric emission will occur
     \item Maximum kinetic energy of the electron emitted is dependant on the frequency of EM wave
     \item Maximum kinetic energy of the electron emitted is not dependant on the intensity of EM wave
     \item Rate of emission is dependant on the intensity of EM wave
 \end{enumerate}
 \end{tcolorbox}
 

 \subsection{Wave-particle Duality}

 
 Wave-particle duality relates the behaviours of wave and particles. As such, light waves can behave like particles (photons), and particles have wave characteristics.
 To display diffraction, the wavelength of radiation should be comparable to the aperture. When an electron beam is directed at a graphite film, diffraction is observed, as shown in the figure below. Since diffraction is observed, this suggests that electrons have a wavelength of about the same as the separation between atoms in crystal lattice, which is of the order of $10^{-10}$m. This wavelength possessed by a particle is called the \textbf{de Broglie wavelength,$\lambda$}.
 \begin{figure}[H]
    \centering
    \captionsetup{justification=centering,margin=2cm}
    \includegraphics[scale=1]{electron difraction.png}
    \caption*{Diffraction of electron \\ (Source: \textit{schoolphysics})}
\end{figure}
 First, assuming that both the particles and wave forms have equal energies.
 $$E=mc^2=hf$$
But since particles do not travel at the speed of light, we replace $c$ with $v$. Hence,
$$mv^2=hf$$
Since $v=f\lambda$,
$$mv^2=\cfrac{hv}{\lambda}$$
$$mv=\cfrac{h}{\lambda}$$
$$\lambda=\cfrac{h}{mv}$$
$$\lambda=\cfrac{h}{p}$$
 
 
 where $h$ is the \textbf{Planck constant} and $p$ is the \textbf{momentum of a particle}.


 
 
 \subsection{Line Spectra \& Band Theory}

 
 In the previous section, we have only discussed about what happen when the electron on the metal surface absorbs a photon, but in actuality, electron in each orbit is able to absorb photon as long as the energy level between the orbits (between any orbits that has a higher energy level than the current orbit that the electron is in and the current orbit) has an energy level difference equal to the energy of the supplied photon that is available to be absorbed. This quantum behaviour allows us to determine the composition of elements in a gas or in a star by analysing the line spectrum of emission and absorption.
 
 Line spectrum of emission can be analysed by heating the gas until photons are emitted and is passed through a diffraction grating to obtain an emission line spectrum. Transmission of heat energy can be thought of as transmission of photons, which are the quantised particle of heat. Different gaseous elements have different emission spectrum due to different energy level difference of each element.
 
 Line spectrum of absorption of a gas can be analysed by passing white light through the gas and the light is then passed through a diffraction grating and the line spectrum obtained is the desired line spectrum of absorption. Just like in line spectrum of emission, each element will have different line spectrum of absorption due to the different energy level difference of each element. 
 
 A usual line spectrum of emission and absorption will look as follow
 \begin{figure}[H]
     \centering
     \captionsetup{justification=centering,margin=2cm}
     \includegraphics[scale=.8]{quantum physics 3.jpg}
     \caption*{Absorption spectrum \\ (Source: \textit{Khan Academy})}
 \end{figure}
 
 In this example, the line spectrum of emission and absorption are of hydrogen gaseous atoms \textbf{Note that the wavelength of the line of absorption are exactly the same as the wavelength of the line of emission this is because the energy difference between the orbit determines the energy of photons that can be absorbed or emitted, since in both cases hydrogen atoms are used, photons that carry a specific amount of energy equivalent to the energy difference of the orbits are absorbed or emitted \textit{(only equivalent energy of photon is absorbed because of the fundamental theorem of quantum physics, unless the energy of the photon is more than the ionisation energy of the element, then the leftover energy will be converted into KE similar to photoelectric effect.)}, hence only this specific frequency of EM wave that carries photons with that particular energy can be absorbed and emitted, as a result, specific wavelength of EM wave that has the particular frequency are missing or present as discrete lines in the line spectra.}
 
 To know the wavelength of EM wave that can be absorbed or emitted, an energy level diagram is needed, an energy of an isolated atom will look something like this 
 \begin{figure}[H]
     \centering
     \includegraphics[scale=.2]{quantum physics 4 (1).jpg}
 \end{figure}
in gaseous atoms the energy level diagram will also look very similar because gaseous atoms are thought to be ideal gases to simplify things so discrete lines of energy are formed similar to isolated items. The photon that carries energy of any combination of $\Delta E_i$ and photon with energy \textbf{more than or equal to} $\sum \Delta E_i$ can be absorbed  but only photons that carry energy of any combination $\Delta E_i$ or \textbf{exactly equal to} $\sum E_i$ can be emitted.
 
 In solid the situation is much more complicated, because the atoms are closely packed together, interactions between them are drastic compare to in gaseous state, as a result each energy level of each atom are disturbed causing many energy  levels with slight variations to occur forming blocks that contain large number of energy levels, this creates a band in the energy level diagram instead of discrete lines of energy level that the electrons can stay in.
 \begin{figure}[H]
     \centering
     \captionsetup{justification=centering,margin=2cm}
     \includegraphics{quantum physics 5.jpg}
     \caption*{(Source: juilee. \textit{Ques 10})}
 \end{figure}
Energy band diagram can be used to explain why most metals have shiny and reflective surface.\textcolor{blue}{ \textbf{Explanation for why metal is shiny is beyond. A-level syllabus.} Let's denote the energy range of $i$th energy band be $\Delta E_i,\; i\in\{1,2,\dots,n\}$ where $\Delta E_1$ is the energy range of the ground band and $\Delta E_n$ is the conduction band, and also denote the minimum and maximum energy of the $i$th band be $E_i$ and ${E_i}'$ respectively, $\Delta G_i$ be the range of energy of the forbidden gap between the $i$th band and $(i+1)$th band, which implies $\Delta G_i= E_{i+1}-{E_i}'$, then the energy of photons, $E_p$, that can be absorbed are in the sets $E_p \in \{ A \cup B \cup  C \cup D\} $}

\textcolor{blue}{Set $A$ is denoted by $a\in A$ for $a$ satisfying the inequality 
$$a \geq \Phi$$
where $\Phi$ is the work function, this set contains all the energy of photons that has enough energy to emit electrons from the conduction band and if $ \exists \; a_x \in A$ such that $a_x \geq \Delta G_i,\; i\in \{1,2,\dots,n-1\}$, aka $\exists \; a_x\in \{A \cap C\}$ then photon with $E_p=a_x$ will cause both excitation across forbidden gaps and electron emission from conduction to occur. If  $\exists \; a_y\in A $ such that  $0\leq a_y \leq {E_i}'-E_i,\; i\in \{1,2,\dots,n\}$, aka $\exists \; a_y\in\{A \cap B\}$, once photon with $E_p=a_y$ is absorbed then both emission of electron from conduction band and excitation of electron to have a higher energy level while still being in the same band will occur.}

\textcolor{blue}{Set $B$ is denoted by $b\in B$ for $b$ satisfying
$$0\leq b\leq {E_i}'-E_i \quad i\in \{1,2,\dots,n\}$$
this set contains all the possible energy of photons that can be absorbed or emitted due to excitation of electron to the same band unless $\exists \; b_x\in B$ such that $b_x\geq \Delta G_i, \; i\in \{1,2,\dots, n-1\}$, aka $\exists \; b_x\in \{B\cap C\}$ then photon with $E_p=b_x$ will cause both excitation of electron to have a higher energy level while still being in the same band and excitation across forbidden gap to occur and if $\exists \; b_z\in B $ such that  $b_z \geq \Phi$, aka $\exists \; b_z\in \{B\cap A\}$ then excitation of electron to have energy in the same band and emission of electron from the conduction band will also occur.}

\textcolor{blue}{Set $C$ is defined as $c\in C$ for $c$ satisfying 
$$E_k-{E_j}' \leq c \leq {E_k}'-E_j \quad j,k\in \{1,2,\dots,n\} \; k\neq 1, \; j \neq n, \; j < k$$
for all combinations of $j$ and $k$. This set contains all the energy of photons that can be absorbed or emitted due to excitation of electron from one band to another.If $\exists \; c_z \in C$ such that $c_z\geq \Phi$, aka $\exists \; c_z\in \{C\cap A\}$, then absorption of photon with $E_p=c_z$ will cause both excitation of electron to jump across forbidden gap and electron emission from conduction band to occur. Note that $\{c \geq \Delta G_i ,\; i\in \{1,2,\dots,n-1\} \;|\; \forall c\in C \}$ }

\textcolor{blue}{Set $D$ is defined as $d\in D$ for $d$ satisfying
$$d \geq \Phi + {E_n}'-E_1$$
this set contains all the photons that once absorbed only emission of electron will occur. Note that $A\subseteq D$.}

\textcolor{blue}{When photons with $E_p\in \{B \cup C\}$ which also includes $\{a_x, a_y, b_x\}$, electrons on the metal will be excited, and eventually the excited electron will go to its ground state by emission of photons, since these sets contain a lot of ranges of energy of photons that can be absorbed, this also means that a number of photons with different energy and wavelength can be emitted, hence the shiny surface. The colour of the metal is also determined by the range these sets. Hence due to many intersection of sets the possibility is endless and contain wide ranges of wavelength that can be absorbed, hence a shiny and reflective surface.}

Energy band diagram can also be used to explain the differences in electrical conductivity between metallic conductor, semiconductor and insulator.
\begin{figure}[H]
    \centering
    \captionsetup{justification=centering,margin=2cm}
    \includegraphics[scale=.5]{quantum physics 6.jpg}
    \caption*{Band theory \\ (Source: \textit{Lumen Learning})}
\end{figure}
In metal, the conduction band and valence band are usually overlapped, this allows the electrons in valence band to readily move to conduction which makes metallic conductor highly conductive. On the other hand, the forbidden gap between the valence band and the highest energy band below valence band is large in a metallic conductor, as a result when metallic conductor is heated, almost no electron can jump from low energy level band to valence band which is merged with conduction band, hence conductivity will not increase since the number of charge carriers are constant in the conduction-valence band, moreover, the atoms vibrate more violently at high temperature which in turn decreases the rate of flow of free electron, as a result increase in resistivity  of the metallic conductor.

In semiconductor, the forbidden gap between the valence band and conduction band has low energy difference, this gives the semiconductor its ability to conduct electricity more efficiently when heated because when heat is supplied, electron in valence band accept phonon and leap across the forbidden gap easily, this increase the number of electron in conduction band and the number of holes in valence band, as a result increase in conductivity and decrease in overall resistivity.

In an insulator, the forbidden gap between conduction and valence band has a large energy difference, and there is no electron in the conduction band to start with, hence electrons aren't able to jump across the forbidden gap from valence band to conduction band, therefore it is an insulator.
 
 
 \begin{tcolorbox}[colframe=black!20!white,title=\color{red}{****Explain why as temperature increases, the electrical conductivity of a semiconductor increases/ resistance of a semiconductor decreases. (5-6m)}]
\begin{itemize}
\color{red}
    \item As temperature increases, electrons gain energy
    \item Electrons gain enough energy and leap across forbidden gap from valence band to conduction band
    \item holes are left in valence band and number of electrons in conduction band increases
    \item more charge carriers hence resistance decreases
    \item although as temperature increases, the lattice vibrations increases
    \item but the effect of increase in electrons and holes outweighs the effect of increased lattice vibrations
\end{itemize}
\end{tcolorbox}
 
 \begin{tcolorbox}[colframe=black!20!white,title=\color{red}{****Explain why as temperature increases, the electrical conductivity of a conductor(metal) decreases/ resistance of a conductors(metal) increases. (5-7m)}]
\begin{itemize}
\color{red}
    \item The conduction band is partially filled because it is overlapped with valence band
    \item As temperature increases, electrons gain energy
    \item Electrons gain enough energy and move to conduction band from valence band
    \item holes are left in valence band and number of electrons in conduction band increase, hence number of charge carriers increases
    \item but the effect of increased in charge carriers is insignificant because of the overlapping between conduction band and valence band
    \item On the other hand, as temperature increases, the lattice vibrations increases
    \item The effect of increased lattice vibrations outweighs the effect of increased number of charge carriers, as a result, resistance increases.
\end{itemize}
\end{tcolorbox}
 
 
 
 \newpage
 \section{Medical Physics}

 
\subsection{Nuclear Magnetic Resonance}


Many atomic nuclei with odd number of protons and/or neutrons have a property called the 'spin', making them behave like small magnets. When a magnetic field is applied, they tend to line up along the field. However, due to their 'spin', the alignment is not perfect and they rotate about the direction of a gravitational field. Rotation can be parallel or anti-parallel to external field.  This rotation is called \textbf{precession}. The frequency of precession is called the \textbf{Larmor frequency}, which is in the radio-frequency(RF) band,$\lambda$ $\leq$10cm 
$$f_L=4.25\times 10^7 B_T$$
It depends on the nature of the nucleus and the magnitude of the magnetic flux density. 
If EM radiation of the same frequency as the Larmor frequency is incident on the precessing nucleus, it resonates and absorbs energy. When parallel precessing protons absorb RF photon, it is promoted to higher energy state-anti-parallel rotation. After a short time, the nuclei will return to its equilibrium state, emitting RF radiation. The short time between the end of RF pulse and re-emitting of radiation is known as \textbf{relaxation time}. The whole process is called \textbf{nuclear magnetic resonance}.

\begin{flushleft}
 \textbf{Magnetic Resonance Imaging (MRI)}
\end{flushleft}

\begin{figure}[H]
    \centering
    \includegraphics[scale=.5]{MRI.PNG}
    \caption*{(Source: Mike Crundell, et al. \textit{Cambridge International AS and A level 2nd edition.} Hodder education, 2014)}
\end{figure}

Between 2 poles a large magnet produces a very \textbf{large magnetic field} ($>$ 1T). This causes all the hydrogen nuclei within the person to precess with the same Larmor frequency.
In order that the hydrogen frequency of only one small part of the body to be detected, a \textbf{non-uniform magnetic field} is applied. The non-uniform field is accurately calibrated, resulting in a different magnitude of magnetic flux density and Larmor frequencies at each point of the body. RF pulses are produced in coils near the patient. After the external RF pulse is switched off, the emitted pulses due to de-excitation of hydrogen nuclei are picked up by the coil to construct an image of the number density of hydrogen atoms in the patient. 
\\The intensity of emitted RF provides information about the proton density or water content of the tissue. 
\begin{enumerate}
    \item High water(proton/H) content - Swelling, infection, inflammation, bleeding cyst
    \item Low water content - Tissues lacking hydrogen, replace by deposit of calcium (calcification), leads to stones in kidney/ liver, fibrosis, scar formation
    \item Extreme proton density - Tumour, leading to fast rate of producing cells, ie. protein/hydrogen
\end{enumerate}
Relaxation time (rate of release of RF) provides information about the tissue surrounding the protons
\begin{enumerate}
    \item Several seconds - watery tissue
    \item Intermediate - cancerous tissue
    \item Several ms - fatty tissues
\end{enumerate}

\begin{tcolorbox}[colframe=black!20!white,title=\color{red}{****Explain the principles of MRI. (8m)}]
\begin{itemize}
\color{red}
    \item A strong uniform magnetic field is applied across the body
    \item Nuclei of hydrogen in the body is aligned at precess about the strong magnetic field
    \item To obtain information about specific parts of the body, a second non-uniform magnetic field is applied across the body
    \item The fact that Larmour frequency of the hydrogen nuclei/protons is dependant on the magnetic field strength
    \item RF pulse with frequency equal to the Larmour frequency to that specific part is emitted
    \item Only the hydrogen nuclei/protons at that part undergo resonance by absorbing the energy of RF pulse emitted
    \item During de-excitation, these nuclei emits RF pulse 
    \item RF pulse emitted by de-excitation is detected and processed

\end{itemize}
\end{tcolorbox}

\subsection{X-rays}

\begin{flushleft}
\textbf{Production of X-rays}
\end{flushleft}

The production of X-ray is based on the concept of Bremmstrahlung radiation, which is also known as 'braking radiation'. Bremmstrahlung radiation is produced when a charged particle undergoes deceleration, converting kinetic energy into EM radiation. The greater the magnitude of deceleration, the higher the frequency of emitted radiation.

\begin{figure}[H]
    \centering
    \includegraphics[scale=.7]{X-ray 1.jpg}
    \caption*{Production of Bremmstrahlung radiation. \textit{Radiology Key}}
\end{figure}
    

To produce X-ray photons, high speed electrons are bombarded on metal targets. An electron beam is produced through thermionic emission, i.e. emission of electrons from a heated metal filament using a low-voltage supply. The electrons are first accelerated through a potential difference of many \textbf{kVs} so that they have high energy and speed. When the high speed electrons strike on a metal target, it changes direction and thus lose kinetic energy very rapidly, as shown in the figure above. Large decelerations in turn give rise to production of X-ray photons. \textbf{However, not all kinetic energy is emitted as X-ray, majority of the energy is converted into thermal energy}.

\begin{figure}[H]
    \centering
    \captionsetup{justification=centering,margin=2cm}
    \includegraphics[scale=.5]{xray2.PNG}
    \caption*{X-ray Tube \\ (Source: Mike Crundell, et al. \textit{Cambridge International AS and A level 2nd edition.} Hodder education, 2014)}
\end{figure}

As shown above in the simplified design of X-ray tube, there are a few characteristics that enable a more effective production of X-rays.
\begin{enumerate}
    \item Metal filament (cathode) - Electrically heated. The rate of thermionic emission is controlled by the temperature of metal filament, which is dependant on the magnitude of current.
    \item Large variable D.C voltage (10kV-90kV) - Electrons are accelerated through a potential difference and gain kinetic energy, $KE=qV$.
    \item Evacuated (vacuum) tube - To prevent the loss of kinetic energy due to collision of electrons with air particles.
    \item Tube is made of material opaque to X-rays - Reduces background radiation around the tube.
    \item Since a lot of thermal energy is produced, metal target - \begin{enumerate}
        \item has high melting point
        \item is embedded into a copper block that spins rapidly so that thermal energy is dispersed over a larger area of the target. Heat is conducted away from the metal target to prevent melting of the metal target.
    \end{enumerate}
    \item Lead grid - X-ray photons pass through a window where divergent rays are collimated by a lead grid into a fine beam. This will be used later in X-ray imaging.
\end{enumerate}

The kinetic energy of the electron $K_e$ is equals to the energy gained by the electron when it is accelerated in the potential difference. Hence,
$$K_e=eV$$
where $e$ is the charge of the electron and $V$ is the accelerating potential difference.
The energy of a photon of X-ray produced is expressed by the equation
$$E_i-E_f=hf$$
where $E_i$ is the initial $K_e$ of electron and $E_f$ is the $K_e$ of electron after deflection \textit{(loss of energy due to deceleration and heat)}. To simplify, we express $E_i-E_f$ as $E$, where $E$ is the energy released as an X-ray photon.
$$E=\cfrac{hc}{\lambda}$$
where $\lambda$ is the wavelength of the photon produced, which is variable, as each electron undergoes different magnitudes of deceleration.
Thus the maximum energy of a photon is given by 
$$eV=\cfrac{hc}{\lambda_o}$$
$$\lambda_o=\cfrac{hc}{eV}$$
where $\lambda_o$ is the cut-off wavelength (the minimum wavelength/maximum frequency produced by a certain accelerating potential difference, when an electron is stopped by a single collision and all the kinetic energy is converted into EM radiation). The larger the potential difference, the greater the $E$, the shorter the wavelength.\textit{($h$ and $c$ are constants)}  
As such, we can see that the wavelengths of X-ray produced is not constant/discrete, instead it is a continuous spectrum.

\begin{figure}[H]
    \centering
    \captionsetup{justification=centering,margin=2cm}
    \includegraphics[scale=.5]{xray-spectrum.jpg}
    \caption*{X-ray Spectrum \\ (Source: \textit{ReseachGate})}
\end{figure}

From the X-ray spectrum, we observe 
\begin{enumerate}
    \item minimum wavelength produced, (cut-off wavelength)
    \item a continuous spectrum (Bremmstrahlung spectrum)
    \item sharp peaks (line spectrum), called the characteristic X-rays of the target
\end{enumerate} 

Characteristic X-rays occur when a bound electron is ejected from the inner shell of the atom after being bombarded by an electron. Consequently, the atom is left with a vacant energy level, also known as a \textbf{core hole}. Electrons from outer shells will fall into inner shells (move from higher energy level to lower energy level), since particles tend to occupy its most stable state.) This transition produces a photon with frequency corresponding to the difference in energy levels. Since each element has a unique set of energy levels, these line spectrum are unique to each metal and can enable us to identify the metal, thus being known as characteristic X-rays.

An X-ray beam can be controlled in terms of 
\begin{enumerate}
    \item \textbf{intensity}. The \textbf{intensity} is \textbf{wave power per unit area}, and this affects the blackness of image produced. Intensity is dependant on the rate of emission of photons, i.e. the number of electrons hitting on the target metal per unit time. This can be controlled by manipulating the rate of thermionic emission, which is affected by the current in the filament, which in turn controls the temperature of metal filament.   
    \item  \textbf{hardness}. The \textbf{hardness} is \textbf{the penetration of the X-ray beam}, which determines its degree of penetration. The hardness can be controlled by changing the \textbf{accelerating voltage} to control the $K_e$ of electrons. The higher the frequency, the shorter the wavelength, the higher its energy, the greater its penetrating power through body tissues. Weak X-rays are easily absorbed by body tissues since it cannot penetrate body tissues to form an image on the screen. These 'soft' X-rays are harmful to the body as it increases exposure to EM radiation without any effective formation of image. Thus, X-ray beams are usually passed through aluminium filters to absorb long-wavelength photons.
\end{enumerate}

\begin{tcolorbox}[colframe=black!20!white,title=\color{red}{****Explain the principles of production of X-ray beam. (5-6m)}]
\begin{itemize}
\color{red}
    \item Current is passed through a metal filament
    \item Electron is emitted through thermionic emission
    \item Electrons are accelerated through a large potential difference and fired at a metal target
    \item The deceleration of electron due to collision produces X-ray photons
    \item Range of deceleration produces a continuous spectrum 
    \item Electrons in inner orbits are also excited 
    \item De-excitation of electrons gives characteristic line spectrum
\end{itemize}
\end{tcolorbox}

\begin{flushleft}
\textbf{X-ray Imaging}
\end{flushleft}

To produce an X-ray image, a collimated X-ray beam incident on the patient. The core concept is that the X-ray passes through tissues/bones of different densities and thickness, and is attenuated to different extents. Hence, the image produced is a shadow with different intensities of black, where the darker parts are where higher intensities of X-ray fall on the screen. A two-dimensional shadow of the bones and surrounding tissues are produced on the screen.

\begin{figure}[H]
    \centering
    \captionsetup{justification=centering,margin=2cm}
    \includegraphics[scale=.55]{xray3.PNG}
    \caption*{(Source: Mike Crundell, et al. \textit{Cambridge International AS and A level 2nd edition.} Hodder education, 2014)}
\end{figure}

The quality of image produced is affected by the \textbf{sharpness} and \textbf{ contrast}. 
\begin{enumerate}
    \item \textbf{Sharpness}
    \\To acquire a sharp image,the edges of the structures should be easily determined. This requires parallel a X-ray beam. This can be achieved by 
    \begin{enumerate}
        \item Reducing partial shadow- Between white and black areas on regions on an X-ray image, there is a region of partial shadow, which is an area of greyness. To obtain a sharp image, the partial shadow should be minimised so that there is a sharp outline. 
        To do this, we can 
        \begin{enumerate}
            \item reduce the area of target anode - The effect of a smaller target anode is illustrated in the following figure.
            \begin{figure}[H]
                \centering
                \captionsetup{justification=centering,margin=2cm}
                \includegraphics[scale=.5]{shadow.PNG}
                \caption*{(Source: Mike Crundell, et al. \textit{Cambridge International AS and A level 2nd edition.} Hodder education, 2014)}
            \end{figure} 
            \newpage
            \item limit the size of the aperture through which the X-ray beam passes - by using overlapping metal sheets, the aperture size can be varied by sliding the metal sheets over one another
            \begin{figure}[H]
                \centering
                \captionsetup{justification=centering,margin=2cm}
                \includegraphics[scale=.7]{aperture.PNG}
                \caption*{Limiting the size of aperture \\ (Source: Mike Crundell, et al. \textit{Cambridge International AS and A level 2nd edition.} Hodder education, 2014)}
            \end{figure}
        \end{enumerate}

        \item reducing the scattering of emergent beam - scattered photons are absorbed in a lead grid placed in front of the X-ray film
        \begin{figure}[H]
            \centering
            \captionsetup{justification=centering,margin=2cm}
            \includegraphics[scale=.7]{lead grid.PNG}
            \caption*{Use of lead grid to reduce scattering of photons \\ (Source: Mike Crundell, et al. \textit{Cambridge International AS and A level 2nd edition.} Hodder education, 2014)}
        \end{figure}
    \end{enumerate} 
    \item \textbf{Contrast}
    \\An X-ray having a wide range of degrees of blackening in different region. This can be achieved when neighbouring organs and tissues absorb X-ray photons to different extents. There are a few ways to achieve this. For instance, 
        \begin{enumerate}
            \item the patient is asked to swallow a \textbf{barium meal} (barium sulfate) as barium is a good absorber of X-ray photons
            \item blood vessels can be made visible by injecting a radio-opaque dye into the blood stream
            \item reasonable exposure time
            \item backing the X-ray film with a fluorescent material
        \end{enumerate}
\end{enumerate}

\begin{flushleft}
\textbf{Attenuation of X-rays}
\end{flushleft}
When a beam of X-rays photons passes through a medium, its intensity is reduced as it is absorbed. The intensity is reduced by the same fraction when it passes through the same medium for the same thickness. 
\begin{figure}[H]
    \centering
    \captionsetup{justification=centering,margin=2cm}
    \includegraphics[scale=.5]{attenuation.jpg}
    \caption*{Graph of intensity against thickness \\ (Source: \textit{Cyberphysics})}
\end{figure}
From the graph above, we can see that a constant thickness of a medium is needed to reduce the intensity by half. This thickness is called the \textbf{half-value thickness, $x_{1/2}$}. \textit{(similar to the concept of half-life)} 
The transmitted intensity is given by
$$I=I_oe^{-\mu x}$$
where $\mu$ is the \textbf{linear attenuation coefficient/linear absorption coefficient}. Its unit is $mm^{-1}/cm^{-1}$.
For $x_{1/2}$, 
$$\cfrac{1}{2}I_o=I_oe^{-\mu x_{1/2}}$$
$$ln\;2=\mu x_{1/2}$$
\textit{Note that this only applies to a parallel beam. If the beam is not parallel, there will be changes to the intensity even if it's not absorbed such as due to the scattering of X-ray photons by atoms.}

\begin{tcolorbox}[colframe=black!20!white,title=\color{red}{****Explain the principles of X-ray imaging. (4m)}]
\begin{itemize}
\color{red}
        \item X-ray beam directed through body onto detector 
        \item different tissues attenuate the X-ray beam by different amount
        \item `shadow' image of structure is produce when the X-ray beam passed through 
        \item degree of blackening or contrast gives information about the tissue
\end{itemize}
\end{tcolorbox}


\subsection{Computed Tomography (CT scanning)/CAT scan (Computerised Axial Tomography)}

The production of image discussed in the earlier part is only 2-D (cannot show depth). Therefore, soft tissues behind very dense structures cannot e detected. To obtain a 3-D image, tomography is used. \textbf{Tomography} is used in a \textbf{CT scanner}, where a series of X-ray images of the 'slice' of the body is obtained from different angles. Data for each image of each angle is fed into a high-power computer to be combined and produce a 3-D image. 
Each 'slice' is divided into many small units, called \textbf{voxels} that has different \textbf{attenuation numbers/pixels}. Attenuation numbers indicate the percentage of X-ray absorbed.
The basic principles of CT scanning can be illustrated by the simple cross-sectional cube below.

\begin{center}
    \begin{tabular}{|c|c|}
    \hline
    a    &  b\\
    \hline
    c    & d\\
    \hline
    \end{tabular}
\end{center}

When the X-ray is incident from the left, the readings shown would be 
\begin{center}
    \begin{tabular}{|c|c|}
    \hline
    a+b & a+b\\
    \hline
    c+d & c+d\\
    \hline
    \end{tabular}
\end{center}

 Next, when the X-ray is incident $45^o$ northwest, the readings are added as
 \begin{center}
    \begin{tabular}{|c|c|}
    \hline
    2a+b+d    &  a+2b\\
    \hline
    2c+d    & a+c+2d\\
    \hline
    \end{tabular}
\end{center}

Adding readings for when X-rays fall from the top,
 \begin{center}
    \begin{tabular}{|c|c|}
    \hline
    3a+b+c+d    &  a+3b+d\\
    \hline
    a+3c+d    & a+b+c+3d\\
    \hline
    \end{tabular}
\end{center}

Lastly, adding readings from $45^o$ northeast,
 \begin{center}
    \begin{tabular}{|c|c|}
    \hline
    4a+b+c+d    &  a+4b+c+d\\
    \hline
    a+b+4c+d    & a+b+c+4d\\
    \hline
    \end{tabular}
\end{center}

As such, the pixel for each voxel can be determined and the corresponding blackness is displayed on the image. For this case of 4 voxels, 4 steps are needed to determine the image of the 'slice'. The number of steps (angles) is dependant on the number of voxels present. Note that there is a constant that has to be removed from each voxel, which is 'a+b+c+d'. We call this the \textbf{background intensity}. Then, we will be left with repeated measurements of each voxels. In this case, we need to divide the reading by 3 after deducting the background intensity.
\\To summarise, to get the magnitude of pixels in each voxels, we need to
\begin{enumerate}
    \item Remove background intensity, i.e. sum of the readings in any step
    \item Divide by (number of steps/angles-1)
\end{enumerate}

In reality, the number of voxels are very large. The greater the number of voxels, the better the definition. The computer enables the brightness and contrast of the image to be varied so that the optimum image is produced.
\newline
\begin{tcolorbox}[colframe=black!20!white,title=
\textcolor{red}{****Explain the principles of CT scanning. (6m)}]
\begin{itemize}
\color{red}
    \item A series of X-ray images of each slice are taken
    \item repeated at different angles
    \item the images are processed
    \item and combined to produce a 2-D image of each slice
    \item This is repeated for successive slices
    \item to build up a 3-D image
    \item the image can be rotated/viewed from different angles
\end{itemize}
\end{tcolorbox}

\subsection{Ultrasound}

\begin{flushleft}
\textbf{Generation of Ultrasound by Reverse Piezo-electric Effect}
\end{flushleft}

The generation of ultrasound is based on the concept of reverse piezo-electric effect. A \textbf{piezo-electric transducer} converts electrical energy into ultrsound energy by using a piezo-electric crystal such as \textbf{quartz}. Quartz is made up of many silicon dioxide, $SiO_2$, in a giant crystal lattice. Since Si-O bonds are polar, we have Si atoms with $\delta +$ charge and O atoms with $\delta -$ charge. At equilibrium, the bond dipoles cancel each other out, thus it is neutral on the whole, i.e. the centres of charges of positive and negative charges coincide at a point, as shown below.

\begin{figure}[H]
    \centering
    \includegraphics[scale=.2]{piezo-electric.jpg}
    \caption*{At equilibrium}
    \label{fig:my_label}
\end{figure}

where the red points represent silicon, $\delta +$ charged, and blue points represent oxygen, $\delta -$ charged. However, when a potential difference is applied across the atoms, the atoms of quartz will be deformed (either compressed or extended depending on how the potential difference is applied). 
\\For instance, when a potential difference from top to bottom, we observe that the centre of negative charge has shifted upwards and centre of positive charge moves downwards. Thus the bottom negative charge shifts upwards slightly and the 2 other vertex of the blue triangle merely moves sideways. In this case, the tetrahedral repeat unit is compressed. This happens throughout the whole crystal. 

\begin{figure}[H]
    \centering
    \includegraphics[scale=.2]{piezo-electric compressed.jpg}
    \caption*{Compressed}
    \label{fig:my_label}
\end{figure}

On the contrary, when a potential difference is applied from bottom to top, the centre of positive charge moves upwards and centre of negative charge moves downwards, as shown in the figure below. In this case the repeat unit is stretched. 

\begin{figure}[H]
    \centering
    \includegraphics[scale=.2]{piezo-electric Stretched 1.jpg}
    \caption*{Stretched}
    \label{fig:my_label}
\end{figure}

An alternating voltage is is applied across the electrodes, causing the crystal to vibrate with the same frequency as the alternating voltage. When the frequency of vibration is equals to the natural frequency of the crystal, resonance occurs and the atoms vibrate with maximum amplitude. The dimensions of crystals are made such that its oscillations are in the range of ultrasound ($>$20kHz). These oscillations produce ultrasound waves.

\begin{flushleft}
\textbf{Piezo-electric Effect}
\end{flushleft}

Conversely to production of ultrasound, the receiving of ultrasound is based on the concept of piezo-electric effect. When stress is applied on the crystal, the position of negative and positive charges in the crystal shifts and thus produces a potential difference across the crystal. (Refer above for mechanism). When ultrasound is incident on the receiver/transducer, variations in pressure will produce voltage variations.

\begin{figure}[H]
    \centering
    \captionsetup{justification=centering,margin=2cm}
    \includegraphics[scale=.5]{transducer.PNG}
    \caption*{Piezo-elecctric transducer/receiver \\(Source: Mike Crundell, et al. \textit{Cambridge International AS and A level 2nd edition.} Hodder education, 2014)}
    \label{fig:my_label}
\end{figure}

When incident on a boundary between 2 waves, ultrasound can either be reflected or transmitted. Hence, the intensities are related by:
$$I=I_R+I_T$$
where $I_R$ is the intensity of the reflected wave and $I_T$ is the intensity of transmitted wave. Not only are the ratio of both intensities dependant on the angle of incidence, but also the medias in which the wave is travelling in. The relative magnitudes of $I_R$ and $I_T$ can be quantified by using the specific acoustic impedance,$Z$ of each media. $Z$ is defined as the product of the density $p$ of the medium and the speed $c$ of the wave in the medium. 
$$Z=pc$$

For a wave incident \textbf{normally} on a boundary between 2 medias, the ratio of $I_R$ to $I$ is given by,
$$\cfrac{I_R}{I}=\cfrac{(Z_2-Z_1)^2}{(Z_2+Z_1)^2}$$
This ratio, $I_R/I$ is called \textbf{intensity reflection coefficient}, with symbol $\alpha$. The smaller the $\alpha$, the smaller the intensity if reflected wave, the greater the transmission between 2 media. Thus, to achieve a greater transmission, the difference in $Z$ for both media has to be small. To overcome the large difference in $Z$ between air and soft tissues, we need to make sure there is no air between the transducer/receiver and the skin, therefore we apply a \textbf{coupling medium }(water-based jelly) on the skin during ultrasonic scans.
\\The intensity of transmitted wave will then decrease due to absorption, in other words, \textbf{attenuation}. For a parallel beam, the absorption is approximately exponential. (Refer to X-ray.)
$$I=I_oe^{-kx}$$
where k is the \textbf{linear absorption(attenuation) coefficient} of the medium. \textit{k also depends on the frequency of ultrasound.} \textbf{Note that this expression only applies for a parallel beam.}

\begin{flushleft}
\textbf{Ultrasound Scan}
\end{flushleft}

Higher frequency ultrasound is used for detecting small structures, but at higher frequency the attenuation is larger and lower penetrating power this is because a high frequency sound has many cycles in a second, and the particles in the medium are therefore vibrating very rapidly. Just as when you rub your hands together very rapidly, this produces more heat than if you rub your hands together slowly. Since the molecules get their energy to vibrate from the sound wave, the sound wave will run out of energy sooner when it is a high frequency sound. This means that, under the same conditions, a high frequency sound won’t travel as far as a low frequency sound(which cause absorption coefficient to change for different frequencies). There are 2 types of ultrasound scans:
\begin{enumerate}
    \item \textbf{A-scan}\\
    The transducer emits pulses of ultrasound, which is transmitted into the body through a coupling medium.The pulses are partly reflected and partly transmitted at the boundaries between media in the body. The transducer now acts as a receiver to detect the reflected wave. The signal is amplified, converted into an alternating voltage and displayed on a c.r.o. The reflected wave from deeper in the body is amplified more as reflection and attenuation happens on its reflected path as well. The \textbf{time intervals} between the emitted wave and reflected wave indicates the size of the organs. The \textbf{intensity} of wave indicates the nature of the boundaries. The greater the intensity, the higher the density of the surface.
    \begin{figure}[H]
        \centering
        \captionsetup{justification=centering,margin=2cm}
        \includegraphics[scale=.5]{a scan.png}
        \caption*{A-scan \\ (Source: \textit{Cyberphysics})}
        \label{fig:my_label}
    \end{figure}
    \newpage
    \item \textbf{B-scan}\\
    A B-scan consists of many A-scans taken at different angles. Therefore, instead of a single crystals, it consists of an array of crystals arranged at slight angles from each other. A B-scan produces a 2-D image by converting the reflected pulse along one section of the organ into bright dots (pixels). The positions of dots shows the curvature of the surface of the organ.
    \begin{figure}[H]
        \centering
        \captionsetup{justification=centering,margin=2cm}
        \includegraphics[scale=.3]{b scan.png}
        \caption*{B-scan \\ (Source: \textit{ResearchGate)}}
        \label{fig:my_label}
    \end{figure}
\end{enumerate}
Detection of tumour is shown by extra bright dots with higher intensities, since tumours have higher densities. Higher frequencies will produce greater resolution as diffraction around small features is reduced.\\
Advantages of ultrasound scanning:
\begin{enumerate}
    \item Lower health risk compared to X-ray (lower frequency)
    \item More portable
    \item More simple to use
\end{enumerate}

\begin{tcolorbox}[colframe=black!20!white,title=
\textcolor{red}{****Explain the principles of ultrasound. (6m)}]
\begin{itemize}
\color{red}
    \item Pulse of ultrasound is produced by piezo-electric crystal when an ac voltage with frequency in the range of ultrasound is applied
    \item The crystal vibrates at resonant frequency 
    \item Coupling medium/gel is used to increase transmission of ultrasound 
    \item Pulse of ultrasound is reflected by the boundary between media
    \item The reflected pulse is detected
    \item Signal processed and displayed
    \item Intensity of pulse reflected gives information about the boundary
    \item Time delay between reflected pulses tells us about depth

\end{itemize}
\end{tcolorbox}
\newpage 
\fancyhead[C]{CONSTANTS}
\section*{Constants}

\begin{align*}
        \mbox{speed of light in free space} \quad & c=3\times 10^{8}\; ms^{-1} \\ \\
        \mbox{permeability of free space} \quad & \mu_0=4\pi\times {10}^{-7} \;Hm^{-1} \\ \\
        \mbox{permittivity of free space}\quad & \varepsilon_0=8.85\times 10^{-12}\;Fm^{-1} \\ \\
            & \left(\cfrac{1}{4\pi\varepsilon_0}=8.99\times 10^{9}\; mF^{-1}\right)\\\\
        \mbox{elementary charge} \quad & e=1.60\times 10^{-19}\; C \\ \\
        \mbox{Planck constant} \quad & h=6.63\times 10^{-34}\; Js \\ \\
        \mbox{unified atomic mass unit} \quad & 1u=1.66\times 10^{-27}\; kg \\ \\
        \mbox{rest mass of electron} \quad & m_e=9.11\times 10^{-31}\;kg \\ \\
        \mbox{rest mass of proton} \quad &  m_p=1.67\times 10^{-27} \; kg \\ \\
        \mbox{molar gas constant} \quad & R=8.31\;JK^{-1}mol^{-1} \\ \\
        \mbox{Avogadro constant} \quad & N_A=6.02\times10^{23}\; mol^{-1} \\ \\
        \mbox{Boltzmann constant} \quad & k=1.38\times 10^{-23} \; JK^{-1} \\ \\
        \mbox{gravitational constant} \quad & G=6.67\times 10^{-11} \; Nm^2kg^{-2} \\ \\
        \mbox{acceleration of free fall} \quad & g=9.81\;ms^{-2}
\end{align*}

\newpage
 \fancyhead[C]{DEFINITIONS}
\section*{Definitions}
\begin{flushleft}

 \textbf{Radian}-angle subtended at the centre of a circle by an arc of length equal to the radius of the circle.
 
 \textbf{Angular frequency}-2$\pi \times$ frequency/ $\frac{2\pi}{Period}$
 
 \textbf{Field of Force}- A region where a particle experiences a force.
 
 \textbf{Gravitational Force}-Force on mass in a gravitational field/ force acting between 2 masses/ force on mass due to another mass
 
 \textbf{Gravitational Field}-region in space that contains gravitational force / region in space where mass experience force 
 
 \textbf{Gravitational Field Strength}- gravitational force per unit mass/ratio of force to mass.
 
 \textbf{Gravitational Potential}- work done to bring a unit mass object from infinity to a point.
 
 \textbf{Newton's Law of Gravitation}- force experienced by 2 masses due to each other is proportional to the masses but inversely proportional to the square of separation of the masses.
 
 \textbf{Line of force in a gravitational field}- direction of force on a small test mass
 
 \textbf{Geostationary orbit}- equatorial orbit with period being 24 hours and from west to east.
 
 \textbf{Oscillation}-to-and-fro motion between 2 limits.
 
 \textbf{Forced Frequency}-Frequency at which an object is made to oscillate
 
 \textbf{Natural Frequency}-Frequency at which an object vibrates when free to do so
 
 \textbf{Resonance}-Maximum vibration of oscillating body when forced frequency is equals to natural frequency
 
 \textbf{Simple Harmonic Motion}- motion such that the acceleration is proportional to the displacement and is always directed towards to equilibrium position.
 
 \textbf{Avogadro's Constant, $N_A$}- number of carbon-12 atoms in 12g of carbon-12
 
 \textbf{A Mole}-a mole is any substance that must contain the Avogrado's constant number of molecules / amount of substance containing the same number of carbon-12 atoms in 12g of carbon-12
 
 \textbf{Ideal Gas}- a model of gas which has no inter-molecular forces and perfectly obeys $PV=nRT$ for all values of $P,V\&T$
 
 \textbf{Internal Energy}- total potential energy and random kinetic energy of the molecules in the gas.
 
 \textbf{Thermal Equilibrium}- same temperature and no net transfer of energy.
 
  \textbf{Specific heat capacity}- heat energy required to raise $1 kg$ of a substance by $1 ^{\circ} C$.
 
 \textbf{Specific Latent Heat of Fusion}-amount of thermal energy needed to change 1kg of a solid into liquid at a constant temperature (melting point)
 
 \textbf{Specific Latent heat of Vaporisation}-amount of thermal energy needed to change 1kg of liquid to vapour at a constant temperature (boiling point)
 
 \textbf{Zeroth Law of Thermodynamics}-If body A is in thermal equilibrium with body C, and body B is also in thermal equilibrium with body C, then body A and body B are in thermal equilibrium.
 
 \textbf{Thermometric Properties}-physical quantities where its magnitude is affected by temperature.
 
 \textbf{Coulomb's Law}- the force between two point charges is proportional to the product of the charges and inversely proportional to the square of the distance between them.
 
 \textbf{Electric Field}- a region containing electric force / a region in space where charge experience force
 
 \textbf{Electric Field Strength}- Force per unit positive charge.
 
 \textbf{Electric Potential}- work done to bring a unit positive charge from infinity to a specific point in space.
 
 \textbf{Electric Potential Energy}- work done needed to bring a charge from infinity to a specific point in space.
 
 \textbf{Capacitance}- ratio of charge to potential for a conductor.
 
 \textbf{Capacitance of a parallel plate capacitor}- Charge on a plate per unit potential difference across the plates.
 
 \textbf{Infinite Slew Rate}-No time delay between change in input voltage and change in output voltage
 
 \textbf{Transducer}- a device that converts one form of energy to another form.
 
 \textbf{Parallel-to-serial Converter}- device that converts a stream of multiple data elements, received simultaneously, into a stream of data elements transmitted in time sequence (i.e one at a time)
 
 \textbf{Attenuation}- gradual loss in signal power during transmission.
 
 \textbf{Amplitude Modulation/ Amplitude-Modulated Wave}- amplitude of the carrier wave varies in synchrony with the displacement of the signal wave.
 
  \textbf{Frequency Modulation/ Frequency-Modulated Wave}- frequency of the carrier wave varies in synchrony with the displacement of the signal wave.
  
  \textbf{Noise}- random unwanted signals that distort the signal we want to transmit
 
 \textbf{Cross-talk}-when a signal in a communication channel interferes with that in another
 
 \textbf{Bandwidth}-Range of frequencies of signals
 
 \textbf{Negative feedback}-Fraction of output voltage is combined with and subtracted from input
 
 \textbf{Magnetic field}- a region in space containing magnetic force / a region in space where a force is experienced by moving charge.
 
 \textbf{Magnetic Flux Density/ Magnetic Field Strength}- force exerted on a $1m$ length wire that is carrying $1A$ of current and placed perpendicularly to a magnetic field.
 
 \textbf{Tesla}-When a straight wire of 1m carrying current of 1A is placed normal to a magnetic field, the magnetic force exerted on it is 1 N. / Newton per metre per Ampere
 
 \textbf{Magnetic Flux Linkage}- product of magnetic flux density that is perpendicular to a surface and the area of the surface times the number of turns of the coil.
 
 \textbf{Faraday's Law}- magnitude of induced emf is directly proportional to the rate of change of magnetic flux linkage.
 
 \textbf{Lenz's Law}- induced electric current flows in a direction such that the current opposes the change that induced it.
 
 \textbf{Root-mean-square voltage/current}- the value of steady voltage/current that produces the same power across the same load as the alternating voltage/current.
 
 \textbf{Smoothing}- reduction in variation of voltage and current.
 
 \textbf{Radioactive decay}- The spontaneous disintegration of a nucleus to form a more stable nucleus, resulting in the emission of photons/particles
 
 \textbf{Binding Energy}- the energy required to separate all the nucleons of a nucleus to infinity .
 
 \textbf{Binding energy per nucleon}- the total energy needed to completely separate all the nucleons in a nucleus divided by the number of nucleons in the nucleus./Amount of energy released when a nucleus is formed from its nucleons.
 
 \textbf{Decay Constant}- probability of a nucleus to decay after a certain time interval.
 
 \textbf{Half-life}- time taken for the initial number of nuclei to reduce to half of its original number of nuclei.
 
 \textbf{Nuclear Fission}- splitting of a big nucleus into 2 or more approximately same size nuclei.
 
 \textbf{Nuclear Fusion}- merging of 2 or more smaller nuclei to form a larger nuclei.
 
 \textbf{Photoelectric Emission}- the release of electrons from the surface of a metal  when electromagnetic radiation is incident on its surface.
 
 \textbf{Photon}- a packet/bundle of energy of an electromagnetic radiation which has an energy value equal to the product of Planck constant and the frequency of the electromagnetic radiation.
 
 \textbf{Work Function}- minimum energy require to remove an electron from the surface of the metal.
 
 \textbf{De Broglie's Hypothesis}- Any particle of mass $m$, moving with velocity $v$, would have an associated wavelength,$\lambda$ which is inversely proportional to its momentum,$p$.
 
 \textbf{De Broglie Wavelength}- the associated wavelength, which is inversely proportional to the momentum, $p$ of a particle that has mass $m$, moving with velocity $v$.
 
 \textbf{Specific acoustic impedance}- the product of the density of a medium and the speed of the wave in the medium.
 
 \textbf{Intensity reflection coefficient,$\alpha$}-ratio of reflected intensity to incident intensity
 
 
 
 \begin{comment}
 
 \textbf{Hoi}- the computer gg
 
 \textbf{UWU}-uwu (´$\cdot \omega \cdot$`)
 
 \textbf{XD}-Eksdee \includegraphics[scale=.05]{XD.png}
 
 \textbf{Yosha}- \includegraphics[scale=0.05]{Pepe.jpg}
 
\textbf{$\int$}- snack
\end{comment} 
 
 
\end{flushleft}
\newpage 

\fancyhead[C]{SOURCES}
\section*{Sources}
\begin{flushleft}

Coaxial cable. Shaik, Asif. \textit{Amplitude Modulation.}Source: www.physics-and-radio-electronics.com/blog/wp-content/uploads/2018/05/amplitudemodulation.png

Amplitude Modulation. \textit{WatElectronic.com} Source: https://www.watelectronics.com/what-is-amplitude-modulation-derivatives-types-and-benefits/

Amplitude Modulation Spectrum. \textit{electronicsnotes}. Source: https://www.electronics-notes.com/articles/radio/modulation/amplitude-modulation-am-bandwidth-spectrum-sidebands.php

Frequency modulation. \textit{physics and radio-electronics.}Source: https://www.physics-and-radio-electronics.com/blog/frequency-modulation/

Electric potential of a charged sphere. \textit{Hyperphysics}. Source: http://hyperphysics.phy-astr.gsu.edu/hbase/electric/potsph.html

Wire strain gauge. \textit{ForumAutomation.com}.Source: https://forumautomation.com/t/strain-gauge-principle-types-advantages-applications/4949

Magnetic field of a solenoid. \textit{Mini physics}. Source: https://www.miniphysics.com/ss-magnetic-field-due-to-current-in-a-solenoid.html

North and South pole of a solenoid. \textit{MammothMemory}. Source: https://mammothmemory.net/physics/magnets-and-electromagnetism/electromagnetism/which-end-of-an-electromagnet-is-north.html

Direction of magnetic field lines around a conducting wire. \textit{physics.stackexchange}. Source: https://physics.stackexchange.com/questions/426662/direction-of-magnetic-field-lines-around-a-conducting-wire


Fleming's Left Hand Rule.\textit{PNGWING}. Source: https://www.pngwing.com/en/free-png-zefog

Motion of charged particle around conducting wire. \textit{The Open Door Website}. Source: https://www.saburchill.com/physics/chapters/0055.html

Fleming's Right Hand Rule. \textit{byju's}. Source: https://byjus.com/physics/flemings-left-hand-rule-and-right-hand-rule/

Self inductance. \textit{Iowa State University}. Source: https://www.nde-ed.org/Physics/Electricity/conductorsinsulators.xhtml

Transformer. \textit{Electrical 4 U}. Source: https://www.electrical4u.com/what-is-transformer-definition-working-principle-of-transformer/

Eddy current reduction. \textit{Wikipedia}. Source: $\text{https://en.wikipedia.org/wiki/Eddy\textunderscore current\#/media/File:Laminated\textunderscore core\textunderscore eddy\textunderscore currents\textunderscore 2.svg}$

Graph of binding energy per nucleon against mass number. \textit{ReseachGate}. Source: $\text{https://www.researchgate.net/figure/Binding-energy-per-nucleon-B-Z-N-A-as-a-function-of-the-mass-number\textunderscore fig1\textunderscore 234065870}$


Photoelectric effect. \textit{CR4-GlobalSpecs}. Source: https://cr4.globalspec.com/thread/126252/Does-the-Photoelectric-Effect-Require-a-Battery-or-a-Circuit

Diffraction of electron. \textit{schoolphysics}. Source: $\text{https://schoolphysics.co.uk/age16-19/Wave\%20properties/Wave\%20properties/text/Electron\textunderscore diffraction/index.html}$

Absorption spectrum. \textit{Khan Academy}. Source: https://www.khanacademy.org/science/class-11-chemistry-india/xfbb6cb8fc2bd00c8:in-in-structure-of-atom/xfbb6cb8fc2bd00c8:in-in-bohr-s-model-of-hydrogen-atom/a/absorptionemission-lines


juilee. \textit{Ques 10}. Source https://www.ques10.com/p/14359/explain-formation-of-energy-bands-in-solids-and-ex/

Band theory. \textit{Lumen Learning}. Source: https://courses.lumenlearning.com/boundless-chemistry/chapter/band-theory-of-electrical-conductivity/

Production of Bremmstrahlung radiation. \textit{Radiology Key}. Source: https://radiologykey.com/x-ray-production-2/

X-ray spectrum. \textit{ResearchGate.} Source:

\text{https://www.researchgate.net/figure/2-Schematic-spectrum-of-an-X-ray-tube-showing-Bremsstrahlung-and-K-Lines}
\text{-65\textunderscore fig5 \textunderscore 262414327}


Graph of intensity agaisnt thickness. \textit{Cyberphysics.} Source: https://www.cyberphysics.co.uk/topics/medical/Xray.html

A-scan. \textit{A Cyberphysics Page.} Source: $\text{https://www.cyberphysics.co.uk/Q\&A/KS5/medical/ultrasound/questions.html}$

B-scan. \textit{ReseachGate.} Source: $\text{https://www.cyberphysics.co.uk/Q\&A/KS5/medical/ultrasound/questions.html}$


\end{flushleft}


%\begin{comment}
\newpage
\input{subtitle page}


 \fancyhead[C]{Paper 5}
\section*{Guidelines}
Question 1 (Planning an Experiment) – 15 marks
 \begin{enumerate}
     \item Read question given carefully on what is required from experiment
      \item Design a suitable experiment based on the question, either to test whether the suggested relationship between 2 variables is valid or to determine the value of a constant
     \item General format of answers:
 \end{enumerate}


    

   
   

    \begin{longtable}{|p{5cm}|p{11.5cm}|}
  

    \hline
           \textbf{Independent variable} &  • Independent variable is variable that is controlled/manipulated in experiment  (is usually the x- axis value) \\ 
        
    \cline{1-1}
       \textbf{Dependent variable} &    •	Dependent variable is variable that is measured from experiment (is usually the y-axis value)\\
    &•	For the independent variable and dependent variables,
they can be obtained easily from the equation given in question itself (after rearranging in the form of $y=mx+c$)\\

    \hline 
   \textbf{Variables to be kept Constant}& •	For the variables to be kept constant, there is often more than one and you should give all of them that you can think of\\
    &•	These are variables that will directly affect the dependent variable\\
    &•	Note that the term 'kept constant' or 'keep constant' must be used in your answer and no other terms are acceptable (do not use ‘controlled’)\\
    &•	Occasionally, the variables to be kept constant can also be
    deduced from the equation given\\
    
    \hline  
   \textbf{Diagram Set-up}&•	For the diagram of set-up apparatus, show all the apparatus and material used in the experiment and label them correctly\\
    &•	If a diagram is provided by the question, the same diagram can be used, but it is usually not complete and you have
    to add other apparatus or materials to it\\
    \hline
    \textbf{Procedures}\newline
    •	varying	independent variable\newline
    •	measuring independent variable\newline
    •	measuring	dependent variable\newline
    •	keeping other variables constant
    &•	When writing the procedures, include the steps to measure the independent variable, dependent variable and how variables are kept constant\newline
    •	Describe also how the independent variable is varied (if applicable)\newline
    •	For each measurement, state the method and apparatus used\newline
    •	For some variables, other quantities may need to be
    measured in order to calculate them. Describe the steps involved and write the equation that can be used \textit{(with the additional variables defined)} \newline
    •	Other steps required to set up the apparatus should also be
    included if they are not shown in the diagram.\\
    \hline
    \textbf{Data Analysis}&•	For the analysis of data, explain how a suitable graph should be plotted that enables you to \textit{test whether the suggested relationship between the 2 variables is valid or to determine the value of the constant}\newline
    •	This often involves rearranging the equation into the general form of a straight line equation $(y=mx+c)$\newline
    •	The gradient and y-intercept is often used to determine the value of constants\newline
    •	Sometimes, the graph may need to be changed a log graph \textit{(if equation given is raised to the power of something)} or $ln$ graph \textit{(if equation given is an exponential function)}\newline
    •	A sketch of the graph may also be shown\newline
    •	To explain whether the suggested relationship is valid: \textbf{\textit{“The relationship is valid if a straight line graph that passes through the origin / does not pass through the origin is produced”}}\newline
    •	When determining the constant based on the graph, \underline{\textbf{always make the constant the subject}}\textit{ (Do not leave it ‘hanging’)}\newline\newline
    \textbf{Correct}:$K=\cfrac{V}{Gradient}$\newline\newline
    \textbf{Wrong}:$Gradient=\cfrac{V}{K}$\\
    \hline
    \textbf{Safety Precautions}&•	State clearly reasoned precautions relevant to the experiment \textit{(at least one potential safety hazards when carrying out the experiment and the ways to avoid each of them)}\\
    \hline
    \textbf{Additional Details}&•	For additional details, they include:\newline
    \textit{-	descriptions on how variables are kept     constant\newline
    -	calibration of the measuring instruments\newline
    -	details on how other variables in the given equation can be obtained\newline
    -	additional steps to improve the accuracy and reliability\newline
    -	ways to reduce potential errors\newline}
    •	Some of them may have already been stated in other parts of your answer and it is not compulsory to write them again in this part. In this section, you should state the additional details that are not stated in other parts of your answer.\\
    \hline
    \end{longtable}

\begin{itemize}
    \item For all parts of the question, you are allowed to write extra points in your answer, but you must be careful not to write any points with wrong facts. Marks will only be given for the correct points. For points that are irrelevant but does not contain wrong facts, no mark will be given or deducted. However, for points that contain wrong facts, marks may be deducted.
    \item You should use the correct experimental and Physics terms in your answer. Do not replace them with other terms that are inappropriate, even if their meaning are the same. You should spell all experimental and Physics terms correctly. If you cannot do so, try to spell it in such a way where it sounds the same as the actual term when read out. Marks are usually not deducted for spelling errors in experimental and Physics terms as long as it still sounds the same and that it is not easily confused with other terms. If you spell other terms wrongly or if you make grammatical errors in your answer, marks will not be deducted for as long as the examiner can understand what you are writing.
    \item 	You are allowed to use suitable short forms in your answer, especially for representing physical quantities or their units.

\end{itemize}
\newpage
\begin{flushleft}
\textbf{Question 2 (Analysis, Conclusions and Evaluation of Experiment)}\\

\end{flushleft}
\begin{enumerate}
    \item \textbf{Determining expressions for gradient and y-intercept}
    \begin{itemize}
        \item Read the question given carefully and determine what the question is asking for
        \item 	An equation related to the experiment is usually given, from which it needs to be rearranged into the general equation of a straight line ( $y=mx+c$ )
        \item 	Once re-arranged, determine the expressions for \textbf{gradient} and \textbf{y-intercept}
    \end{itemize}
    \item \textbf{Calculating Values in Table}
    \begin{itemize}
        \item The number of significant figures (s.f.) in the calculated value must be equal to or one s.f. more than the least s.f. of the raw values used
        \item When calculating the absolute uncertainty, there are 2 methods:
        \begin{enumerate}
            \item Consequential Uncertainty Equations (from AS-level)\\
            -	This method CANNOT be used if the expression has trigonometric ($sin x, cos x, tan x$), logarithmic ($lg x$ or $ln x$), exponential ($e^x$) function\\
            -To calculate uncertainty:
            \begin{longtable}{|p{7cm}|p{7cm}|}
              \hline
              If $ y= a + b$ or $y = a - b$ \newline
              $\Delta y=\Delta a + \Delta b $ & If $y=a\times b$ or $y=\cfrac{a}{b}$ \newline\newline
              Expressed in fractional uncertainties, \newline
              $\cfrac{\Delta y}{y}=\cfrac{\Delta a}{a}+\cfrac{\Delta b}{b}$ \newline\newline
              Expressed percentage uncertainties, \newline
              $\left(\cfrac{\Delta y}{y}\times 100\right)=(\cfrac{\Delta a}{a}\times 100)+\left(\cfrac{\Delta b}{b}\times 100\right)$ \newline\\
              \hline 
              \newline 
              If $y = a^mb^n$ or $y = \cfrac{a^m}{b^n}$\newline
              Then.\newline
             $\cfrac{\Delta y}{y}=m\left(\cfrac{\Delta a}{a}\right)+n\left(\cfrac{\Delta b}{b}\right)$ \newline & \newline When $N$ is a constant value: \newline  
             If $y = Na$ \newline  
             Then,\newline
             $y=N(a\pm \Delta a)$ \\
             
             \cline{3-1}
             \newline
             If $y = a\pm N\%$\newline
             Then,\newline
             $\Delta y = a \times \cfrac{N}{100}$&
             
             $\Delta y = N\Delta a$ \newline
             If $y=\cfrac{a}{N}$ \newline
             Then, \newline
             $y=\cfrac{1}{N}(a\pm \Delta a)$ \newline
             $\Delta y =\cfrac{1}{N}\Delta a$ \newline\\
             \hline
             
            \end{longtable}
    
            \item Maximum/Minimum Method (applicable for all cases)\\
            -	If the calculation is too complicated or it involves trigonometry, logarithms, exponentials, this is an alternative method\\
            -	Determine either the maximum value and minimum value of the calculated value by using the maximum values and/or minimum values of the raw values\\
            -	The maximum/minimum values are determined based on the uncertainties of the raw values\\
            -   For any calculated value of $X$, the absolute uncertainty is found from any of these equations:
            $$\Delta X = X_{max} - X$$
            $$\Delta X = X - X_{min}$$
            $$\Delta X = \cfrac{X_{max}-X_{min}}{2}$$

    
        \end{enumerate}
        \item Absolute uncertainties DO NOT need to be expressed to strictly 1 s.f. anymore
        \item It can be expressed to up to 2 s.f. or 3 s.f.
        \item \textbf{NOTE: Follow the s.f. rule first for calculated value, then from the number of decimal places available in calculated value, match the number of d.p. of absolute uncertainty to the calculated value}
    \end{itemize}
    \item \textbf{Plotting Graph}
    \begin{itemize}
        \item Plot all points on the graph accurately. Points should be accurate to half a small square
        \item Recommended good practices:\\
            -   Write the values of one large square and one small square for the x and y axes\\
            -	To determine the position of a plot point, say (A, B)\\
            -	Find labelled value on x-axis that is lesser than A\\
            -	Calculate the difference between A and the x value, before dividing by value of small square on x-axis\\
            -   This indicates the number of small squares that A should be positioned from the x-value:
            $$\mbox{No. of small squares}=\cfrac{A-x}{\mbox{Small square value}}$$
            -   Repeat the same steps in y-axis for B, so that:
            $$\mbox{No. of small squares}=\cfrac{B-y}{\mbox{Small square value}}$$
        \item Draw error bars based on the uncertainties of the values given in the question
        \item After that, draw line of best fit. The line should pass through as many points on the graph as possible, all the points should be close to it and the number of points above and below the line should be almost equal.
        \item Next, draw the worst acceptable line. It should be either the steepest possible line which passes from the top of top error bar to the bottom of bottom error bar or the shallowest possible line which passes from the bottom of top error bar to top of bottom error bar. The line must also pass through all error bars.
        \item Where a dashed line is used to represent the worst acceptable line, the dashed parts of the line should cross the error bars
        \item Clearly label both the best and worst fit lines
    \end{itemize}
    \item \textbf{Determining Gradient}
    \begin{itemize}
        \item For the uncertainty in gradient, determine the gradient of worst acceptable line using the same method. The uncertainty in gradient is equal to the difference between the 2 gradients:
        $\Delta Gradient = |Gradient_{worst}-Gradient_{best}|$
        \item \textbf{It is good practice to write the values of the points used for the best fit and worst fit line before doing the calculations for gradient and y-intercept}
        \item Be careful of power-of-ten (POT) errors. Always check the axes carefully when substituting values for gradient calculation
        \item Units do not necessarily need to be written down in final answer, but is good practice to do so
        \item When considering the units to write, note that certain quantities do not have units, such as:\\
        Trigonometric ratios ($sin \;x,\; cos \;x,\; tan \;x$)\\
        Logarithmic values ($lg \; or\; ln$)
    \end{itemize}
    \item \textbf{Determining y-intercept}
    \begin{itemize}
        \item When determining the y-intercept of line of best fit, usually it cannot be read off directly from the y-axis of graph since the x-axis does not start from 0.
        \item Instead, choose \textbf{a point on the best fit line}, preferably one of the points that was used to calculate its gradient, and substitute its $x$ and $y$ values as well as the \textbf{gradient of best fit} into the equation $y=mx+c$ to determine the value of c which is the y-intercept.
        \item Then, determine the y-intercept of worst acceptable line using the same method. 
        \item Use \textbf{a point on worst fit line} and \textbf{gradient of worst fit line}
        \item Uncertainty in y-intercept is equal to the difference between the 2 y-intercepts:
        $$\Delta(y-intercept) = |(y-intercept)_{worst} - (y-intercept)_{best}|$$
    \end{itemize}
    \item \textbf{Further Calculations}
\begin{itemize}
    \item For subsequent calculations, the final answer is usually expressed in \textbf{2 or 3 s.f.}\\
    Values used in intermediate calculations are usually expressed in more s.f. before rounding up in the final answer.
    \item For absolute uncertainties, the number of decimal places should be \textbf{equal to the number of decimal places} of the calculated value
    \item Show all workings and do not skip any important steps
    \item Show the values substituted into the equations \textbf{(correct substitution of numbers must been seen)}
    \item When asked to include absolute uncertainties in final answer, either use “consequential uncertainty equations” or “max/min method” to calculate the uncertainty (depends on which is easier to apply)
    \item It is good practice to write the correct unit for the final answer if it is not provided
    \item For a calculation question which requires the answer from the previous question, even if the answer for the previous question is wrong and you use it for this question causing your answer for this question to be wrong, full marks is usually still awarded for this question as long as the calculation for this question is correct. This is known as 'error carried forward'.
    \item Certain questions may require your knowledge and understanding in Physics to answer them

\end{itemize}
\end{enumerate}

\begin{flushleft}
\textbf{Useful points for Paper 5} \\ 
General Measurements
\end{flushleft}
\begin{enumerate}
    \item To measure length, $L$ 
    \begin{itemize}
        \item Use a metre rule, ruler or caliper to measure length
        \item 	For measurements with metre rule/ruler, make use of a fiducial mark (i.e. pointer/pin/set square/marked position) to help in reading accurately from scale
* A fiducial mark is a visual mark that is used as a point of reference when taking a measurement *
        \item 	To improve accuracy of measurement, clamp ruler to a retort stand (either horizontally or vertically)
    \end{itemize}
    
    \underline{To determine distance from centre of object to a surface:}
    \begin{itemize}
        \item Measure from surface to top and bottom of object and take average
    \end{itemize}
$$h=\cfrac{h_{top}-h_{btm}}{2}$$
\begin{figure}[H]
    \centering
    \includegraphics[scale=.7]{measure distance 1.PNG}
\end{figure}
    
    \underline{To determine distance from centre of object to centre of another similar object}
    \begin{itemize}
        \item Measure top-to-top distance
        \item Measure bottom-to-bottom distance
        \item Measure distance between top surface and bottom surface, then add the value of radius (after measuring diameter separately, and finding the radius)
    \end{itemize}
    \begin{figure}[H]
        \centering
        \includegraphics[scale=.7]{measure distance 2.PNG}
    \end{figure}
    $$h=l+r+r$$
    
    \underline{To determine extension, $e$ of spring/cord}
    \begin{itemize}
        \item Clamp metre rule vertically to a retord stand 
        \item Attach pointer needle on spring/cord
	\item Measure un-stretched length, $L_O$ before load is applied
    \item	Measure stretched length, L after load is applied
\item Use equation: $e = L-L_O$
    \end{itemize}
    
    \item To measure diameter, $D$
    \begin{itemize}
        \item 	Use a micrometer screw gauge (for small diameters like wires)
\item Take repeated measurement of D along length of wire / different angles across wire
    \end{itemize}
    \textbf{OR}
    \begin{itemize}
        \item Use a vernier caliper (for larger diameters like tubes)
\item Vernier caliper can be used to measure either internal or outer diameters of a tube
    \end{itemize}

    \item To measure temperature, $T$
    \begin{itemize}
        \item Use a thermocouple attached to the object/immersed in the liquid
    \end{itemize}
    \textbf{OR}
    \begin{itemize}
        \item Use a liquid-in-glass thermometer for liquids (whose temperature measured is not to exceed 100°C)
\item Stir well to ensure temperature is uniformly distributed
\item 	Wait for thermal equilibrium to be reached before taking temperature
    \end{itemize}
    
    \item To measure angle, $\theta$
    \begin{itemize}
        \item Use protractor
    \end{itemize}
    \textbf{OR}
    \begin{itemize}
        \item Measure vertical distance, $y$ and horizontal distance, $x$ using a ruler

\item To find $\theta$, use equation: $\tan \theta = \cfrac{y}{x}$
    \end{itemize}

\item To measure  mass, $m$
\begin{itemize}
    \item 	Use a mass balance to measure $m$
\item To find load/weight applied, use equation: $W = mg$
\end{itemize}

\item To measure volume, $V$
\begin{itemize}
    \item Use a measuring cylinder or calibrated beaker
\end{itemize}

\item When repeating measurements
\begin{itemize}
    \item Example, when measuring velocity, $v$(dependent variable) at an angle, $\theta$(independent variable) for an object moving on an inclined plane
\item	“For each $\theta$, repeat experiment and find average $v$”
\item	General form of answer: \textbf{“For each (independent variable), repeat experiment and determine average (dependent variable)”}
\end{itemize}

\item To check whether an object is horizontal 
\begin{itemize}
    \item Use a spirit level
\end{itemize}

\item To check whether an object is perpendicular to a surface/ vertical to bench
\begin{itemize}
    \item Use set squares
\end{itemize}

\item To fix an object in place
\begin{itemize}
    \item Use retort stand and clamp
\item	Use tape
\item 	To vary the position of object, adjust the position of the clamp
\end{itemize}

\item To vary inclination or angle of an inclined plane 
\begin{itemize}
    \item 	Use retort stand and clamp
\item	Use tape
\item 	To vary the position of object, adjust the position of the clamp
\end{itemize}

\item To check whether an object starts to lose contact with a surface/starts to move 
\begin{itemize}
    \item 	Use a high speed video camera
\item	Slowly/Gradually/Incrementally increase or decrease frequency until object is observed to start losing contact with surface / starts to move

\end{itemize}
\underline{If determining point where object starts to lose contact}
\begin{itemize}
    \item Example: Mass on vibrating plate
\item	Place video camera at approximately same level as vibrating plate when looking for separation between object and surface
\end{itemize}
\underline{If determining point where object starts to move}
\begin{itemize}
    \item 	Example: Mass on turntable
\item	Place video camera directly on top of object
\item	Draw mark/guide to indicate original position of object

\end{itemize}

\item To determine when an LED is just lights up
\begin{itemize}
    \item 	Use a light meter/detector or LDR
\item	There will be a change in reading of light meter/detector or voltmeter connected across LDR when LED just lights up
\item	Slowly increase or decrease p.d. until LED just lights up
\end{itemize}

\item To ensure two surfaces are parallel
\begin{itemize}
    \item Measure distance between the surfaces at several locations
\item	If the surfaces are parallel, the distances should be approximately the same, $d_1\approx d_2 \approx d_3$
\end{itemize}
\begin{figure}[H]
    \centering
    \includegraphics[scale=.8]{measure distance 3.PNG}
\end{figure}

\end{enumerate}

\begin{flushleft}
\textbf{Electrical-related Measurements}
\end{flushleft}
\begin{enumerate}
    \item To measure resistance, $R$
    \begin{itemize}
        \item Use an ohmmeter
\item	Ohmmeter needs to be connected in parallel across the component whose resistance is to be measured
\item	\textbf{DO NOT} draw a power supply connected in series with an ohmmeter (ohmmeter does not need to power supply to operate)
    \end{itemize}
    \begin{figure}[H]
        \centering
        \includegraphics{measure 1.PNG}
    \end{figure}
    \textbf{OR}
    \begin{itemize}
        \item Use an ammeter (connected in series) and a voltmeter (connected in parallel) to the component whose resistance is to be measured
\item	Use equation: $R=\cfrac{V}{I}$
    \end{itemize}
    
    \item To measure potential difference, $V$
    \begin{itemize}
        \item Use a voltmeter\textbf{ connected in parallel} to component
\item	For circuits with alternating voltage, state “\textbf{A.C. voltmeter}”
\item	Alternatively, use an “\textbf{multimeter}”
\item	State whether r.m.s. voltage, $V_{rms}$ or peak voltage, $V_0$ is measured
-	Related equation:$V_{rms}=\cfrac{V_0}{\sqrt{2}}$
    \end{itemize}
 
 \item To measure current, $I$   
    \begin{itemize}
        \item Use an ammeter connected in series to component
        \item For circuits with alternating current, state “A.C. Ammeter”
\item	Alternatively, use an “multimeter”
\item	State whether r.m.s. current, $I_{rms}$ or peak current, $I_0$ is measured
\item Related equation: $I_{rms}=\cfrac{I_0}{\sqrt{2}}$

    \end{itemize}
    
    \item To measure power, $P$
    \begin{itemize}
        \item Measure current $I$ and p.d., $V$
        \item Use equation: $P=VI$
    \end{itemize}
    
    \item To vary current or p.d
    \begin{itemize}
        \item 	Use a variable resistor connected in series to circuit
\item 	Vary the resistance in order to vary the current flowing or p.d. across component
\item	Connect an ammeter in series to determine current flowing
\item Connect a voltmeter in parallel across component to determine p.d.

    \end{itemize}
    \begin{figure}[H]
        \centering
        \includegraphics[scale=.5]{measure 2.PNG}
    \end{figure}


\underline{To ensure that current is constant}

The same arrangement can also be used to keep current constant (if current is expected to
fluctuate in circuit)
\begin{itemize}
    \item Adjust the resistance of variable resistor to keep current constant
\item Ammeter reading would be constant
\end{itemize}
(Note: P.d. across component cannot be kept constant by adjusting variable resistor.
Adjusting the variable resistor will definitely change the p.d. across the component)

\item To measure frequency, $f$ of alternating current/p.d. supplied
\begin{itemize}
    \item Use a variable frequency A.C. power supply
\item  Vary frequency by adjusting frequency on power supply
\item Determine frequency by reading off the power supply
\end{itemize}
\textbf{OR}
\begin{itemize}
    \item Use a signal generator connected to power supply
\item Vary frequency by adjusting frequency on signal generator
\item Determine frequency by reading off the signal generator
\end{itemize}
\textbf{OR}
\begin{itemize}
    \item Connect a CRO in parallel to the electrical component
\item Measure the time taken for one cycle, $T$
\item Use equation: $T= \; Length \; of \;one \; period \;\times \; Time\; base$
\item Determine frequency, $f$ by using equation: $f=\cfrac{1}{T}$\\
\begin{minipage}[t]{0.5\textwidth}
\includegraphics{Capture.PNG}
\end{minipage}
\begin{minipage}[t]{0.5\textwidth}
 \includegraphics{v.PNG}
\end{minipage}
\end{itemize}

\item To vary and measure alternating p.d. supplied

 \begin{itemize}
     \item Use a variable A.C. power supply
\item Vary p.d. by adjusting voltage setting on power supply
\item Determine p.d. by reading off the power supply
 \end{itemize}
\textbf{OR}
\begin{itemize}
    \item Use a signal generator connected to power supply
\item Vary alternating p.d. by adjusting p.d. on signal generator
\item Determine p.d. by reading off the signal generator
\end{itemize}
\textbf{OR}
\begin{itemize}
    \item Connect a CRO in parallel to the electrical component
\item  Measure amplitude of waveform in CRO
\item Use equation: $V_o=Maximum \; height \;\times\; Voltage\; Gain$
\end{itemize}

\textbf{Note}:Connecting a variable resistor in the circuit and adjusting its resistance could also
be used to vary the p.d. or current across a component (without changing anything in the
A.C. power supply or signal generator)
\begin{figure}[H]
    \centering
    \includegraphics{1,2.PNG}
\end{figure}

\item To measure heat energy, Q supplied electrically to a liquid

\begin{itemize}
    \item Use an electrical heater/heating coils to heat up liquid
\item  For the heating element, determine the potential difference, V and current, I
\item  Determine the time, t used to heat up the liquid
\item Use equation: 
\begin{align*}
    P&=VI \\ 
    \cfrac{Q}{t}&=VI \\ 
    Q&=VIt
\end{align*}
\item Add insulating material to container to prevent heat loss
\end{itemize}

If heating up a solid object
\begin{itemize}
    \item Immerse the object in a water/oil bath
\item Apply the same steps above
\end{itemize}
\end{enumerate}

\begin{flushleft}
\textbf{Motion-related Measurements}
\end{flushleft}
\begin{enumerate}
    \item To measure velocity, v of an object
    \begin{flushleft}
    a) Using stopwatch
    \end{flushleft}
    \begin{itemize}
        \item Least reliable but still accepted in marking scheme (if unsure about other methods)
\item Using a stopwatch, measure the time taken, t to travel a distance s
\item  $v$ can be found from equation: $v=\cfrac{s}{t}$
    \end{itemize}
    
    \begin{flushleft}
    b)Using two light gates
    \end{flushleft}
    \begin{itemize}
        \item Mount a piece of card on the moving object
\item  Position two light gates connected to a timer/data logger a distance s apart
\item As object moves and cuts the 1st light gate, it starts a timer
\item When the object cuts the 2nd light gate, it stops the timer
\item Obtain time taken, to travel through s from timer/data logger
\item $v$ can be found from equation: $v=\cfrac{s}{t}$
\item Note: 
    \begin{enumerate}
        \item If there is acceleration, this method finds average velocity
        \item If there is no acceleration, this method finds the constant velocity
throughout the motion
    \end{enumerate}
\begin{figure}[H]
    \centering
    \includegraphics{3.PNG}
\end{figure}
    \end{itemize}
    
    \begin{flushleft}
    c) Using a single light gate
    \end{flushleft}
    \begin{itemize}
        \item Mount a piece of card of length $x$ on the moving object
\item  Position a light gate connected to a timer/data-logger at position indicated in diagram
(say point P)
\item As object moves and card cuts the light gate, the light gate measures the time taken, $t$
for length of card to interrupt light beam
\item  $v$ at that point P can be found from equation: $v=\cfrac{x}{t}$
\item Note: 
\begin{enumerate}
    \item If there is acceleration, this method finds instantaneous velocity at that
point only
\item If there is no acceleration, this method finds the constant velocity
throughout motion
\end{enumerate}
\begin{figure}[H]
    \centering
    \includegraphics{4.PNG}
\end{figure}
    \end{itemize}
    
    \begin{flushleft}
    d)Using motion sensor 
    \end{flushleft}
    \begin{itemize}
        \item Place a motion sensor connected to a data logger at location shown in diagram,
(typically is at the back of moving object)
\item Measure distance, $x$ of motion sensor to the point where velocity needs to be
determined (say point P)
\item The motion sensor connected to data logger records the variation of distance with time
of the moving object
\item A graph of distance versus time can be produced from data logger, from which
velocity, v can be determined at distance $x$ (which is point P)
(This is found by the gradient of the distance-time graph produced from data-logger)
    \end{itemize}
    \begin{figure}[H]
        \centering
        \includegraphics{5.PNG}
    \end{figure}
    \begin{flushleft}
    NOTE:
    \end{flushleft}
    \begin{itemize}
        \item The positions in which light gates and motion sensor are mounted in a motion
experiment are often different
\item Light gates can be connected to either a timer or data logger and must always be in
the path of moving object (such that a card cuts the light beam of light gate)
\item  Motion sensor is only connected to data logger and must be either at the back or front
of moving object (never in the path of the moving object)
    \end{itemize}
 
 \item    To measure acceleration, $\vec{a}$ of an object
 \begin{flushleft}
 a) Using a single light gate and kinematic equation
 \end{flushleft}
    \begin{itemize}
        \item Assuming the object start moving from rest (u = 0) through a distance of $s$ from
point P to Q
\item Mount a piece of card of length $x$ on the moving object
\item  Position a light gate connected to a timer/data-logger at point Q
\item  As object moves and card cuts the light gate, the light gate measures the time taken, $t$
for the card to pass through
\item  Final velocity, $v$ at Q can be found from equation: $v=\cfrac{x}{t}$
\item Use kinematics equation: $$v^2=u^2+2as \\ \implies a=\cfrac{v^2}{2S}$$
    \end{itemize}
    \begin{flushleft}
    Note:This method applicable is only for constant acceleration
    \end{flushleft}
    \begin{figure}[H]
        \centering
        \includegraphics{6.PNG}
    \end{figure}
    \begin{flushleft}
    b) Using two light gates
    \end{flushleft}
    \begin{itemize}
        \item Mount on the moving object a piece of card with lengths $x_1$ and $x_2$ as shown in the diagram
\begin{figure}[H]
    \centering
    \includegraphics{7.PNG}
\end{figure}
        \item Position a light gate connected to a timer/data-logger at point P
        \item As object moves and card cuts the light gate, the light gate measures the time taken for specified lengths of card to travel through light gate:
        \begin{enumerate}
            \item $t_1$ for $x_1$
            \item $t_2$ for $x_2$
            \item $t$ for time taken between first and second cutting of light gate
        \end{enumerate}
       \item Determine initial velocity ], $u=\cfrac{x_1}{t_1}$ and final velocity, $v=\cfrac{x_2}{t_2}$
       \item Finally, use equation: $a=\cfrac{v-u}{t}$ to find acceleration at $P$
       \begin{figure}[H]
           \centering
           \includegraphics{8.PNG}
       \end{figure}
      \begin{flushleft}
      Note: \begin{enumerate}
          \item This method can determine constant acceleration throughout motion
\item  This method can determine instantaneous acceleration at a particular point (if acceleration is not constant throughout motion)
      \end{enumerate}
      \end{flushleft}
    \end{itemize}
    
    \begin{flushleft}
    c) Using motion sensor
    \end{flushleft}
    \begin{itemize}
        \item Place a motion sensor connected to a data logger at a fixed location, at the back of the
moving object
\item Measure distance, $x$ of motion sensor to the point where acceleration needs to be
determined (say point P)
\item The motion sensor connected to data logger records the variation of distance, $x$ with
time, $t$ as the object moves
\item From data recorded in data logger, the acceleration at point $P$ (at distance $x$) can be
deduced
\begin{figure}[H]
    \centering
    \includegraphics{9.PNG}
\end{figure}

    \end{itemize}
    \item To measure maximum height reached by a moving object
    \begin{itemize}
        \item Use video camera with slow-motion playback
    \end{itemize}
\end{enumerate}

\begin{flushleft}
\textbf{Magnetic-related Measurements}
\end{flushleft}
\begin{enumerate}
    \item To measure magnetic flux density, $B$
    \begin{itemize}
        \item Use a calibrated Hall probe to measure $B$
        \item $B$ is proportional to Hall voltage, $V_H$ on Hall probe
        \item  Calibrate Hall probe with a known magnetic field / magnetic flux density
        \item  Place Hall probe at right-angles to direction of magnetic field
        \item To ensure maximum $V_H$ is obtained, rotate Hall probe 180° until maximum reading is
obtained
        \item  Reverse Hall probe direction when repeating experiment to determine average
        \item Avoid external magnetic fields / placing other metallic objects nearby
        \item Use large current to create large magnetic field \textit{(if applicable)}
        \item Use large number of coils to create large magnetic field \textit{(if applicable)}
    \end{itemize}
    
    \item To measure e.m.f induced, $V$
    \begin{itemize}
        \item Use a CRO or A.C. voltmeter connected in parallel to circuit to determine e.m.f.
induced
        \item  Use high frequency A.C. to produce larger induced e.m.f.
        \item Use an iron core to produce larger induced e.m.f.     \item Use large current/number of turns of coil to produce large induced e.m.f.
    \end{itemize} 
    
\end{enumerate}

\begin{flushleft}
\textbf{Sound-related Measurements}
\end{flushleft}
\begin{enumerate}
    \item To produce sound waves of
    \begin{itemize}
        \item  Connect a speaker to a signal generator
        \item Signal generator can be used to vary the frequency of sound wave
    \end{itemize}
    
    \item To measure sound wave
    \begin{itemize}
        \item Use a microphone / sound detector 
        \item Connect to sound meter to measure intensity
        \item Connect to CRO to measure amplitude or frequency
        \item Conduct experiment in a quiet room / Ensure no other sources of sound
        \item Use sound barrier to isolate unwanted sound
    \end{itemize}
\end{enumerate} 
\begin{flushleft}
\textbf{Wind-related Measurements}
\end{flushleft}
\begin{enumerate}
    \item To measure wind speed, $v$
    \begin{itemize}
        \item Use a fan to generate wind
        \item  Use a wind speed meter or anemometer to measure wind speed, $v$
        \item  Ensure no other draughts or airflows
        \item Avoid turbulence or reflection of air flow
    \end{itemize}
\end{enumerate}

\begin{flushleft}
\textbf{Light-related Measurements}
\end{flushleft}

\begin{enumerate}
    \item To vary wavelength, $\lambda$ of light
\begin{itemize}
        \item Use laser light/high-intensity LED of different colours
        \item  Wavelength used can be read off data-sheet or label for laser/LED
\end{itemize}
\textbf{OR}
\begin{itemize}
        \item Use a source of white light/lamp
        \item  Use different colour filters to vary the wavelength
        \item Wavelength selected can be read off label/data-sheet of colour filter
(Note: Colour filters cannot be used with laser light/LED, only with white light source)
\end{itemize}
\item To measure wavelength, $\lambda$ of light
\begin{itemize}
    \item Conduct Young’s double slit experiment
    \item $\lambda$ can be found from equation:  $\lambda=\cfrac{ax}{D}$
where $a$–distance between slits, $x$–spacing between adjacent bright fringes and $D$-distance between slit and screen
\end{itemize}
\textbf{OR}
\begin{itemize}
    \item Conduct experiment using a diffraction grating
    \item $\lambda$ can be found from equation: $d\sin \theta=n\lambda$
where $d$–distance between slits, $\theta$–angle measured from zero order, $n$–order
number
    \item  Conduct experiment in a dark room
    \item Use high intensity lamp/LED
    \item  Use collimated beam/laser
\end{itemize}
\end{enumerate}

\begin{flushleft}
\textbf{Safety Precautions:}
Protecting the hand
\end{flushleft}
\begin{itemize}
    \item Wear rubber gloves/insulating gloves to prevent electrocution
    \item Wear heat-proof gloves to protect again hot surfaces/coils
    \item Wear thick gloves to protect against cuts from sharp edges
\end{itemize}

\begin{flushleft}
Protecting the eyes
\end{flushleft}
\begin{itemize}
    \item Do not look directly at bright light source
    \item Wear dark glasses to protect against bright light
    \item Wear safety goggles to protect against any flying objects
\end{itemize}
\begin{flushleft}
Protecting against flying / falling objects
\end{flushleft}
\begin{itemize}
    \item Install safety screen to protect against any flying object
    \item  Put a sand tray/box lined with bubble wrap/cushion below falling object
\end{itemize}
\begin{flushleft}
Protecting from heat
\end{flushleft}
\begin{itemize}
    \item Switch off power supply/current when not taking measurements to prevent coils/component/circuit from heating up
\end{itemize}
\begin{flushleft}
Handling hot objects
\end{flushleft}
\begin{itemize}
    \item Use tongs/gloves
\end{itemize}
% \end{comment}
\end{document}